\documentclass[a4paper,oneside]{book}
\newenvironment{abstract}{}{}
\usepackage{abstract}
\usepackage{noweb}
% Needed to relax penalty for breaking code chunks across pages, otherwise 
% there might be a lot of space following a code chunk.
\def\nwendcode{\endtrivlist \endgroup}
\let\nwdocspar=\smallbreak

\usepackage[hyphens]{url}
\usepackage{hyperref}
\usepackage{authblk}

\submitted{November 2012}
\copyrightyear{2012}
\adviser{Professor John P. Smith}
\department{Computer Science}

\newcommand{\proquestmode}{}

\abstract{
We provide a Python wrapper for the LADOK3 REST API\@.
We provide a useful object-oriented framework and direct API calls that return 
the unprocessed JSON data from LADOK.

}

\acknowledgements{
\input{acknowledgements.inc}
}

\dedication{
\input{dedication.inc}
}

%%%%%%%%%%%%%%%%%%%%%%%%%%%%%%%%%%%%%%%%%%%%%%%%%%%%%%%%%%%%%\
%%%% Tweak float placements
% From: http://mintaka.sdsu.edu/GF/bibliog/latex/floats.html "Controlling LaTeX Floats"
% and based on: http://www.tex.ac.uk/cgi-bin/texfaq2html?label=floats
% LaTeX defaults listed at: http://people.cs.uu.nl/piet/floats/node1.html

% Alter some LaTeX defaults for better treatment of figures:
    % See p.105 of "TeX Unbound" for suggested values.
    % See pp. 199-200 of Lamport's "LaTeX" book for details.
    %   General parameters, for ALL pages:
    \renewcommand{\topfraction}{0.85}   % max fraction of floats at top
    \renewcommand{\bottomfraction}{0.6} % max fraction of floats at bottom
    %   Parameters for TEXT pages (not float pages):
    \setcounter{topnumber}{2}
    \setcounter{bottomnumber}{2}
    \setcounter{totalnumber}{4}     % 2 may work better
    \setcounter{dbltopnumber}{2}    % for 2-column pages
    \renewcommand{\dbltopfraction}{0.66}    % fit big float above 2-col. text
    \renewcommand{\textfraction}{0.15}  % allow minimal text w. figs
    %   Parameters for FLOAT pages (not text pages):
    \renewcommand{\floatpagefraction}{0.66} % require fuller float pages
    % N.B.: floatpagefraction MUST be less than topfraction !!
    \renewcommand{\dblfloatpagefraction}{0.66}  % require fuller float pages


%%%%%%%%%%%%%%%%%%%%%%%%%%%%%%%%%%%%%%%%%%%%%%%%%%%%%%%%%%%%%\
%%%% Use packages

%\usepackage{amsfonts}

%%% For figures
\usepackage{graphicx}
%\usepackage{subfig,rotate}

%%% for comments
\usepackage{verbatim}

% nice references
\usepackage[sort,numbers]{natbib}

% nice description style
\usepackage{enumitem}
\setlist[description]{style=nextline,
                      leftmargin=1.5em,
                      labelindent=1em,
                      topsep=1em,
                      itemsep=0em}

% for subfigure
\usepackage[margin=8pt]{subcaption}
\captionsetup{labelfont=bf}

% for code listings
\usepackage{listings}
\IfFileExists{inconsolata.sty}{\usepackage{inconsolata}}{\usepackage{zi4}}
\usepackage{color}
\definecolor{codebg}{RGB}{248,248,248} % mimics html code style
\definecolor{codeborder}{RGB}{204,204,204}
\lstset{breaklines=true,
        breakatwhitespace=false,
        basicstyle=\singlespacing\footnotesize\ttfamily,
        columns=fixed,
        frame=single,
        rulecolor=\color{codeborder},
        backgroundcolor=\color{codebg}
        }

%%% For tables
\usepackage{multirow}
% Longtable lets you have tables that span multiple pages.
\usepackage{longtable}

% Booktabs produces far nicer tables than the standard LaTeX tables.
%   see: http://en.wikibooks.org/wiki/LaTeX/Tables
\usepackage{booktabs}

% adds more formatting options to tables
\usepackage{array}

%set parameters for longtable:
% default caption width is 4in for longtable, but wider for normal tables
\setlength{\LTcapwidth}{\textwidth}

%%%%%%%%%%%%%%%%%%%%%%%%%%%%%%%%%%%%%%%%%%%%%%%%%%%%%%%%%%
%%% Printed vs. online formatting
\ifdefined\printmode

% Printed copy
% url package understands urls (with proper line-breaks) without hyperlinking them
\usepackage{url}


\else

\ifdefined\proquestmode
%ProQuest copy -- http://www.princeton.edu/~mudd/thesis/Submissionguide.pdf

% ProQuest requires a double spaced version (set previously). They will take an electronic copy, so we want links in the pdf, but also copies may be printed or made into microfilm in black and white, so we want outlined links instead of colored links.
\usepackage{hyperref}
\hypersetup{bookmarksnumbered}

% copy the already-set title and author to use in the pdf properties
\makeatletter
\hypersetup{pdftitle=\@title,pdfauthor=\@author}
\makeatother

\else
% Online copy

% adds internal linked references, pdf bookmarks, etc

% turn all references and citations into hyperlinks:
%  -- not for printed copies
% -- automatically includes url package
% options:
%   colorlinks makes links by coloring the text instead of putting a rectangle around the text.
\usepackage{hyperref}
\hypersetup{colorlinks,bookmarksnumbered}

% copy the already-set title and author to use in the pdf properties
\makeatletter
\hypersetup{pdftitle=\@title,pdfauthor=\@author}
\makeatother

% make the page number rather than the text be the link for ToC entries
%\hypersetup{linktocpage}
\fi % proquest or online formatting
\fi % printed or online formatting


\title{%
  A LADOK3 Python API
}
\author{%
  Alexander Baltatzis,
  Daniel Bosk,
  Gerald Q.\ \enquote{Chip} Maguire Jr.
}
\affil{%
  KTH EECS\\
  \texttt{\{alba,dbosk,maguire\}@kth.se}
}

\begin{document}
\frontmatter
\maketitle

\vspace*{\fill}
\VerbatimInput{../LICENSE}
\clearpage

\begin{abstract}
  We provide a Python wrapper for the LADOK3 REST API\@.
We provide a useful object-oriented framework and direct API calls that return 
the unprocessed JSON data from LADOK.

\end{abstract}
\clearpage

\tableofcontents
\clearpage

\mainmatter
\chapter{Introduction}

LADOK (abbreviation of \foreignlanguage{swedish}{Lokalt Adb–baserat 
DOKumentationssystem}, in Swedish) is the national documentation system for 
higher education in Sweden.
This is the documented source code of \texttt{ladok3}, a LADOK3 API wrapper for 
Python.

The \texttt{ladok3} library provides a non-GUI application that, similar to 
access via a web browser, only provides the user with access to the LADOK data 
and functionality that they would actually have based on their specific user 
permissions in LADOK.
It can be seem as a very streamlined web browser just for LADOK's web 
interface.
While the library exploits caching to reduce the load on the LADOK server, this 
represents a subset of the information that would otherwise be obtained via
LADOK's web GUI export functions.

You can install the \texttt{ladok3} package by running
\begin{minted}{bash}
pip3 install ladok3
\end{minted}
in the terminal.
You can find a quick reference by running
\begin{minted}[firstnumber=last]{bash}
pydoc ladok3
\end{minted}

We provide the main class \texttt{LadokSession} (\cref{LadokSession}).
The \texttt{LadokSession} class acts like \enquote{factories} and will return 
objects representing various LADOK data.
These data objects' classes inherit the \texttt{LadokData} (\cref{LadokData}) 
and \texttt{LadokRemoteData} (\cref{LadokRemoteData}) classes.
Data of the type \texttt{LadokData} is not expected to change, unlike 
\texttt{LadokRemoteData}, which is.
Objects of type \texttt{LadokRemoteData} know how to update themselves, \ie fetch 
and refresh their data from LADOK.
When relevant they can also write data to LADOK, \ie update entries such as 
results.

One design criteria is to improve performance.
We do this by caching all factory methods of any \texttt{LadokSession}.
The \texttt{LadokSession} itself is also designed to be cacheable: if the session to 
LADOK expires, it will automatically reauthenticate to establish a new session.



\part{The library}

\documentclass[a4paper,oneside]{book}
\newenvironment{abstract}{}{}
\usepackage{abstract}
\usepackage{noweb}
% Needed to relax penalty for breaking code chunks across pages, otherwise 
% there might be a lot of space following a code chunk.
\def\nwendcode{\endtrivlist \endgroup}
\let\nwdocspar=\smallbreak

\usepackage[hyphens]{url}
\usepackage{hyperref}
\usepackage{authblk}

\submitted{November 2012}
\copyrightyear{2012}
\adviser{Professor John P. Smith}
\department{Computer Science}

\newcommand{\proquestmode}{}

\abstract{
We provide a Python wrapper for the LADOK3 REST API\@.
We provide a useful object-oriented framework and direct API calls that return 
the unprocessed JSON data from LADOK.

}

\acknowledgements{
\input{acknowledgements.inc}
}

\dedication{
\input{dedication.inc}
}

%%%%%%%%%%%%%%%%%%%%%%%%%%%%%%%%%%%%%%%%%%%%%%%%%%%%%%%%%%%%%\
%%%% Tweak float placements
% From: http://mintaka.sdsu.edu/GF/bibliog/latex/floats.html "Controlling LaTeX Floats"
% and based on: http://www.tex.ac.uk/cgi-bin/texfaq2html?label=floats
% LaTeX defaults listed at: http://people.cs.uu.nl/piet/floats/node1.html

% Alter some LaTeX defaults for better treatment of figures:
    % See p.105 of "TeX Unbound" for suggested values.
    % See pp. 199-200 of Lamport's "LaTeX" book for details.
    %   General parameters, for ALL pages:
    \renewcommand{\topfraction}{0.85}   % max fraction of floats at top
    \renewcommand{\bottomfraction}{0.6} % max fraction of floats at bottom
    %   Parameters for TEXT pages (not float pages):
    \setcounter{topnumber}{2}
    \setcounter{bottomnumber}{2}
    \setcounter{totalnumber}{4}     % 2 may work better
    \setcounter{dbltopnumber}{2}    % for 2-column pages
    \renewcommand{\dbltopfraction}{0.66}    % fit big float above 2-col. text
    \renewcommand{\textfraction}{0.15}  % allow minimal text w. figs
    %   Parameters for FLOAT pages (not text pages):
    \renewcommand{\floatpagefraction}{0.66} % require fuller float pages
    % N.B.: floatpagefraction MUST be less than topfraction !!
    \renewcommand{\dblfloatpagefraction}{0.66}  % require fuller float pages


%%%%%%%%%%%%%%%%%%%%%%%%%%%%%%%%%%%%%%%%%%%%%%%%%%%%%%%%%%%%%\
%%%% Use packages

%\usepackage{amsfonts}

%%% For figures
\usepackage{graphicx}
%\usepackage{subfig,rotate}

%%% for comments
\usepackage{verbatim}

% nice references
\usepackage[sort,numbers]{natbib}

% nice description style
\usepackage{enumitem}
\setlist[description]{style=nextline,
                      leftmargin=1.5em,
                      labelindent=1em,
                      topsep=1em,
                      itemsep=0em}

% for subfigure
\usepackage[margin=8pt]{subcaption}
\captionsetup{labelfont=bf}

% for code listings
\usepackage{listings}
\IfFileExists{inconsolata.sty}{\usepackage{inconsolata}}{\usepackage{zi4}}
\usepackage{color}
\definecolor{codebg}{RGB}{248,248,248} % mimics html code style
\definecolor{codeborder}{RGB}{204,204,204}
\lstset{breaklines=true,
        breakatwhitespace=false,
        basicstyle=\singlespacing\footnotesize\ttfamily,
        columns=fixed,
        frame=single,
        rulecolor=\color{codeborder},
        backgroundcolor=\color{codebg}
        }

%%% For tables
\usepackage{multirow}
% Longtable lets you have tables that span multiple pages.
\usepackage{longtable}

% Booktabs produces far nicer tables than the standard LaTeX tables.
%   see: http://en.wikibooks.org/wiki/LaTeX/Tables
\usepackage{booktabs}

% adds more formatting options to tables
\usepackage{array}

%set parameters for longtable:
% default caption width is 4in for longtable, but wider for normal tables
\setlength{\LTcapwidth}{\textwidth}

%%%%%%%%%%%%%%%%%%%%%%%%%%%%%%%%%%%%%%%%%%%%%%%%%%%%%%%%%%
%%% Printed vs. online formatting
\ifdefined\printmode

% Printed copy
% url package understands urls (with proper line-breaks) without hyperlinking them
\usepackage{url}


\else

\ifdefined\proquestmode
%ProQuest copy -- http://www.princeton.edu/~mudd/thesis/Submissionguide.pdf

% ProQuest requires a double spaced version (set previously). They will take an electronic copy, so we want links in the pdf, but also copies may be printed or made into microfilm in black and white, so we want outlined links instead of colored links.
\usepackage{hyperref}
\hypersetup{bookmarksnumbered}

% copy the already-set title and author to use in the pdf properties
\makeatletter
\hypersetup{pdftitle=\@title,pdfauthor=\@author}
\makeatother

\else
% Online copy

% adds internal linked references, pdf bookmarks, etc

% turn all references and citations into hyperlinks:
%  -- not for printed copies
% -- automatically includes url package
% options:
%   colorlinks makes links by coloring the text instead of putting a rectangle around the text.
\usepackage{hyperref}
\hypersetup{colorlinks,bookmarksnumbered}

% copy the already-set title and author to use in the pdf properties
\makeatletter
\hypersetup{pdftitle=\@title,pdfauthor=\@author}
\makeatother

% make the page number rather than the text be the link for ToC entries
%\hypersetup{linktocpage}
\fi % proquest or online formatting
\fi % printed or online formatting


\title{%
  A LADOK3 Python API
}
\author{%
  Alexander Baltatzis,
  Daniel Bosk,
  Gerald Q.\ \enquote{Chip} Maguire Jr.
}
\affil{%
  KTH EECS\\
  \texttt{\{alba,dbosk,maguire\}@kth.se}
}

\begin{document}
\frontmatter
\maketitle

\vspace*{\fill}
\VerbatimInput{../LICENSE}
\clearpage

\begin{abstract}
  We provide a Python wrapper for the LADOK3 REST API\@.
We provide a useful object-oriented framework and direct API calls that return 
the unprocessed JSON data from LADOK.

\end{abstract}
\clearpage

\tableofcontents
\clearpage

\mainmatter
\chapter{Introduction}

LADOK (abbreviation of \foreignlanguage{swedish}{Lokalt Adb–baserat 
DOKumentationssystem}, in Swedish) is the national documentation system for 
higher education in Sweden.
This is the documented source code of \texttt{ladok3}, a LADOK3 API wrapper for 
Python.

The \texttt{ladok3} library provides a non-GUI application that, similar to 
access via a web browser, only provides the user with access to the LADOK data 
and functionality that they would actually have based on their specific user 
permissions in LADOK.
It can be seem as a very streamlined web browser just for LADOK's web 
interface.
While the library exploits caching to reduce the load on the LADOK server, this 
represents a subset of the information that would otherwise be obtained via
LADOK's web GUI export functions.

You can install the \texttt{ladok3} package by running
\begin{minted}{bash}
pip3 install ladok3
\end{minted}
in the terminal.
You can find a quick reference by running
\begin{minted}[firstnumber=last]{bash}
pydoc ladok3
\end{minted}

We provide the main class \texttt{LadokSession} (\cref{LadokSession}).
The \texttt{LadokSession} class acts like \enquote{factories} and will return 
objects representing various LADOK data.
These data objects' classes inherit the \texttt{LadokData} (\cref{LadokData}) 
and \texttt{LadokRemoteData} (\cref{LadokRemoteData}) classes.
Data of the type \texttt{LadokData} is not expected to change, unlike 
\texttt{LadokRemoteData}, which is.
Objects of type \texttt{LadokRemoteData} know how to update themselves, \ie fetch 
and refresh their data from LADOK.
When relevant they can also write data to LADOK, \ie update entries such as 
results.

One design criteria is to improve performance.
We do this by caching all factory methods of any \texttt{LadokSession}.
The \texttt{LadokSession} itself is also designed to be cacheable: if the session to 
LADOK expires, it will automatically reauthenticate to establish a new session.



\part{The library}

\documentclass[a4paper,oneside]{book}
\newenvironment{abstract}{}{}
\usepackage{abstract}
\usepackage{noweb}
% Needed to relax penalty for breaking code chunks across pages, otherwise 
% there might be a lot of space following a code chunk.
\def\nwendcode{\endtrivlist \endgroup}
\let\nwdocspar=\smallbreak

\usepackage[hyphens]{url}
\usepackage{hyperref}
\usepackage{authblk}

\submitted{November 2012}
\copyrightyear{2012}
\adviser{Professor John P. Smith}
\department{Computer Science}

\newcommand{\proquestmode}{}

\abstract{
\input{abstract.inc}
}

\acknowledgements{
\input{acknowledgements.inc}
}

\dedication{
\input{dedication.inc}
}

%%%%%%%%%%%%%%%%%%%%%%%%%%%%%%%%%%%%%%%%%%%%%%%%%%%%%%%%%%%%%\
%%%% Tweak float placements
% From: http://mintaka.sdsu.edu/GF/bibliog/latex/floats.html "Controlling LaTeX Floats"
% and based on: http://www.tex.ac.uk/cgi-bin/texfaq2html?label=floats
% LaTeX defaults listed at: http://people.cs.uu.nl/piet/floats/node1.html

% Alter some LaTeX defaults for better treatment of figures:
    % See p.105 of "TeX Unbound" for suggested values.
    % See pp. 199-200 of Lamport's "LaTeX" book for details.
    %   General parameters, for ALL pages:
    \renewcommand{\topfraction}{0.85}   % max fraction of floats at top
    \renewcommand{\bottomfraction}{0.6} % max fraction of floats at bottom
    %   Parameters for TEXT pages (not float pages):
    \setcounter{topnumber}{2}
    \setcounter{bottomnumber}{2}
    \setcounter{totalnumber}{4}     % 2 may work better
    \setcounter{dbltopnumber}{2}    % for 2-column pages
    \renewcommand{\dbltopfraction}{0.66}    % fit big float above 2-col. text
    \renewcommand{\textfraction}{0.15}  % allow minimal text w. figs
    %   Parameters for FLOAT pages (not text pages):
    \renewcommand{\floatpagefraction}{0.66} % require fuller float pages
    % N.B.: floatpagefraction MUST be less than topfraction !!
    \renewcommand{\dblfloatpagefraction}{0.66}  % require fuller float pages


%%%%%%%%%%%%%%%%%%%%%%%%%%%%%%%%%%%%%%%%%%%%%%%%%%%%%%%%%%%%%\
%%%% Use packages

%\usepackage{amsfonts}

%%% For figures
\usepackage{graphicx}
%\usepackage{subfig,rotate}

%%% for comments
\usepackage{verbatim}

% nice references
\usepackage[sort,numbers]{natbib}

% nice description style
\usepackage{enumitem}
\setlist[description]{style=nextline,
                      leftmargin=1.5em,
                      labelindent=1em,
                      topsep=1em,
                      itemsep=0em}

% for subfigure
\usepackage[margin=8pt]{subcaption}
\captionsetup{labelfont=bf}

% for code listings
\usepackage{listings}
\IfFileExists{inconsolata.sty}{\usepackage{inconsolata}}{\usepackage{zi4}}
\usepackage{color}
\definecolor{codebg}{RGB}{248,248,248} % mimics html code style
\definecolor{codeborder}{RGB}{204,204,204}
\lstset{breaklines=true,
        breakatwhitespace=false,
        basicstyle=\singlespacing\footnotesize\ttfamily,
        columns=fixed,
        frame=single,
        rulecolor=\color{codeborder},
        backgroundcolor=\color{codebg}
        }

%%% For tables
\usepackage{multirow}
% Longtable lets you have tables that span multiple pages.
\usepackage{longtable}

% Booktabs produces far nicer tables than the standard LaTeX tables.
%   see: http://en.wikibooks.org/wiki/LaTeX/Tables
\usepackage{booktabs}

% adds more formatting options to tables
\usepackage{array}

%set parameters for longtable:
% default caption width is 4in for longtable, but wider for normal tables
\setlength{\LTcapwidth}{\textwidth}

%%%%%%%%%%%%%%%%%%%%%%%%%%%%%%%%%%%%%%%%%%%%%%%%%%%%%%%%%%
%%% Printed vs. online formatting
\ifdefined\printmode

% Printed copy
% url package understands urls (with proper line-breaks) without hyperlinking them
\usepackage{url}


\else

\ifdefined\proquestmode
%ProQuest copy -- http://www.princeton.edu/~mudd/thesis/Submissionguide.pdf

% ProQuest requires a double spaced version (set previously). They will take an electronic copy, so we want links in the pdf, but also copies may be printed or made into microfilm in black and white, so we want outlined links instead of colored links.
\usepackage{hyperref}
\hypersetup{bookmarksnumbered}

% copy the already-set title and author to use in the pdf properties
\makeatletter
\hypersetup{pdftitle=\@title,pdfauthor=\@author}
\makeatother

\else
% Online copy

% adds internal linked references, pdf bookmarks, etc

% turn all references and citations into hyperlinks:
%  -- not for printed copies
% -- automatically includes url package
% options:
%   colorlinks makes links by coloring the text instead of putting a rectangle around the text.
\usepackage{hyperref}
\hypersetup{colorlinks,bookmarksnumbered}

% copy the already-set title and author to use in the pdf properties
\makeatletter
\hypersetup{pdftitle=\@title,pdfauthor=\@author}
\makeatother

% make the page number rather than the text be the link for ToC entries
%\hypersetup{linktocpage}
\fi % proquest or online formatting
\fi % printed or online formatting


\title{%
  A LADOK3 Python API
}
\author{%
  Alexander Baltatzis,
  Daniel Bosk,
  Gerald Q.\ \enquote{Chip} Maguire Jr.
}
\affil{%
  KTH EECS\\
  \texttt{\{alba,dbosk,maguire\}@kth.se}
}

\begin{document}
\frontmatter
\maketitle

\vspace*{\fill}
\VerbatimInput{../LICENSE}
\clearpage

\begin{abstract}
  We provide a Python wrapper for the LADOK3 REST API\@.
We provide a useful object-oriented framework and direct API calls that return 
the unprocessed JSON data from LADOK.

\end{abstract}
\clearpage

\tableofcontents
\clearpage

\mainmatter
\chapter{Introduction}

LADOK (abbreviation of \foreignlanguage{swedish}{Lokalt Adb–baserat 
DOKumentationssystem}, in Swedish) is the national documentation system for 
higher education in Sweden.
This is the documented source code of \texttt{ladok3}, a LADOK3 API wrapper for 
Python.

The \texttt{ladok3} library provides a non-GUI application that, similar to 
access via a web browser, only provides the user with access to the LADOK data 
and functionality that they would actually have based on their specific user 
permissions in LADOK.
It can be seem as a very streamlined web browser just for LADOK's web 
interface.
While the library exploits caching to reduce the load on the LADOK server, this 
represents a subset of the information that would otherwise be obtained via
LADOK's web GUI export functions.

You can install the \texttt{ladok3} package by running
\begin{minted}{bash}
pip3 install ladok3
\end{minted}
in the terminal.
You can find a quick reference by running
\begin{minted}[firstnumber=last]{bash}
pydoc ladok3
\end{minted}

We provide the main class \texttt{LadokSession} (\cref{LadokSession}).
The \texttt{LadokSession} class acts like \enquote{factories} and will return 
objects representing various LADOK data.
These data objects' classes inherit the \texttt{LadokData} (\cref{LadokData}) 
and \texttt{LadokRemoteData} (\cref{LadokRemoteData}) classes.
Data of the type \texttt{LadokData} is not expected to change, unlike 
\texttt{LadokRemoteData}, which is.
Objects of type \texttt{LadokRemoteData} know how to update themselves, \ie fetch 
and refresh their data from LADOK.
When relevant they can also write data to LADOK, \ie update entries such as 
results.

One design criteria is to improve performance.
We do this by caching all factory methods of any \texttt{LadokSession}.
The \texttt{LadokSession} itself is also designed to be cacheable: if the session to 
LADOK expires, it will automatically reauthenticate to establish a new session.



\part{The library}

\documentclass[a4paper,oneside]{book}
\newenvironment{abstract}{}{}
\usepackage{abstract}
\usepackage{noweb}
% Needed to relax penalty for breaking code chunks across pages, otherwise 
% there might be a lot of space following a code chunk.
\def\nwendcode{\endtrivlist \endgroup}
\let\nwdocspar=\smallbreak

\usepackage[hyphens]{url}
\usepackage{hyperref}
\usepackage{authblk}

\input{preamble.tex}

\title{%
  A LADOK3 Python API
}
\author{%
  Alexander Baltatzis,
  Daniel Bosk,
  Gerald Q.\ \enquote{Chip} Maguire Jr.
}
\affil{%
  KTH EECS\\
  \texttt{\{alba,dbosk,maguire\}@kth.se}
}

\begin{document}
\frontmatter
\maketitle

\vspace*{\fill}
\VerbatimInput{../LICENSE}
\clearpage

\begin{abstract}
  \input{abstract.tex}
\end{abstract}
\clearpage

\tableofcontents
\clearpage

\mainmatter
\chapter{Introduction}

LADOK (abbreviation of \foreignlanguage{swedish}{Lokalt Adb–baserat 
DOKumentationssystem}, in Swedish) is the national documentation system for 
higher education in Sweden.
This is the documented source code of \texttt{ladok3}, a LADOK3 API wrapper for 
Python.

The \texttt{ladok3} library provides a non-GUI application that, similar to 
access via a web browser, only provides the user with access to the LADOK data 
and functionality that they would actually have based on their specific user 
permissions in LADOK.
It can be seem as a very streamlined web browser just for LADOK's web 
interface.
While the library exploits caching to reduce the load on the LADOK server, this 
represents a subset of the information that would otherwise be obtained via
LADOK's web GUI export functions.

You can install the \texttt{ladok3} package by running
\begin{minted}{bash}
pip3 install ladok3
\end{minted}
in the terminal.
You can find a quick reference by running
\begin{minted}[firstnumber=last]{bash}
pydoc ladok3
\end{minted}

We provide the main class \texttt{LadokSession} (\cref{LadokSession}).
The \texttt{LadokSession} class acts like \enquote{factories} and will return 
objects representing various LADOK data.
These data objects' classes inherit the \texttt{LadokData} (\cref{LadokData}) 
and \texttt{LadokRemoteData} (\cref{LadokRemoteData}) classes.
Data of the type \texttt{LadokData} is not expected to change, unlike 
\texttt{LadokRemoteData}, which is.
Objects of type \texttt{LadokRemoteData} know how to update themselves, \ie fetch 
and refresh their data from LADOK.
When relevant they can also write data to LADOK, \ie update entries such as 
results.

One design criteria is to improve performance.
We do this by caching all factory methods of any \texttt{LadokSession}.
The \texttt{LadokSession} itself is also designed to be cacheable: if the session to 
LADOK expires, it will automatically reauthenticate to establish a new session.



\part{The library}

\input{../src/ladok3/ladok3.tex}


\part{Logging in at different universities}

\input{../src/ladok3/kth.tex}


\part{API calls}

\input{../src/ladok3/api.tex}
\input{../src/ladok3/undoc.tex}



\part{A command-line interface}

\chapter{The base interface}

\input{../src/ladok3/cli.tex}

\chapter{The \texttt{data} command}

\input{../src/ladok3/data.tex}

\chapter{The \texttt{report} command}

\input{../src/ladok3/report.tex}

\chapter{The \texttt{student} command}

\input{../src/ladok3/student.tex}



\part{Other example applications}

\chapter{Transfer results from KTH Canvas to LADOK}

Here we provide an example program~\texttt{canvas2ladok.py} which exports 
results from KTH Canvas to LADOK for the introductory programming course~prgi 
(DD1315).
You can find an up-to-date version of this chapter at
\begin{center}
  \url{https://github.com/dbosk/intropy/tree/master/adm/reporting}.
\end{center}

\input{../examples/canvas2ladok.tex}


\backmatter
\printbibliography


\end{document}



\part{Logging in at different universities}

\input{../src/ladok3/kth.tex}


\part{API calls}

\input{../src/ladok3/api.tex}
\input{../src/ladok3/undoc.tex}



\part{A command-line interface}

\chapter{The base interface}

\input{../src/ladok3/cli.tex}

\chapter{The \texttt{data} command}

\input{../src/ladok3/data.tex}

\chapter{The \texttt{report} command}

%%%%%%%%%%%%%%%%%%%%%%%%%%%%%%%%%%%%%%%%%
% Stylish Article
% LaTeX Template
% Version 2.0 (13/4/14)
%
% This template has been downloaded from:
% http://www.LaTeXTemplates.com
%
% Original author:
% Mathias Legrand (legrand.mathias@gmail.com)
%
% License:
% CC BY-NC-SA 3.0 (http://creativecommons.org/licenses/by-nc-sa/3.0/)
%
%%%%%%%%%%%%%%%%%%%%%%%%%%%%%%%%%%%%%%%%%

%----------------------------------------------------------------------------------------
%   PACKAGES AND OTHER DOCUMENT CONFIGURATIONS
%----------------------------------------------------------------------------------------

\documentclass[fleqn,10pt]{SelfArx} % Document font size and equations flushed left

\usepackage{xspace}
\usepackage{color}
\usepackage{bm} % bold math

%----------------------------------------------------------------------------------------
%   COLUMNS
%----------------------------------------------------------------------------------------

\setlength{\columnsep}{0.55cm} % Distance between the two columns of text
\setlength{\fboxrule}{0.75pt} % Width of the border around the abstract

%----------------------------------------------------------------------------------------
%   COLORS
%----------------------------------------------------------------------------------------

\definecolor{color1}{RGB}{0,0,90} % Color of the article title and sections
\definecolor{color2}{RGB}{0,20,20} % Color of the boxes behind the abstract and headings

%----------------------------------------------------------------------------------------
%   PLOTTING and DRAWING
%----------------------------------------------------------------------------------------
\usepackage{tikz}
\usetikzlibrary{positioning}
\usetikzlibrary{shapes,arrows,backgrounds,fit,shapes.geometric,calc}
\usetikzlibrary{pgfplots.groupplots}
\usepackage{pgfplots}
\usepackage{pgfplotstable}

%----------------------------------------------------------------------------------------
%   HYPERLINKS
%----------------------------------------------------------------------------------------

\usepackage{hyperref} % Required for hyperlinks
\hypersetup{hidelinks,colorlinks,breaklinks=true,urlcolor=color2,citecolor=color1,linkcolor=color1,bookmarksopen=false,pdftitle={Title},pdfauthor={Author}}

%----------------------------------------------------------------------------------------
%   CUSTOM COMMANDS
%----------------------------------------------------------------------------------------

\newcommand{\todo}[1]{\textbf{\textcolor{blue}{TODO: #1}}} % add a comment to the article
\newcommand{\note}[1]{\textbf{\textcolor{red}{NOTE: #1}}} % add a note to the article

\newcommand{\tbl}[1]{\textbf{Table \ref{#1}}\xspace}
\newcommand{\fig}[1]{\textbf{Figure \ref{#1}}\xspace}
\newcommand{\eq}[1]{\textbf{(\ref{#1})}\xspace}
\newcommand{\ssec}[1]{\textbf{\S\ref{#1}}\xspace}

\newcommand{\deriv}[1]{~\text{d}{#1}}
\newcommand{\pder}[2]{\frac{\partial{#1}}{\partial{#2}}}
\newcommand{\dder}[2]{\frac{\deriv{#1}}{\deriv{#2}}}
\newcommand{\vv}[1]{\bm{#1}\xspace}
\newcommand{\cmi}{c_{m,i}}

\newcommand{\unit}[1]{\left[{#1}\right]}
\newcommand{\txtunit}[1]{$\left[{#1}\right]$}

%----------------------------------------------------------------------------------------
%   ARTICLE INFORMATION
%----------------------------------------------------------------------------------------

\JournalInfo{CSCS}      % name of "journal" on upper right corner
\Archive{working document}

\PaperTitle{Cable Equation}

%===============================================================================

\begin{document}

\flushbottom    % Makes all text pages the same height
\maketitle      % Print the title and authors
\tableofcontents

\thispagestyle{empty} % Removes page numbering from the first page

%------------------------------------------------
\section{Formulation}
\input{formulation.tex}

\section{Appendix}
\input{appendix.tex}

\section{Symbols and Units}
\input{symbols.tex}

%*************************************************
\bibliographystyle{abbrv}
\bibliography{bibliography}
%*************************************************

\end{document}



\chapter{The \texttt{student} command}

\input{../src/ladok3/student.tex}



\part{Other example applications}

\chapter{Transfer results from KTH Canvas to LADOK}

Here we provide an example program~\texttt{canvas2ladok.py} which exports 
results from KTH Canvas to LADOK for the introductory programming course~prgi 
(DD1315).
You can find an up-to-date version of this chapter at
\begin{center}
  \url{https://github.com/dbosk/intropy/tree/master/adm/reporting}.
\end{center}

\input{../examples/canvas2ladok.tex}


\backmatter
\printbibliography


\end{document}



\part{Logging in at different universities}

\input{../src/ladok3/kth.tex}


\part{API calls}

\input{../src/ladok3/api.tex}
\input{../src/ladok3/undoc.tex}



\part{A command-line interface}

\chapter{The base interface}

\input{../src/ladok3/cli.tex}

\chapter{The \texttt{data} command}

\input{../src/ladok3/data.tex}

\chapter{The \texttt{report} command}

%%%%%%%%%%%%%%%%%%%%%%%%%%%%%%%%%%%%%%%%%
% Stylish Article
% LaTeX Template
% Version 2.0 (13/4/14)
%
% This template has been downloaded from:
% http://www.LaTeXTemplates.com
%
% Original author:
% Mathias Legrand (legrand.mathias@gmail.com)
%
% License:
% CC BY-NC-SA 3.0 (http://creativecommons.org/licenses/by-nc-sa/3.0/)
%
%%%%%%%%%%%%%%%%%%%%%%%%%%%%%%%%%%%%%%%%%

%----------------------------------------------------------------------------------------
%   PACKAGES AND OTHER DOCUMENT CONFIGURATIONS
%----------------------------------------------------------------------------------------

\documentclass[fleqn,10pt]{SelfArx} % Document font size and equations flushed left

\usepackage{xspace}
\usepackage{color}
\usepackage{bm} % bold math

%----------------------------------------------------------------------------------------
%   COLUMNS
%----------------------------------------------------------------------------------------

\setlength{\columnsep}{0.55cm} % Distance between the two columns of text
\setlength{\fboxrule}{0.75pt} % Width of the border around the abstract

%----------------------------------------------------------------------------------------
%   COLORS
%----------------------------------------------------------------------------------------

\definecolor{color1}{RGB}{0,0,90} % Color of the article title and sections
\definecolor{color2}{RGB}{0,20,20} % Color of the boxes behind the abstract and headings

%----------------------------------------------------------------------------------------
%   PLOTTING and DRAWING
%----------------------------------------------------------------------------------------
\usepackage{tikz}
\usetikzlibrary{positioning}
\usetikzlibrary{shapes,arrows,backgrounds,fit,shapes.geometric,calc}
\usetikzlibrary{pgfplots.groupplots}
\usepackage{pgfplots}
\usepackage{pgfplotstable}

%----------------------------------------------------------------------------------------
%   HYPERLINKS
%----------------------------------------------------------------------------------------

\usepackage{hyperref} % Required for hyperlinks
\hypersetup{hidelinks,colorlinks,breaklinks=true,urlcolor=color2,citecolor=color1,linkcolor=color1,bookmarksopen=false,pdftitle={Title},pdfauthor={Author}}

%----------------------------------------------------------------------------------------
%   CUSTOM COMMANDS
%----------------------------------------------------------------------------------------

\newcommand{\todo}[1]{\textbf{\textcolor{blue}{TODO: #1}}} % add a comment to the article
\newcommand{\note}[1]{\textbf{\textcolor{red}{NOTE: #1}}} % add a note to the article

\newcommand{\tbl}[1]{\textbf{Table \ref{#1}}\xspace}
\newcommand{\fig}[1]{\textbf{Figure \ref{#1}}\xspace}
\newcommand{\eq}[1]{\textbf{(\ref{#1})}\xspace}
\newcommand{\ssec}[1]{\textbf{\S\ref{#1}}\xspace}

\newcommand{\deriv}[1]{~\text{d}{#1}}
\newcommand{\pder}[2]{\frac{\partial{#1}}{\partial{#2}}}
\newcommand{\dder}[2]{\frac{\deriv{#1}}{\deriv{#2}}}
\newcommand{\vv}[1]{\bm{#1}\xspace}
\newcommand{\cmi}{c_{m,i}}

\newcommand{\unit}[1]{\left[{#1}\right]}
\newcommand{\txtunit}[1]{$\left[{#1}\right]$}

%----------------------------------------------------------------------------------------
%   ARTICLE INFORMATION
%----------------------------------------------------------------------------------------

\JournalInfo{CSCS}      % name of "journal" on upper right corner
\Archive{working document}

\PaperTitle{Cable Equation}

%===============================================================================

\begin{document}

\flushbottom    % Makes all text pages the same height
\maketitle      % Print the title and authors
\tableofcontents

\thispagestyle{empty} % Removes page numbering from the first page

%------------------------------------------------
\section{Formulation}
%%%%%%%%%%%%%%%%%%%%%%%%%%%%%%%%%%%%%%%%%%%%%%%%%%%%%%%%%%%%%%%%%%%%%%%%%%%%%%%%
\subsection{Brief background for derivation of the cable equation}
%%%%%%%%%%%%%%%%%%%%%%%%%%%%%%%%%%%%%%%%%%%%%%%%%%%%%%%%%%%%%%%%%%%%%%%%%%%%%%%%
See \cite{lindsay_2004} for a detailed derivation of the cable equation, and extensions to the one-dimensional model that account for radial variation of potential.

The one-dimensional cable equation introduced later in equations~\eq{eq:cable} and~\eq{eq:cable_balance} is based on the following expression in three dimensions (based on Maxwell's equations adapted for neurological modelling)
\begin{equation}
    \nabla \cdot \vv{J} = 0,
    \label{eq:J}
\end{equation}
where $\vv{J}$ is current density (units $A/m^2$).
Current density is in turn defined in terms of electric field $\vv{E}$ (units $V/m$)
\begin{equation}
    \vv{J} = \sigma \vv{E},
\end{equation}
where $\sigma$ is the specific electrical conductivity of intra-cellular fluid (typically 3.3 $S/m$).

The derivation of the cable equation is based on two assumptions:
\begin{enumerate}
    \item that charge disperion is effectively instantaneous for the purposes of dendritic modelling.
    \item that diffusion of magnetic field is instant, i.e. it behaves quasi-statically in the sense that it is determined by the electric field through the Maxwell equations.
\end{enumerate}
Under these conditions, $\vv{E}$ is conservative, and as such can be expressed in terms of a potential field
\begin{equation}
    \vv{E} = \nabla \phi,
\end{equation}
where the extra/intra-cellular potential field $\phi$ has units $mV$.

The derivation of the one-dimensional conservation equation \eq{eq:cable_balance} is based on the assumption that the intra-cellular potential (i.e. inside the cell) does not vary radially.
That is, potential is a function of the axial distance $x$ alone
\begin{equation}
    \vv{E} = \nabla \phi = \pder{V}{x}.
\end{equation}
This is not strictly true, because a potential field that is a variable of $x$ and $t$ alone can't support the axial gradients required to drive the potential difference over the cell membrane.
I am still trying to get my head around the assumptions made in mapping a three-dimensional problem to a pseudo one-dimensional one.

%%%%%%%%%%%%%%%%%%%%%%%%%%%%%%%%%%%%%%%%%%%%%%%%%%%%%%%%%%%%%%%%%%%%%%%%%%%%%%%%
\subsection{The cable equation}
%%%%%%%%%%%%%%%%%%%%%%%%%%%%%%%%%%%%%%%%%%%%%%%%%%%%%%%%%%%%%%%%%%%%%%%%%%%%%%%%
The cable equation is a nonlinear parabolic PDE that can be written in the form
\begin{equation}
    \label{eq:cable}
    c_m \pder{V}{t} = \frac{1}{2\pi a r_{L}} \pder{}{x} \left( a^2 \pder{V}{x} \right) - i_m + i_e,
\end{equation}
where
\begin{itemize}
    \item $V$ is the potential relative to the ECM $[mV]$
    \item $a$ is the cable radius $(mm)$, and can vary with $x$
    \item $c_m$ is the {specific membrane capacitance}, approximately the same for all neurons $\approx 10~nF/mm^2$. Related to \emph{membrane capacitance} $C_m$ by the relationship $C_m=c_{m}A$, where $A$ is the surface area of the cell.
    \item $i_m$ is the membrane current $[A\cdot/mm^{2}]$ per unit area. The total contribution from ion and synaptic channels is expressed as a the product of current per unit area $i_m$ and the surface area.
    \item $i_e$ is the electrode current flowing into the cell, divided by surface area, i.e. $i_e=I_e/A$.
    \item $r_L$ is intracellular resistivity, typical value $1~k\Omega \text{cm}$
\end{itemize}

Note that the standard convention is followed, whereby membrane and synapse currents ($i_m$) are positive when outward, and electrod currents ($i_e$) are positive inward.

The PDE in (\ref{eq:cable}) is derived by integrating~\eq{eq:J} over the volume of segment $i$:
\begin{equation*}
    \int_{\Omega_i}{\nabla \cdot \vv{J} } \deriv{v} = 0,
\end{equation*}
Then applying the divergence theorem to turn the volume integral on the lhs into a surface integral
\begin{equation}
          \sum_{j\in\mathcal{N}_i} {\int_{\Gamma_{i,j}} J_{i,j} \deriv{s} }
        + \int_{\Gamma_{i}} {J_m} \deriv{s} = 0
    \label{eq:cable_balance_intermediate}
\end{equation}
where
\begin{itemize}
    \item $\int_\Omega \cdot \deriv{v}$ is shorthand for the volume integral over the segment $\Omega_i$
    \item $\int_\Gamma \cdot \deriv{s}$ is shorthand for the surface integral over the surface $\Gamma$
    \item $J_{i,j}=-\frac{1}{r_L}\pder{V}{x} n_{i,j}$ is the flux per unit area of current \emph{from segment $i$ to segment $j$} over the interface $\Gamma_{i,j}$ between the two segments.
    \item the set $\mathcal{N}_i$ is the set of segments that are neighbours of $\Omega_i$
\end{itemize}
The transmembrane current density $J_m=\vv{J}\cdot\vv{n}$ is a function of the membrane potential
\begin{equation}
    J_m = c_m\pder{V}{t} + i_m - i_e,
    \label{eq:Jm}
\end{equation}
which has contributions from the ion channels and synapses ($i_m$), electrodes ($i_e$) and capacitive current due to polarization of the membrane whose bi-layer lipid structure causes it to behave locally like a parallel plate capacitor.

Substituting~\eq{eq:Jm} into~\eq{eq:cable_balance_intermediate} and rearanging gives
\begin{equation}
      \int_{\Gamma_{i}} {c_m\pder{V}{t}} \deriv{s}
    = - \sum_{j\in\mathcal{N}_i} {\int_{\Gamma_{i,j}} J_{i,j} \deriv{s} }
    - \int_{\Gamma_{i}} {(i_m - i_e)} \deriv{s}
    \label{eq:cable_balance}
\end{equation}


The surface of the cable segment is sub-divided into the internal and external surfaces.
The external surface $\Gamma_{i}$ is the cell membrane at the interface between the extra-cellular and intra-cellular regions.
The current, which is the conserved quantity in our conservation law, over the surface is composed of the synapse and ion channel contributions.
This is derived from a thin film approximation to the cell membrane, whereby the membrane is treated as an infinitesimally thin interface between the intra and extra cellular regions.

Note that some information is lost when going from a three-dimensional description of a neuron to a system of branching one-dimensional cable segments.
If the cell is represented by cylinders or frustums\footnote{a frustum is a truncated cone, where the truncation plane is parallel to the base of the cone.}, the three-dimensional values for volume and surface area at branch points can't be retrieved from the one-dimensional description.

\begin{figure}
    \begin{center}
        \includegraphics[width=0.5\textwidth]{./images/cable.pdf}
    \end{center}
    \caption{A single segment, or control volume, on an unbranching dendrite.}
    \label{fig:segment}
\end{figure}

%%%%%%%%%%%%%%%%%%%%%%%%%%%%%%%%%%%%%%%%%%%%%%%%%%%%%%%%%%%%%%%%%%%%%%%%%%%%%%%%
\subsection{Finite volume discretization}
%%%%%%%%%%%%%%%%%%%%%%%%%%%%%%%%%%%%%%%%%%%%%%%%%%%%%%%%%%%%%%%%%%%%%%%%%%%%%%%%
The finite volume method is a natural choice for the solution of the conservation law in~\eq{eq:cable_balance}.

\begin{itemize}
    \item   the $x_i$ are spaced uniformly with distance $x_{i+1}-x_{i} = \Delta x$
    \item   control volumes are formed by locating the boundaries between adjacent points at $(x_{i+1}+x_{i})/2$
    \item   this discretization differs from the finite differences used in Neuron because the equation is explicitly solved for at the end of cable segments, and because the finite volume discretization is applied to all points. Neuron uses special algebraic \emph{zero area} formulation for nodes at branch points.
\end{itemize}

%-------------------------------------------------------------------------------
\subsubsection{Temporal derivative}
%-------------------------------------------------------------------------------
The integral on the lhs of~\eq{eq:cable_balance} can be approximated by assuming that the average transmembrane potential $V$ in $\Omega_i$ is equal to the potential $V_i$ defined at the centre of the segment:
\begin{equation}
    \int_{\Gamma_i}{c_m \pder{V}{t} } \deriv{v} \approx \sigma_i \cmi \pder{V_i}{t},
    \label{eq:dvdt}
\end{equation}
where $\sigma_i$ is the surface area, and $\cmi$ is the average specific membrane capacitance, respectively of the surface $\Gamma_i$.

Each control volume is composed of \emph{sub control volumes}, which are illustrated as the coloured sub-regions in \fig{fig:segment}.
\begin{equation*}
    \Omega_i = \bigcup_{j\in\mathcal{N}_i}{\Omega_i^j}.
\end{equation*}
Likewise, the surface $\Gamma_i$ is composed of subsufaces as follows
\begin{equation*}
    \Gamma_i = \bigcup_{j\in\mathcal{N}_i}{\Gamma_i^j},
\end{equation*}
where $\Gamma_i^j$ is the surface of each of the sub-control volumes in $\Omega_i$.
Thus, the surface area of the CV as
\begin{equation*}
    \sigma_i = \sum_{i\in\mathcal{N}_i}{\sigma_i^j},
\end{equation*}
where $\sigma_i^j$ is the area of $\Gamma_i^j$, and the average specific membrane capacitance $\cmi$ is
\begin{equation*}
    \cmi = \frac{1}{\sigma_i}\sum_{i\in\mathcal{N}_i}{\sigma_i^j c_m^{i,j}}.
\end{equation*}
\todo{This  is included as a placeholder, we really need more illustrations to show how CV averages are computed for quantities that vary between sub-control volumes of the same CV.}

%-------------------------------------------------------------------------------
\subsubsection{Intra-cellular flux}
%-------------------------------------------------------------------------------
The intracellular flux terms in~\eq{eq:cable_balance} are sum of the flux over the interfaces between compartment $i$ and its set of neighbouring compartments $\mathcal{N}_i$ is
\begin{equation}
    \sum_{j\in\mathcal{N}_i} { \int_{\Gamma_{i,j}} { J_{i,j} \deriv{s} } }.
\end{equation}
where the flux per unit area from compartment $i$ to compartment $j$ is
\begin{align}
    J_{i,j} = - \frac{1}{r_L}\pder{V}{x} n_{i,j}.
    \label{eq:J_ij_exact}
\end{align}
The derivative with respect to the outward-facing normal can be approximated as follows
\begin{equation*}
    \pder{V}{x} n_{i,j} \approx \frac{V_j - V_i}{\Delta x_{i,j}}
\end{equation*}
where $\Delta x_{i,j}$ is the distance between $x_i$ and $x_j$, i.e. $\Delta x_{i,j}=|x_i-x_j|$.
Using this approximation for the derivative, the flux over the surface in~\eq{eq:J_ij_exact} is approximated as
\begin{align}
    J_{i,j} \approx \frac{1}{r_L}\frac{V_i - V_j}{\Delta x_{i,j}}.
    \label{eq:J_ij_intermediate}
\end{align}

The terms inside the integral in equation~\eq{eq:J_ij_intermediate} are constant everywhere on the surface $\Gamma_{i,j}$, so the integral becomes
\begin{align}
  J_{i,j} &\approx \int_{\Gamma_{i,j}}  \frac{1}{r_L}\frac{V_i-V_j}{\Delta x_{i,j}} \deriv{s} \nonumber \\
          &= \frac{1}{r_L \Delta x_{i,j}}(V_i-V_j) \int_{\Gamma_{i,j}} 1 \deriv{s} \nonumber \\
          &= \frac{\sigma_{ij}}{r_L \Delta x_{i,j}}(V_i-V_j) \nonumber \\
          \label{eq:J_ij}
\end{align}
where $\sigma_{i,j}=\pi a_{i,j}^2$ is the area of the surface $\Gamma_{i,j}$, which is a circle of radius $a_{i,j}$.

Some symmetries
\begin{itemize}
    \item $\sigma_{i,j}=\sigma_{j,i}$ : surface area of $\Gamma_{i,j}$
    \item $\Delta x_{i,j}=\Delta x_{j,i}$ : distance between $x_i$ and $x_j$
    \item $n_{i,j}=-n_{j,i}$ : surface ``norm''/orientation
    \item $J_{i,j}=n_{j,i}\cdot J_{i,j}=-J_{j,i}$ : charge flux over $\Gamma_{i,j}$
\end{itemize}

%-------------------------------------------------------------------------------
\subsubsection{Cell membrane flux}
%-------------------------------------------------------------------------------
The final term in~\eq{eq:cable_balance} with an integral is the cell membrane flux contribution
\begin{equation}
    \int_{\Gamma_{ext}} {(i_m - i_e)} \deriv{s},
\end{equation}
where the current $i_m$ is due to ion channel and synapses, and $i_e$ is any artificial electrode current.
The $i_m$ term is dependent on the potential difference over the cell membrane $V_i$.
The current terms are an average per unit area, therefore the total flux 
\begin{equation}
    \int_{\Gamma_{ext}} {(i_m - i_e)} \deriv{s}
        \approx
    \sigma_i(i_m(V_i) - i_e(x_i)),
        \label{eq:J_im}
\end{equation}
where $\sigma_i$ is the surface area the of the exterior of the cable segment, i.e. the surface corresponding to the cell membrane.

Each cable segment is a conical frustum, as illustrated in \fig{fig:segment}.
The lateral surface area of a frustum with height $\Delta x_i$ and radii of  is
The area of the external surface $\Gamma_{i}$ is
\begin{equation}
    \sigma_i = \pi (a_{i,\ell} + a_{i,r}) \sqrt{\Delta x_i^2 + (a_{i,\ell} - a_{i,r})^2},
    \label{eq:cv_volume}
\end{equation}
where $a_{i,\ell}$ and $a_{i,r}$ are the radii of at the left and right end of the segment respectively (see~\eq{eq:frustum_area} for derivation of this formula).
%-------------------------------------------------------------------------------
\subsubsection{Putting it all together}
%-------------------------------------------------------------------------------
By substituting the volume averaging of the temporal derivative in~\eq{eq:dvdt} approximations for the flux over the surfaces in~\eq{eq:J_ij} and~\eq{eq:cv_volume} respectively into the conservation equation~\eq{eq:cable_balance} we get the following ODE defined for each node in the cell
\begin{equation}
    \sigma_i \cmi \dder{V_i}{t}
       = -\sum_{j\in\mathcal{N}_i} {\frac{\sigma_{i,j}}{r_L \Delta x_{i,j}} (V_i-V_j)} - \sigma_i\cdot(i_m(V_i) - i_e(x_i)),
    \label{eq:ode}
\end{equation}
where
\begin{equation}
    \sigma_{i,j} = \pi a_{i,j}^2
    \label{eq:sigma_ij}
\end{equation}
is the area of the surface between two adjacent segments $i$ and $j$, and
\begin{equation}
    \sigma_{i}   = \pi(a_{i,\ell} + a_{i,r}) \sqrt{\Delta x_i^2 + (a_{i,\ell} - a_{i,r})^2},
    \label{eq:sigma_i}
\end{equation}
is the lateral area of the conical frustum describing segment $i$.

%-------------------------------------------------------------------------------
\subsubsection{Time Stepping}
%-------------------------------------------------------------------------------
The finite volume discretization approximates spatial derivatives, reducing the original continuous formulation into the set of ODEs, with one ODE for each compartment, in equation~\eq{eq:ode}.
Here we employ an implicit euler temporal integration sheme, wherby the temporal derivative on the lhs is approximated using forward differences
\begin{align}
    \sigma_i c_m \frac{V_i^{k+1}-V_i^{k}}{\Delta t}
        = & -\sum_{j\in\mathcal{N}_i} {\frac{\sigma_{i,j}}{r_L \Delta x_{i,j}} (V_i^{k+1}-V_j^{k+1})} \nonumber \\
          & - \sigma_i\cdot(i_m(V_i^{k}) - i_e),
    \label{eq:ode_subs}
\end{align}
Where $V^k$ is the value of $V$ in compartment $i$ at time step $k$.
Note that on the rhs the value of $V$ at the target time step $k+1$ is used, with the exception of calculating the ion channel and synaptic currents $i_m$.
The current $i_m$ is often a nonlinear function of voltage, so if it was formulated in terms of $V^{k+1}$ the system in~\eq{eq:ode_subs} would be nonlinear, requiring Newton iterations to resolve.

The equations can be rearranged to have all unknown voltage values on the lhs, and values that can be calculated directly on the rhs:
\begin{align}
    & \frac{\sigma_i \cmi}{\Delta t} V_i^{k+1} + \sum_{j\in\mathcal{N}_i} {\alpha_{ij} (V_i^{k+1}-V_j^{k+1})}
            \nonumber \\
    = & \frac{\sigma_i \cmi}{\Delta t} V_i^k -  \sigma_i(i_m^{k} - i_e),
    \label{eq:ode_linsys}
\end{align}
where the value
\begin{equation}
    \alpha_{ij} = \alpha_{ji} = \frac{\sigma_{ij}}{ r_L \Delta x_{ij}}
    \label{eq:alpha_linsys}
\end{equation}
is a constant that can be computed for each interface between adjacent compartments during set up.

The left hand side of \eq{eq:ode_linsys} can be rearranged
\begin{equation}
    \left[ \frac{\sigma_i \cmi}{\Delta t} + \sum_{j\in\mathcal{N}_i} {\alpha_{ij}} \right] V_i^{k+1}
    - \sum_{j\in\mathcal{N}_i} { \alpha_{ij} V_j^{k+1}},
    \label{eq:rhs_linsys}
\end{equation}
which gives the coefficients for the linear system.

The capacitance of the cell membrane, $\sigma_i \cmi$, varies between control volumes, while the $\alpha_{i,j}$ term in symmetric (i.e. $\alpha_{i,j}=\alpha_{j,i}$.)
With this in mind, we can see that when the linear system is written in the form~\eq{eq:ode_linsys}, the matrix is symmetric.
Furthermore, because $\alpha_{i,j} > 0$, the linear system is diagonally dominant for sufficiently small $\Delta t$.

%-------------------------------------------------------------------------------
\subsubsection{The Soma}
%-------------------------------------------------------------------------------
We model the soma as a sphere with a centre $\vv{x}_s$ and a radius $r_s$.
The soma is conceptually treated as the centre of the cell: the point from which all other parts of the cell branch.
It requires special treatment, because of its size relative to the radius of the dendrites and axons that branch off from it.

Though the soma is usually visualized as a sphere, it is \emph{worth noting} that Neuron models the soma as a cylinder
\begin{itemize}
    \item A cylinder that has the same diameter as length ($L=2r$) has the same area as a sphere with radius $r$, i.e. $4\pi r^2$.
    \item However a sphere has 4/3 times the volume of the cylinder.
\end{itemize}

If the soma is modelled as a single compartment the potential is assumed constant throughout the cell.
This is exactly the same model used to handle branches, with the main difference being how the area and volume of the control volume centred on the soma are calculated as a result of the spherical model.

\begin{figure}
    \begin{center}
        \includegraphics[width=0.5\textwidth]{./images/soma.pdf}
    \end{center}
    \caption{A soma with two dendrites.}
    \label{fig:soma}
\end{figure}

\fig{fig:soma} shows a soma that is attached to two unbranched dendrites.
In this example the simplest possible compartment model is used, with three compartments: one for the soma and one for each of the dendrites.
The soma CV extends to half way along each dendrite.
To calculate the flux from the soma compartment to the dendrites the flux over the CV face half way along each dendrite has to be computed.
For this we use the voltage defined at each end of the dendrite, which raises the question about what value to use for the end of the dendrite that is attached to the soma.
For this we use the voltage as defined for the soma, i.e. we use the same value at the start for each dendrite, as illustrated in \fig{fig:soma}.

This effectivly of ``collapses'' the node that defines the start of dendrites attached to the soma onto the soma in the mathematical formulation.
This requires a little slight of hand when loading the cell description from file (for example a \verb!.swc! file).
In such file formats there would be 5 points used to describe the cell in \fig{fig:soma}:
\begin{itemize}
    \item 1 to describe the centre of the soma.
    \item 2 for each dendrite: 1 describing where the dendrite is attached to the soma, and 1 for the terminal end.
\end{itemize}
However, in the mathematical formulation there are only 3 values that are solved for: the soma and one for each of the dendrites.

On a side note, one possible extension is to model the soma as a set of shells, like an onion.
Conceptually this would look like an additional branch from the surface of the soma, with the central core being the terminal of the branch.
However, the cable equation would have to be reformulated in spherical coordinates.


%-------------------------------------------------------------------------------
\subsubsection{Handling Branches}
%-------------------------------------------------------------------------------
The value of the lateral area $\sigma_i$ in~\eq{eq:sigma_i} is the sum of the areaof each branch at branch points.

\todo{a picture of a branching point to illustrate}

\todo{a picture of a soma to illustrate the ball and stick model with a sphere for the soma and sticks for the dendrites branching off the soma.}

\begin{equation}
    \sigma_i = \sum_{j\in\mathcal{N}_i} {\dots}
\end{equation}



\section{Appendix}
%-----------------------------------
\subsection{The conic frustum}
%-----------------------------------
The derivation of the surface area of conic frustum.
The edge length $l$ is defined
\begin{equation}
    l = \sqrt{(x_r - x_l)^2 + (a_r - a_l)^2} = \sqrt{\Delta x^2 + \Delta a^2}.
\end{equation}
The lateral area of the surface is found by integrating along surface of rotation:
\begin{align}
    \sigma_{\text{lateral}}
        &= \int_{0}^{l} {2\pi a(s)} \deriv{s} \nonumber \\
        &= 2\pi \int_{0}^{l} {a_{\ell} + \frac{s}{l}\left( a_r - a_\ell \right)} \deriv{s} \nonumber \\
        &= 2\pi \left[ a_{\ell}s + \frac{s^2}{2l}\left( a_r - a_\ell \right) \right]_0^l \nonumber \\
        &= \pi l \left( a_{\ell} + a_r \right) \nonumber \\
        &= \pi \left( a_{\ell} + a_r \right) \sqrt{\Delta x^2 + \Delta a^2}. \label{eq:frustum_area}
\end{align}

There are two degenerate cases of interest. The first is the \emph{cylinder}, for which the radii at each end are euqal, i.e. $a_\ell = a_r = a$. In this case the lateral area of the surface is
\begin{equation}
    \sigma_{\text{lateral}} = 2\pi a \Delta x.
\end{equation}
The second is a cone, for which $a_\ell=0$ and $a_r=a$:
\begin{equation}
    \sigma_{\text{lateral}} = \pi a \sqrt{\Delta x^2 + a^2}.
\end{equation}


\section{Symbols and Units}
%-------------------------------------------------------------------------------
%\subsubsection{Balancing Units}
%-------------------------------------------------------------------------------
Ensuring that units are balanced and correct requires care.
Take the system of linear equations that arises from the finite volume discretisation, specified in \eq{eq:ode_linsys} and \eq{eq:rhs_linsys}
\begin{equation}
    \label{eq:linsys_FV}
    \left[
        \frac{\sigma_i \cmi}{\Delta t} + \sum_{j\in\mathcal{N}_i} {\alpha_{ij}}
    \right]
    V_i^{k+1} - \sum_{j\in\mathcal{N}_i} { \alpha_{ij} V_j^{k+1}}
        =
    \frac{\sigma_i \cmi}{\Delta t} V_i^k -  \sigma_i(i_m^{k} - i_e).
\end{equation}

The choice of units for a parameter, e.g. $\mu m^2$ or $m^2$ for the area $\sigma_{ij}$, introduces a constant of proportionality wherever it is used ($10^{-12}$ in the case of $\mu m^2 \rightarrow m^2$).
Wherever terms are added in \eq{eq:linsys_FV} the units must be checked, and constants of proportionality balanced.

First, appropriate units for each of the parameters and variables are chosen in~\tbl{tbl:units}.
We try to use the same units as NEURON, except for the specific membrane capacitance $c_m$, for which $F\cdot m^{-2}$ is used in place of $nF\cdot mm^{-2}$.

\begin{table}[hp!]
\begin{tabular}{lllr}
    \hline
    term                      &   units                 &  normalized units                         & NEURON \\
    \hline
    $t$                       &   $ms$                  &  $10^{-3} \cdot s$                        & yes    \\
    $V$                       &   $mV$                  &  $10^{-3} \cdot V$                        & yes    \\
    $a,~\Delta x$             &   $\mu m$               &  $10^{-6} \cdot m$                        & yes    \\
    $\sigma_{i},~\sigma_{ij}$ &   $\mu m^2$             &  $10^{-12} \cdot m^2$                     & yes    \\
    $c_m$                     &   $F\cdot m^{-2}$       &  $s\cdot A\cdot V^{-1}\cdot m^{-2}$       & no     \\
    $r_L$                     &   $\Omega\cdot cm$      &  $10^{-2} \cdot A^{-1}\cdot V\cdot m$     & yes    \\
    $\overline{g}$            &   $S\cdot cm^{-2}$      &  $10^{4} \cdot A\cdot V^{-1}\cdot m^{-2}$ & yes    \\
    $g_s$                     &   $\mu S$               &  $10^{-6} \cdot A\cdot V^{-1}$            & yes    \\
    $I_e$                     &   $nA$                  &  $10^{-9} \cdot A$                        & yes    \\
    \hline
\end{tabular}
\caption{The units chosen for parameters and variables in Arbor. The NEURON column indicates whether the same units have been used as NEURON.}
\label{tbl:units}
\end{table}

\subsection{Left Hand Side}
First, we calculate the units for the term $\frac{\sigma_i \cmi}{\Delta t}$, as follows
\begin{align}
    \left[ \frac{\sigma_i \cmi}{\Delta t} \right]
        &=
    \frac{10^{-12}\cdot m^2 \cdot s\cdot A\cdot V^{-1}\cdot m^{-2} } {10^{-3}\cdot s}
            \nonumber \\
        &=
    10^{-9} \cdot A \cdot V^{-1}. \label{eq:units_lhs_diag}
\end{align}
The units of $\alpha_{i,j}$ are:
\begin{align}
    \left[ \alpha_{i,j} \right]
        &=
    \left[ \frac{\sigma_{ij}}{ r_L \Delta x_{ij}} \right] \nonumber \\
        &=
    \frac{10^{-12}\cdot m^2}{ 10^{-2} \cdot A^{-1}\cdot V\cdot m \cdot 10^{-6}\cdot m}
        \nonumber \\
        &=
    10^{-4}\cdot A \cdot V^{-1}, \label{eq:units_lhs_lu}
\end{align}
which can be scaled by $10^{5}$ to have the same scale as in \eq{eq:units_lhs_diag}.

Thus, the RHS with scaling for units is:
\begin{equation}
    \label{eq:linsys_LHS_scaled}
    \left[
        \frac{\sigma_i \cmi}{\Delta t} + \sum_{j\in\mathcal{N}_i} {10^5\cdot\alpha_{ij}}
    \right]
    V_i^{k+1} - \sum_{j\in\mathcal{N}_i} { 10^5\cdot\alpha_{ij} V_j^{k+1}}.
\end{equation}
The implementation folds the $10^5$ factor into the $\alpha_{ij}$ terms when they are used to calculate the invariant component of the matrix during the initialization phase.

After this scaling, the units of the LHS are
\begin{align}
    \text{units on LHS}
        &= (10^{-9} \cdot A \cdot V^{-1})( 10^{-3} \cdot V) \nonumber \\
        &= 10^{-12} \cdot A. \label{eq:balanced_units}
\end{align}

\subsection{Right Hand Side}
The first term on the RHS has the same scaling factor as the LHS, so does not need to be changed.

Density channels and point processes describe the membrane current differently in NMODL;
as current densities ($10\cdot A\cdot m^{-2}$) and currents ($10^{-9}\cdot A$) respectively.
The current term can be written as follows:
\begin{equation}
    \sigma_i (i_m - i_e) \equiv \sigma_i \bar{i}_m + I_m - I_e,
\end{equation}
where $\bar{i}_m$ is the current density contribution from ion channels, $I_m$ is the current contribution from synapses, and $I_e$ is current contribution from electrodes.

The units of the current density as calculated via NMODL are
\begin{equation}
    \label{eq:im_unit}
    \unit{ \bar{i}_m } =  \unit{ \overline{g} } \unit{ V }
                       =  10 \cdot A \cdot m^{-2},
\end{equation}
so the units of the current from density channels are
\begin{equation}
    \unit{\sigma_i \bar{i}_m} = 10^{-12}\cdot{m}^2 \cdot 10 \cdot A \cdot m^{-2} = 10^{-11}\cdot A.
\end{equation}
Hence, the $\sigma_i \bar{i}_m$ term must be scaled by 10 to match units.

Likewise the units of synapse and electrode current
\begin{equation}
    \label{eq:Im_unit}
    \unit{ I_e } = \unit{ I_m } = \unit{ g_s } \unit{ V }
                 = 10^{-9}\cdot A,
\end{equation}
which must be scaled by $10^3$ to match units.

The properly scaled RHS is
\begin{equation}
    \label{eq:linsys_RHS_scaled}
    \frac{\sigma_i \cmi}{\Delta t} V_i^k -
        (10\cdot\sigma_i \bar{i}_m + 10^3(I_m - I_e)).
\end{equation}

\subsection{Putting It Together}
From \eq{eq:linsys_LHS_scaled} and \eq{eq:linsys_RHS_scaled} the full scaled linear system is
\begin{align}
    &
    \left[
        \frac{\sigma_i \cmi}{\Delta t} + \sum_{j\in\mathcal{N}_i} {10^5\cdot\alpha_{ij}}
    \right]
    V_i^{k+1} - \sum_{j\in\mathcal{N}_i} { 10^5\cdot\alpha_{ij} V_j^{k+1}} \nonumber \\
       & =
    \frac{\sigma_i \cmi}{\Delta t} V_i^k -
        (10\cdot\sigma_i \bar{i}_m + 10^3(I_m - I_e)),
\end{align}
where the units on both sides have been scaled from $pA$ to $nA$, i.e. by a factor of $10^3$.
This can be expressed more generally in terms of weights
\begin{align}
    &
    \left[
        g_i + \sum_{j\in\mathcal{N}_i} {g_{ij}}
    \right]
    V_i^{k+1} - \sum_{j\in\mathcal{N}_i} { g_{ij} V_j^{k+1}} \nonumber \\
       & =
    g_i V_i^k -
        (w_i^d \bar{i}_m + w_i^p(I_m - I_e)),
\end{align}
which can be expressed more compactly as
\begin{equation}
    Gv=i,
\end{equation}
where $G\in\mathbb{R}^{n\times n}$ is the conductance matrix, and $v, i \in \mathbb{R}^{n}$ are voltage and current vectors respectively.

In Arbor the weights are chosen such that the conductance has units $\mu S$, voltage has units $mV$ and current has units $nA$.

    \begin{center}

    \begin{tabular}{llll}
        \hline
        weight & value & units  & SI \\
        \hline
        $g_i$    & $10^{-3}\frac{\sigma_i \cmi}{\Delta t}$ & $\mu S$ & $10^{-6} \cdot A\cdot V^{-1}$ \\
        $g_{ij}$ & $10^2\alpha_{ij}$                       & $\mu S$ & $10^{-6} \cdot A\cdot V^{-1}$ \\
        $w_i^d$  & $10^{-2}\cdot\sigma_i$                  & $10^2\mu m^{-2}$ & $10^{-10}m^2$ \\
        $w_i^p$  & $1$                                     & $1$     & $1$ \\
        \hline
    \end{tabular}

    \end{center}
%------------------------------------------
\subsection{Supplementary Unit Information}
%------------------------------------------
Here is some information about units scraped from Wikipedia for convenience.

\begin{table*}[htp!]
    \begin{center}

    \begin{tabular}{llll}
        \hline
        quality & symbol & unit  & notes \\
        \hline
        energy     & $J$ & joule   $j$  & work to push 1 $N$ through 1 $m$ \\
        charge     & $q$ & coulomb $C$  & $6.25\cdot10^{18}$ electrons, $[A\cdot s]$ \\
        current    & $I$ & ampere  $A$  & $[C\cdot s^{-1}]$, $A$ is SI base unit\\
        voltage    & $V$ & volt    $V$  & potential work per unit charge \\
        resistance & $R$ & ohm $\Omega$ & recall Ohm's law $V=IR$ \\
        capacitance& $C$ & farad   $F$  & $C=\frac{q}{V}$, $[J\cdot C^{2}]$\\
        conductance& $g$ & siemens $S$  & $g=1/R$ i.e. inverse of resistance \\
        \hline
    \end{tabular}

    \vspace{20pt}

    \begin{tabular}{llll}
        \hline
        symbol & unit & equivalents & SI base \\
        \hline
        $J$    & $j$      &  $J\cdot s^{-1}$, $V\cdot A$ &
            $kg\cdot m^{2}\cdot s^{-2}$ \\

        $q$    & $C$      & $s\cdot A$ &
            $s\cdot A$ \\

        $I$    & $A$  & $C\cdot s^{-1}$ &
            $A$ \\

        $V$    & $V$      & $W\cdot A$ &
            $kg\cdot m^{2}\cdot s^{-3}\cdot A^{-1}$ \\

        $R$    & $\Omega$ & $V\cdot A^{-1}$ &
            $kg\cdot m^{2}\cdot s^{-3}\cdot A^{-2}$ \\

        $C$    & $F$      & $C\cdot V^{-1}$  &
            $kg^{-1}\cdot m^{-2}\cdot s^{4}\cdot A^{2}$ \\
        $g$    & $S$      & $A\cdot V^{-1}$  &
            $kg^{-1}\cdot m^{-2}\cdot s^3\cdot A^2$ \\
        \hline
    \end{tabular}

    \end{center}
    \caption{Symbols and quantities.}
\end{table*}



%*************************************************
\bibliographystyle{abbrv}
\bibliography{bibliography}
%*************************************************

\end{document}



\chapter{The \texttt{student} command}

\input{../src/ladok3/student.tex}



\part{Other example applications}

\chapter{Transfer results from KTH Canvas to LADOK}

Here we provide an example program~\texttt{canvas2ladok.py} which exports 
results from KTH Canvas to LADOK for the introductory programming course~prgi 
(DD1315).
You can find an up-to-date version of this chapter at
\begin{center}
  \url{https://github.com/dbosk/intropy/tree/master/adm/reporting}.
\end{center}

\input{../examples/canvas2ladok.tex}


\backmatter
\printbibliography


\end{document}



\part{Logging in at different universities}

\input{../src/ladok3/kth.tex}


\part{API calls}

\input{../src/ladok3/api.tex}
\input{../src/ladok3/undoc.tex}



\part{A command-line interface}

\chapter{The base interface}

\input{../src/ladok3/cli.tex}

\chapter{The \texttt{data} command}

\input{../src/ladok3/data.tex}

\chapter{The \texttt{report} command}

%%%%%%%%%%%%%%%%%%%%%%%%%%%%%%%%%%%%%%%%%
% Stylish Article
% LaTeX Template
% Version 2.0 (13/4/14)
%
% This template has been downloaded from:
% http://www.LaTeXTemplates.com
%
% Original author:
% Mathias Legrand (legrand.mathias@gmail.com)
%
% License:
% CC BY-NC-SA 3.0 (http://creativecommons.org/licenses/by-nc-sa/3.0/)
%
%%%%%%%%%%%%%%%%%%%%%%%%%%%%%%%%%%%%%%%%%

%----------------------------------------------------------------------------------------
%   PACKAGES AND OTHER DOCUMENT CONFIGURATIONS
%----------------------------------------------------------------------------------------

\documentclass[fleqn,10pt]{SelfArx} % Document font size and equations flushed left

\usepackage{xspace}
\usepackage{color}
\usepackage{bm} % bold math

%----------------------------------------------------------------------------------------
%   COLUMNS
%----------------------------------------------------------------------------------------

\setlength{\columnsep}{0.55cm} % Distance between the two columns of text
\setlength{\fboxrule}{0.75pt} % Width of the border around the abstract

%----------------------------------------------------------------------------------------
%   COLORS
%----------------------------------------------------------------------------------------

\definecolor{color1}{RGB}{0,0,90} % Color of the article title and sections
\definecolor{color2}{RGB}{0,20,20} % Color of the boxes behind the abstract and headings

%----------------------------------------------------------------------------------------
%   PLOTTING and DRAWING
%----------------------------------------------------------------------------------------
\usepackage{tikz}
\usetikzlibrary{positioning}
\usetikzlibrary{shapes,arrows,backgrounds,fit,shapes.geometric,calc}
\usetikzlibrary{pgfplots.groupplots}
\usepackage{pgfplots}
\usepackage{pgfplotstable}

%----------------------------------------------------------------------------------------
%   HYPERLINKS
%----------------------------------------------------------------------------------------

\usepackage{hyperref} % Required for hyperlinks
\hypersetup{hidelinks,colorlinks,breaklinks=true,urlcolor=color2,citecolor=color1,linkcolor=color1,bookmarksopen=false,pdftitle={Title},pdfauthor={Author}}

%----------------------------------------------------------------------------------------
%   CUSTOM COMMANDS
%----------------------------------------------------------------------------------------

\newcommand{\todo}[1]{\textbf{\textcolor{blue}{TODO: #1}}} % add a comment to the article
\newcommand{\note}[1]{\textbf{\textcolor{red}{NOTE: #1}}} % add a note to the article

\newcommand{\tbl}[1]{\textbf{Table \ref{#1}}\xspace}
\newcommand{\fig}[1]{\textbf{Figure \ref{#1}}\xspace}
\newcommand{\eq}[1]{\textbf{(\ref{#1})}\xspace}
\newcommand{\ssec}[1]{\textbf{\S\ref{#1}}\xspace}

\newcommand{\deriv}[1]{~\text{d}{#1}}
\newcommand{\pder}[2]{\frac{\partial{#1}}{\partial{#2}}}
\newcommand{\dder}[2]{\frac{\deriv{#1}}{\deriv{#2}}}
\newcommand{\vv}[1]{\bm{#1}\xspace}
\newcommand{\cmi}{c_{m,i}}

\newcommand{\unit}[1]{\left[{#1}\right]}
\newcommand{\txtunit}[1]{$\left[{#1}\right]$}

%----------------------------------------------------------------------------------------
%   ARTICLE INFORMATION
%----------------------------------------------------------------------------------------

\JournalInfo{CSCS}      % name of "journal" on upper right corner
\Archive{working document}

\PaperTitle{Cable Equation}

%===============================================================================

\begin{document}

\flushbottom    % Makes all text pages the same height
\maketitle      % Print the title and authors
\tableofcontents

\thispagestyle{empty} % Removes page numbering from the first page

%------------------------------------------------
\section{Formulation}
%%%%%%%%%%%%%%%%%%%%%%%%%%%%%%%%%%%%%%%%%%%%%%%%%%%%%%%%%%%%%%%%%%%%%%%%%%%%%%%%
\subsection{Brief background for derivation of the cable equation}
%%%%%%%%%%%%%%%%%%%%%%%%%%%%%%%%%%%%%%%%%%%%%%%%%%%%%%%%%%%%%%%%%%%%%%%%%%%%%%%%
See \cite{lindsay_2004} for a detailed derivation of the cable equation, and extensions to the one-dimensional model that account for radial variation of potential.

The one-dimensional cable equation introduced later in equations~\eq{eq:cable} and~\eq{eq:cable_balance} is based on the following expression in three dimensions (based on Maxwell's equations adapted for neurological modelling)
\begin{equation}
    \nabla \cdot \vv{J} = 0,
    \label{eq:J}
\end{equation}
where $\vv{J}$ is current density (units $A/m^2$).
Current density is in turn defined in terms of electric field $\vv{E}$ (units $V/m$)
\begin{equation}
    \vv{J} = \sigma \vv{E},
\end{equation}
where $\sigma$ is the specific electrical conductivity of intra-cellular fluid (typically 3.3 $S/m$).

The derivation of the cable equation is based on two assumptions:
\begin{enumerate}
    \item that charge disperion is effectively instantaneous for the purposes of dendritic modelling.
    \item that diffusion of magnetic field is instant, i.e. it behaves quasi-statically in the sense that it is determined by the electric field through the Maxwell equations.
\end{enumerate}
Under these conditions, $\vv{E}$ is conservative, and as such can be expressed in terms of a potential field
\begin{equation}
    \vv{E} = \nabla \phi,
\end{equation}
where the extra/intra-cellular potential field $\phi$ has units $mV$.

The derivation of the one-dimensional conservation equation \eq{eq:cable_balance} is based on the assumption that the intra-cellular potential (i.e. inside the cell) does not vary radially.
That is, potential is a function of the axial distance $x$ alone
\begin{equation}
    \vv{E} = \nabla \phi = \pder{V}{x}.
\end{equation}
This is not strictly true, because a potential field that is a variable of $x$ and $t$ alone can't support the axial gradients required to drive the potential difference over the cell membrane.
I am still trying to get my head around the assumptions made in mapping a three-dimensional problem to a pseudo one-dimensional one.

%%%%%%%%%%%%%%%%%%%%%%%%%%%%%%%%%%%%%%%%%%%%%%%%%%%%%%%%%%%%%%%%%%%%%%%%%%%%%%%%
\subsection{The cable equation}
%%%%%%%%%%%%%%%%%%%%%%%%%%%%%%%%%%%%%%%%%%%%%%%%%%%%%%%%%%%%%%%%%%%%%%%%%%%%%%%%
The cable equation is a nonlinear parabolic PDE that can be written in the form
\begin{equation}
    \label{eq:cable}
    c_m \pder{V}{t} = \frac{1}{2\pi a r_{L}} \pder{}{x} \left( a^2 \pder{V}{x} \right) - i_m + i_e,
\end{equation}
where
\begin{itemize}
    \item $V$ is the potential relative to the ECM $[mV]$
    \item $a$ is the cable radius $(mm)$, and can vary with $x$
    \item $c_m$ is the {specific membrane capacitance}, approximately the same for all neurons $\approx 10~nF/mm^2$. Related to \emph{membrane capacitance} $C_m$ by the relationship $C_m=c_{m}A$, where $A$ is the surface area of the cell.
    \item $i_m$ is the membrane current $[A\cdot/mm^{2}]$ per unit area. The total contribution from ion and synaptic channels is expressed as a the product of current per unit area $i_m$ and the surface area.
    \item $i_e$ is the electrode current flowing into the cell, divided by surface area, i.e. $i_e=I_e/A$.
    \item $r_L$ is intracellular resistivity, typical value $1~k\Omega \text{cm}$
\end{itemize}

Note that the standard convention is followed, whereby membrane and synapse currents ($i_m$) are positive when outward, and electrod currents ($i_e$) are positive inward.

The PDE in (\ref{eq:cable}) is derived by integrating~\eq{eq:J} over the volume of segment $i$:
\begin{equation*}
    \int_{\Omega_i}{\nabla \cdot \vv{J} } \deriv{v} = 0,
\end{equation*}
Then applying the divergence theorem to turn the volume integral on the lhs into a surface integral
\begin{equation}
          \sum_{j\in\mathcal{N}_i} {\int_{\Gamma_{i,j}} J_{i,j} \deriv{s} }
        + \int_{\Gamma_{i}} {J_m} \deriv{s} = 0
    \label{eq:cable_balance_intermediate}
\end{equation}
where
\begin{itemize}
    \item $\int_\Omega \cdot \deriv{v}$ is shorthand for the volume integral over the segment $\Omega_i$
    \item $\int_\Gamma \cdot \deriv{s}$ is shorthand for the surface integral over the surface $\Gamma$
    \item $J_{i,j}=-\frac{1}{r_L}\pder{V}{x} n_{i,j}$ is the flux per unit area of current \emph{from segment $i$ to segment $j$} over the interface $\Gamma_{i,j}$ between the two segments.
    \item the set $\mathcal{N}_i$ is the set of segments that are neighbours of $\Omega_i$
\end{itemize}
The transmembrane current density $J_m=\vv{J}\cdot\vv{n}$ is a function of the membrane potential
\begin{equation}
    J_m = c_m\pder{V}{t} + i_m - i_e,
    \label{eq:Jm}
\end{equation}
which has contributions from the ion channels and synapses ($i_m$), electrodes ($i_e$) and capacitive current due to polarization of the membrane whose bi-layer lipid structure causes it to behave locally like a parallel plate capacitor.

Substituting~\eq{eq:Jm} into~\eq{eq:cable_balance_intermediate} and rearanging gives
\begin{equation}
      \int_{\Gamma_{i}} {c_m\pder{V}{t}} \deriv{s}
    = - \sum_{j\in\mathcal{N}_i} {\int_{\Gamma_{i,j}} J_{i,j} \deriv{s} }
    - \int_{\Gamma_{i}} {(i_m - i_e)} \deriv{s}
    \label{eq:cable_balance}
\end{equation}


The surface of the cable segment is sub-divided into the internal and external surfaces.
The external surface $\Gamma_{i}$ is the cell membrane at the interface between the extra-cellular and intra-cellular regions.
The current, which is the conserved quantity in our conservation law, over the surface is composed of the synapse and ion channel contributions.
This is derived from a thin film approximation to the cell membrane, whereby the membrane is treated as an infinitesimally thin interface between the intra and extra cellular regions.

Note that some information is lost when going from a three-dimensional description of a neuron to a system of branching one-dimensional cable segments.
If the cell is represented by cylinders or frustums\footnote{a frustum is a truncated cone, where the truncation plane is parallel to the base of the cone.}, the three-dimensional values for volume and surface area at branch points can't be retrieved from the one-dimensional description.

\begin{figure}
    \begin{center}
        \includegraphics[width=0.5\textwidth]{./images/cable.pdf}
    \end{center}
    \caption{A single segment, or control volume, on an unbranching dendrite.}
    \label{fig:segment}
\end{figure}

%%%%%%%%%%%%%%%%%%%%%%%%%%%%%%%%%%%%%%%%%%%%%%%%%%%%%%%%%%%%%%%%%%%%%%%%%%%%%%%%
\subsection{Finite volume discretization}
%%%%%%%%%%%%%%%%%%%%%%%%%%%%%%%%%%%%%%%%%%%%%%%%%%%%%%%%%%%%%%%%%%%%%%%%%%%%%%%%
The finite volume method is a natural choice for the solution of the conservation law in~\eq{eq:cable_balance}.

\begin{itemize}
    \item   the $x_i$ are spaced uniformly with distance $x_{i+1}-x_{i} = \Delta x$
    \item   control volumes are formed by locating the boundaries between adjacent points at $(x_{i+1}+x_{i})/2$
    \item   this discretization differs from the finite differences used in Neuron because the equation is explicitly solved for at the end of cable segments, and because the finite volume discretization is applied to all points. Neuron uses special algebraic \emph{zero area} formulation for nodes at branch points.
\end{itemize}

%-------------------------------------------------------------------------------
\subsubsection{Temporal derivative}
%-------------------------------------------------------------------------------
The integral on the lhs of~\eq{eq:cable_balance} can be approximated by assuming that the average transmembrane potential $V$ in $\Omega_i$ is equal to the potential $V_i$ defined at the centre of the segment:
\begin{equation}
    \int_{\Gamma_i}{c_m \pder{V}{t} } \deriv{v} \approx \sigma_i \cmi \pder{V_i}{t},
    \label{eq:dvdt}
\end{equation}
where $\sigma_i$ is the surface area, and $\cmi$ is the average specific membrane capacitance, respectively of the surface $\Gamma_i$.

Each control volume is composed of \emph{sub control volumes}, which are illustrated as the coloured sub-regions in \fig{fig:segment}.
\begin{equation*}
    \Omega_i = \bigcup_{j\in\mathcal{N}_i}{\Omega_i^j}.
\end{equation*}
Likewise, the surface $\Gamma_i$ is composed of subsufaces as follows
\begin{equation*}
    \Gamma_i = \bigcup_{j\in\mathcal{N}_i}{\Gamma_i^j},
\end{equation*}
where $\Gamma_i^j$ is the surface of each of the sub-control volumes in $\Omega_i$.
Thus, the surface area of the CV as
\begin{equation*}
    \sigma_i = \sum_{i\in\mathcal{N}_i}{\sigma_i^j},
\end{equation*}
where $\sigma_i^j$ is the area of $\Gamma_i^j$, and the average specific membrane capacitance $\cmi$ is
\begin{equation*}
    \cmi = \frac{1}{\sigma_i}\sum_{i\in\mathcal{N}_i}{\sigma_i^j c_m^{i,j}}.
\end{equation*}
\todo{This  is included as a placeholder, we really need more illustrations to show how CV averages are computed for quantities that vary between sub-control volumes of the same CV.}

%-------------------------------------------------------------------------------
\subsubsection{Intra-cellular flux}
%-------------------------------------------------------------------------------
The intracellular flux terms in~\eq{eq:cable_balance} are sum of the flux over the interfaces between compartment $i$ and its set of neighbouring compartments $\mathcal{N}_i$ is
\begin{equation}
    \sum_{j\in\mathcal{N}_i} { \int_{\Gamma_{i,j}} { J_{i,j} \deriv{s} } }.
\end{equation}
where the flux per unit area from compartment $i$ to compartment $j$ is
\begin{align}
    J_{i,j} = - \frac{1}{r_L}\pder{V}{x} n_{i,j}.
    \label{eq:J_ij_exact}
\end{align}
The derivative with respect to the outward-facing normal can be approximated as follows
\begin{equation*}
    \pder{V}{x} n_{i,j} \approx \frac{V_j - V_i}{\Delta x_{i,j}}
\end{equation*}
where $\Delta x_{i,j}$ is the distance between $x_i$ and $x_j$, i.e. $\Delta x_{i,j}=|x_i-x_j|$.
Using this approximation for the derivative, the flux over the surface in~\eq{eq:J_ij_exact} is approximated as
\begin{align}
    J_{i,j} \approx \frac{1}{r_L}\frac{V_i - V_j}{\Delta x_{i,j}}.
    \label{eq:J_ij_intermediate}
\end{align}

The terms inside the integral in equation~\eq{eq:J_ij_intermediate} are constant everywhere on the surface $\Gamma_{i,j}$, so the integral becomes
\begin{align}
  J_{i,j} &\approx \int_{\Gamma_{i,j}}  \frac{1}{r_L}\frac{V_i-V_j}{\Delta x_{i,j}} \deriv{s} \nonumber \\
          &= \frac{1}{r_L \Delta x_{i,j}}(V_i-V_j) \int_{\Gamma_{i,j}} 1 \deriv{s} \nonumber \\
          &= \frac{\sigma_{ij}}{r_L \Delta x_{i,j}}(V_i-V_j) \nonumber \\
          \label{eq:J_ij}
\end{align}
where $\sigma_{i,j}=\pi a_{i,j}^2$ is the area of the surface $\Gamma_{i,j}$, which is a circle of radius $a_{i,j}$.

Some symmetries
\begin{itemize}
    \item $\sigma_{i,j}=\sigma_{j,i}$ : surface area of $\Gamma_{i,j}$
    \item $\Delta x_{i,j}=\Delta x_{j,i}$ : distance between $x_i$ and $x_j$
    \item $n_{i,j}=-n_{j,i}$ : surface ``norm''/orientation
    \item $J_{i,j}=n_{j,i}\cdot J_{i,j}=-J_{j,i}$ : charge flux over $\Gamma_{i,j}$
\end{itemize}

%-------------------------------------------------------------------------------
\subsubsection{Cell membrane flux}
%-------------------------------------------------------------------------------
The final term in~\eq{eq:cable_balance} with an integral is the cell membrane flux contribution
\begin{equation}
    \int_{\Gamma_{ext}} {(i_m - i_e)} \deriv{s},
\end{equation}
where the current $i_m$ is due to ion channel and synapses, and $i_e$ is any artificial electrode current.
The $i_m$ term is dependent on the potential difference over the cell membrane $V_i$.
The current terms are an average per unit area, therefore the total flux 
\begin{equation}
    \int_{\Gamma_{ext}} {(i_m - i_e)} \deriv{s}
        \approx
    \sigma_i(i_m(V_i) - i_e(x_i)),
        \label{eq:J_im}
\end{equation}
where $\sigma_i$ is the surface area the of the exterior of the cable segment, i.e. the surface corresponding to the cell membrane.

Each cable segment is a conical frustum, as illustrated in \fig{fig:segment}.
The lateral surface area of a frustum with height $\Delta x_i$ and radii of  is
The area of the external surface $\Gamma_{i}$ is
\begin{equation}
    \sigma_i = \pi (a_{i,\ell} + a_{i,r}) \sqrt{\Delta x_i^2 + (a_{i,\ell} - a_{i,r})^2},
    \label{eq:cv_volume}
\end{equation}
where $a_{i,\ell}$ and $a_{i,r}$ are the radii of at the left and right end of the segment respectively (see~\eq{eq:frustum_area} for derivation of this formula).
%-------------------------------------------------------------------------------
\subsubsection{Putting it all together}
%-------------------------------------------------------------------------------
By substituting the volume averaging of the temporal derivative in~\eq{eq:dvdt} approximations for the flux over the surfaces in~\eq{eq:J_ij} and~\eq{eq:cv_volume} respectively into the conservation equation~\eq{eq:cable_balance} we get the following ODE defined for each node in the cell
\begin{equation}
    \sigma_i \cmi \dder{V_i}{t}
       = -\sum_{j\in\mathcal{N}_i} {\frac{\sigma_{i,j}}{r_L \Delta x_{i,j}} (V_i-V_j)} - \sigma_i\cdot(i_m(V_i) - i_e(x_i)),
    \label{eq:ode}
\end{equation}
where
\begin{equation}
    \sigma_{i,j} = \pi a_{i,j}^2
    \label{eq:sigma_ij}
\end{equation}
is the area of the surface between two adjacent segments $i$ and $j$, and
\begin{equation}
    \sigma_{i}   = \pi(a_{i,\ell} + a_{i,r}) \sqrt{\Delta x_i^2 + (a_{i,\ell} - a_{i,r})^2},
    \label{eq:sigma_i}
\end{equation}
is the lateral area of the conical frustum describing segment $i$.

%-------------------------------------------------------------------------------
\subsubsection{Time Stepping}
%-------------------------------------------------------------------------------
The finite volume discretization approximates spatial derivatives, reducing the original continuous formulation into the set of ODEs, with one ODE for each compartment, in equation~\eq{eq:ode}.
Here we employ an implicit euler temporal integration sheme, wherby the temporal derivative on the lhs is approximated using forward differences
\begin{align}
    \sigma_i c_m \frac{V_i^{k+1}-V_i^{k}}{\Delta t}
        = & -\sum_{j\in\mathcal{N}_i} {\frac{\sigma_{i,j}}{r_L \Delta x_{i,j}} (V_i^{k+1}-V_j^{k+1})} \nonumber \\
          & - \sigma_i\cdot(i_m(V_i^{k}) - i_e),
    \label{eq:ode_subs}
\end{align}
Where $V^k$ is the value of $V$ in compartment $i$ at time step $k$.
Note that on the rhs the value of $V$ at the target time step $k+1$ is used, with the exception of calculating the ion channel and synaptic currents $i_m$.
The current $i_m$ is often a nonlinear function of voltage, so if it was formulated in terms of $V^{k+1}$ the system in~\eq{eq:ode_subs} would be nonlinear, requiring Newton iterations to resolve.

The equations can be rearranged to have all unknown voltage values on the lhs, and values that can be calculated directly on the rhs:
\begin{align}
    & \frac{\sigma_i \cmi}{\Delta t} V_i^{k+1} + \sum_{j\in\mathcal{N}_i} {\alpha_{ij} (V_i^{k+1}-V_j^{k+1})}
            \nonumber \\
    = & \frac{\sigma_i \cmi}{\Delta t} V_i^k -  \sigma_i(i_m^{k} - i_e),
    \label{eq:ode_linsys}
\end{align}
where the value
\begin{equation}
    \alpha_{ij} = \alpha_{ji} = \frac{\sigma_{ij}}{ r_L \Delta x_{ij}}
    \label{eq:alpha_linsys}
\end{equation}
is a constant that can be computed for each interface between adjacent compartments during set up.

The left hand side of \eq{eq:ode_linsys} can be rearranged
\begin{equation}
    \left[ \frac{\sigma_i \cmi}{\Delta t} + \sum_{j\in\mathcal{N}_i} {\alpha_{ij}} \right] V_i^{k+1}
    - \sum_{j\in\mathcal{N}_i} { \alpha_{ij} V_j^{k+1}},
    \label{eq:rhs_linsys}
\end{equation}
which gives the coefficients for the linear system.

The capacitance of the cell membrane, $\sigma_i \cmi$, varies between control volumes, while the $\alpha_{i,j}$ term in symmetric (i.e. $\alpha_{i,j}=\alpha_{j,i}$.)
With this in mind, we can see that when the linear system is written in the form~\eq{eq:ode_linsys}, the matrix is symmetric.
Furthermore, because $\alpha_{i,j} > 0$, the linear system is diagonally dominant for sufficiently small $\Delta t$.

%-------------------------------------------------------------------------------
\subsubsection{The Soma}
%-------------------------------------------------------------------------------
We model the soma as a sphere with a centre $\vv{x}_s$ and a radius $r_s$.
The soma is conceptually treated as the centre of the cell: the point from which all other parts of the cell branch.
It requires special treatment, because of its size relative to the radius of the dendrites and axons that branch off from it.

Though the soma is usually visualized as a sphere, it is \emph{worth noting} that Neuron models the soma as a cylinder
\begin{itemize}
    \item A cylinder that has the same diameter as length ($L=2r$) has the same area as a sphere with radius $r$, i.e. $4\pi r^2$.
    \item However a sphere has 4/3 times the volume of the cylinder.
\end{itemize}

If the soma is modelled as a single compartment the potential is assumed constant throughout the cell.
This is exactly the same model used to handle branches, with the main difference being how the area and volume of the control volume centred on the soma are calculated as a result of the spherical model.

\begin{figure}
    \begin{center}
        \includegraphics[width=0.5\textwidth]{./images/soma.pdf}
    \end{center}
    \caption{A soma with two dendrites.}
    \label{fig:soma}
\end{figure}

\fig{fig:soma} shows a soma that is attached to two unbranched dendrites.
In this example the simplest possible compartment model is used, with three compartments: one for the soma and one for each of the dendrites.
The soma CV extends to half way along each dendrite.
To calculate the flux from the soma compartment to the dendrites the flux over the CV face half way along each dendrite has to be computed.
For this we use the voltage defined at each end of the dendrite, which raises the question about what value to use for the end of the dendrite that is attached to the soma.
For this we use the voltage as defined for the soma, i.e. we use the same value at the start for each dendrite, as illustrated in \fig{fig:soma}.

This effectivly of ``collapses'' the node that defines the start of dendrites attached to the soma onto the soma in the mathematical formulation.
This requires a little slight of hand when loading the cell description from file (for example a \verb!.swc! file).
In such file formats there would be 5 points used to describe the cell in \fig{fig:soma}:
\begin{itemize}
    \item 1 to describe the centre of the soma.
    \item 2 for each dendrite: 1 describing where the dendrite is attached to the soma, and 1 for the terminal end.
\end{itemize}
However, in the mathematical formulation there are only 3 values that are solved for: the soma and one for each of the dendrites.

On a side note, one possible extension is to model the soma as a set of shells, like an onion.
Conceptually this would look like an additional branch from the surface of the soma, with the central core being the terminal of the branch.
However, the cable equation would have to be reformulated in spherical coordinates.


%-------------------------------------------------------------------------------
\subsubsection{Handling Branches}
%-------------------------------------------------------------------------------
The value of the lateral area $\sigma_i$ in~\eq{eq:sigma_i} is the sum of the areaof each branch at branch points.

\todo{a picture of a branching point to illustrate}

\todo{a picture of a soma to illustrate the ball and stick model with a sphere for the soma and sticks for the dendrites branching off the soma.}

\begin{equation}
    \sigma_i = \sum_{j\in\mathcal{N}_i} {\dots}
\end{equation}



\section{Appendix}
%-----------------------------------
\subsection{The conic frustum}
%-----------------------------------
The derivation of the surface area of conic frustum.
The edge length $l$ is defined
\begin{equation}
    l = \sqrt{(x_r - x_l)^2 + (a_r - a_l)^2} = \sqrt{\Delta x^2 + \Delta a^2}.
\end{equation}
The lateral area of the surface is found by integrating along surface of rotation:
\begin{align}
    \sigma_{\text{lateral}}
        &= \int_{0}^{l} {2\pi a(s)} \deriv{s} \nonumber \\
        &= 2\pi \int_{0}^{l} {a_{\ell} + \frac{s}{l}\left( a_r - a_\ell \right)} \deriv{s} \nonumber \\
        &= 2\pi \left[ a_{\ell}s + \frac{s^2}{2l}\left( a_r - a_\ell \right) \right]_0^l \nonumber \\
        &= \pi l \left( a_{\ell} + a_r \right) \nonumber \\
        &= \pi \left( a_{\ell} + a_r \right) \sqrt{\Delta x^2 + \Delta a^2}. \label{eq:frustum_area}
\end{align}

There are two degenerate cases of interest. The first is the \emph{cylinder}, for which the radii at each end are euqal, i.e. $a_\ell = a_r = a$. In this case the lateral area of the surface is
\begin{equation}
    \sigma_{\text{lateral}} = 2\pi a \Delta x.
\end{equation}
The second is a cone, for which $a_\ell=0$ and $a_r=a$:
\begin{equation}
    \sigma_{\text{lateral}} = \pi a \sqrt{\Delta x^2 + a^2}.
\end{equation}


\section{Symbols and Units}
%-------------------------------------------------------------------------------
%\subsubsection{Balancing Units}
%-------------------------------------------------------------------------------
Ensuring that units are balanced and correct requires care.
Take the system of linear equations that arises from the finite volume discretisation, specified in \eq{eq:ode_linsys} and \eq{eq:rhs_linsys}
\begin{equation}
    \label{eq:linsys_FV}
    \left[
        \frac{\sigma_i \cmi}{\Delta t} + \sum_{j\in\mathcal{N}_i} {\alpha_{ij}}
    \right]
    V_i^{k+1} - \sum_{j\in\mathcal{N}_i} { \alpha_{ij} V_j^{k+1}}
        =
    \frac{\sigma_i \cmi}{\Delta t} V_i^k -  \sigma_i(i_m^{k} - i_e).
\end{equation}

The choice of units for a parameter, e.g. $\mu m^2$ or $m^2$ for the area $\sigma_{ij}$, introduces a constant of proportionality wherever it is used ($10^{-12}$ in the case of $\mu m^2 \rightarrow m^2$).
Wherever terms are added in \eq{eq:linsys_FV} the units must be checked, and constants of proportionality balanced.

First, appropriate units for each of the parameters and variables are chosen in~\tbl{tbl:units}.
We try to use the same units as NEURON, except for the specific membrane capacitance $c_m$, for which $F\cdot m^{-2}$ is used in place of $nF\cdot mm^{-2}$.

\begin{table}[hp!]
\begin{tabular}{lllr}
    \hline
    term                      &   units                 &  normalized units                         & NEURON \\
    \hline
    $t$                       &   $ms$                  &  $10^{-3} \cdot s$                        & yes    \\
    $V$                       &   $mV$                  &  $10^{-3} \cdot V$                        & yes    \\
    $a,~\Delta x$             &   $\mu m$               &  $10^{-6} \cdot m$                        & yes    \\
    $\sigma_{i},~\sigma_{ij}$ &   $\mu m^2$             &  $10^{-12} \cdot m^2$                     & yes    \\
    $c_m$                     &   $F\cdot m^{-2}$       &  $s\cdot A\cdot V^{-1}\cdot m^{-2}$       & no     \\
    $r_L$                     &   $\Omega\cdot cm$      &  $10^{-2} \cdot A^{-1}\cdot V\cdot m$     & yes    \\
    $\overline{g}$            &   $S\cdot cm^{-2}$      &  $10^{4} \cdot A\cdot V^{-1}\cdot m^{-2}$ & yes    \\
    $g_s$                     &   $\mu S$               &  $10^{-6} \cdot A\cdot V^{-1}$            & yes    \\
    $I_e$                     &   $nA$                  &  $10^{-9} \cdot A$                        & yes    \\
    \hline
\end{tabular}
\caption{The units chosen for parameters and variables in Arbor. The NEURON column indicates whether the same units have been used as NEURON.}
\label{tbl:units}
\end{table}

\subsection{Left Hand Side}
First, we calculate the units for the term $\frac{\sigma_i \cmi}{\Delta t}$, as follows
\begin{align}
    \left[ \frac{\sigma_i \cmi}{\Delta t} \right]
        &=
    \frac{10^{-12}\cdot m^2 \cdot s\cdot A\cdot V^{-1}\cdot m^{-2} } {10^{-3}\cdot s}
            \nonumber \\
        &=
    10^{-9} \cdot A \cdot V^{-1}. \label{eq:units_lhs_diag}
\end{align}
The units of $\alpha_{i,j}$ are:
\begin{align}
    \left[ \alpha_{i,j} \right]
        &=
    \left[ \frac{\sigma_{ij}}{ r_L \Delta x_{ij}} \right] \nonumber \\
        &=
    \frac{10^{-12}\cdot m^2}{ 10^{-2} \cdot A^{-1}\cdot V\cdot m \cdot 10^{-6}\cdot m}
        \nonumber \\
        &=
    10^{-4}\cdot A \cdot V^{-1}, \label{eq:units_lhs_lu}
\end{align}
which can be scaled by $10^{5}$ to have the same scale as in \eq{eq:units_lhs_diag}.

Thus, the RHS with scaling for units is:
\begin{equation}
    \label{eq:linsys_LHS_scaled}
    \left[
        \frac{\sigma_i \cmi}{\Delta t} + \sum_{j\in\mathcal{N}_i} {10^5\cdot\alpha_{ij}}
    \right]
    V_i^{k+1} - \sum_{j\in\mathcal{N}_i} { 10^5\cdot\alpha_{ij} V_j^{k+1}}.
\end{equation}
The implementation folds the $10^5$ factor into the $\alpha_{ij}$ terms when they are used to calculate the invariant component of the matrix during the initialization phase.

After this scaling, the units of the LHS are
\begin{align}
    \text{units on LHS}
        &= (10^{-9} \cdot A \cdot V^{-1})( 10^{-3} \cdot V) \nonumber \\
        &= 10^{-12} \cdot A. \label{eq:balanced_units}
\end{align}

\subsection{Right Hand Side}
The first term on the RHS has the same scaling factor as the LHS, so does not need to be changed.

Density channels and point processes describe the membrane current differently in NMODL;
as current densities ($10\cdot A\cdot m^{-2}$) and currents ($10^{-9}\cdot A$) respectively.
The current term can be written as follows:
\begin{equation}
    \sigma_i (i_m - i_e) \equiv \sigma_i \bar{i}_m + I_m - I_e,
\end{equation}
where $\bar{i}_m$ is the current density contribution from ion channels, $I_m$ is the current contribution from synapses, and $I_e$ is current contribution from electrodes.

The units of the current density as calculated via NMODL are
\begin{equation}
    \label{eq:im_unit}
    \unit{ \bar{i}_m } =  \unit{ \overline{g} } \unit{ V }
                       =  10 \cdot A \cdot m^{-2},
\end{equation}
so the units of the current from density channels are
\begin{equation}
    \unit{\sigma_i \bar{i}_m} = 10^{-12}\cdot{m}^2 \cdot 10 \cdot A \cdot m^{-2} = 10^{-11}\cdot A.
\end{equation}
Hence, the $\sigma_i \bar{i}_m$ term must be scaled by 10 to match units.

Likewise the units of synapse and electrode current
\begin{equation}
    \label{eq:Im_unit}
    \unit{ I_e } = \unit{ I_m } = \unit{ g_s } \unit{ V }
                 = 10^{-9}\cdot A,
\end{equation}
which must be scaled by $10^3$ to match units.

The properly scaled RHS is
\begin{equation}
    \label{eq:linsys_RHS_scaled}
    \frac{\sigma_i \cmi}{\Delta t} V_i^k -
        (10\cdot\sigma_i \bar{i}_m + 10^3(I_m - I_e)).
\end{equation}

\subsection{Putting It Together}
From \eq{eq:linsys_LHS_scaled} and \eq{eq:linsys_RHS_scaled} the full scaled linear system is
\begin{align}
    &
    \left[
        \frac{\sigma_i \cmi}{\Delta t} + \sum_{j\in\mathcal{N}_i} {10^5\cdot\alpha_{ij}}
    \right]
    V_i^{k+1} - \sum_{j\in\mathcal{N}_i} { 10^5\cdot\alpha_{ij} V_j^{k+1}} \nonumber \\
       & =
    \frac{\sigma_i \cmi}{\Delta t} V_i^k -
        (10\cdot\sigma_i \bar{i}_m + 10^3(I_m - I_e)),
\end{align}
where the units on both sides have been scaled from $pA$ to $nA$, i.e. by a factor of $10^3$.
This can be expressed more generally in terms of weights
\begin{align}
    &
    \left[
        g_i + \sum_{j\in\mathcal{N}_i} {g_{ij}}
    \right]
    V_i^{k+1} - \sum_{j\in\mathcal{N}_i} { g_{ij} V_j^{k+1}} \nonumber \\
       & =
    g_i V_i^k -
        (w_i^d \bar{i}_m + w_i^p(I_m - I_e)),
\end{align}
which can be expressed more compactly as
\begin{equation}
    Gv=i,
\end{equation}
where $G\in\mathbb{R}^{n\times n}$ is the conductance matrix, and $v, i \in \mathbb{R}^{n}$ are voltage and current vectors respectively.

In Arbor the weights are chosen such that the conductance has units $\mu S$, voltage has units $mV$ and current has units $nA$.

    \begin{center}

    \begin{tabular}{llll}
        \hline
        weight & value & units  & SI \\
        \hline
        $g_i$    & $10^{-3}\frac{\sigma_i \cmi}{\Delta t}$ & $\mu S$ & $10^{-6} \cdot A\cdot V^{-1}$ \\
        $g_{ij}$ & $10^2\alpha_{ij}$                       & $\mu S$ & $10^{-6} \cdot A\cdot V^{-1}$ \\
        $w_i^d$  & $10^{-2}\cdot\sigma_i$                  & $10^2\mu m^{-2}$ & $10^{-10}m^2$ \\
        $w_i^p$  & $1$                                     & $1$     & $1$ \\
        \hline
    \end{tabular}

    \end{center}
%------------------------------------------
\subsection{Supplementary Unit Information}
%------------------------------------------
Here is some information about units scraped from Wikipedia for convenience.

\begin{table*}[htp!]
    \begin{center}

    \begin{tabular}{llll}
        \hline
        quality & symbol & unit  & notes \\
        \hline
        energy     & $J$ & joule   $j$  & work to push 1 $N$ through 1 $m$ \\
        charge     & $q$ & coulomb $C$  & $6.25\cdot10^{18}$ electrons, $[A\cdot s]$ \\
        current    & $I$ & ampere  $A$  & $[C\cdot s^{-1}]$, $A$ is SI base unit\\
        voltage    & $V$ & volt    $V$  & potential work per unit charge \\
        resistance & $R$ & ohm $\Omega$ & recall Ohm's law $V=IR$ \\
        capacitance& $C$ & farad   $F$  & $C=\frac{q}{V}$, $[J\cdot C^{2}]$\\
        conductance& $g$ & siemens $S$  & $g=1/R$ i.e. inverse of resistance \\
        \hline
    \end{tabular}

    \vspace{20pt}

    \begin{tabular}{llll}
        \hline
        symbol & unit & equivalents & SI base \\
        \hline
        $J$    & $j$      &  $J\cdot s^{-1}$, $V\cdot A$ &
            $kg\cdot m^{2}\cdot s^{-2}$ \\

        $q$    & $C$      & $s\cdot A$ &
            $s\cdot A$ \\

        $I$    & $A$  & $C\cdot s^{-1}$ &
            $A$ \\

        $V$    & $V$      & $W\cdot A$ &
            $kg\cdot m^{2}\cdot s^{-3}\cdot A^{-1}$ \\

        $R$    & $\Omega$ & $V\cdot A^{-1}$ &
            $kg\cdot m^{2}\cdot s^{-3}\cdot A^{-2}$ \\

        $C$    & $F$      & $C\cdot V^{-1}$  &
            $kg^{-1}\cdot m^{-2}\cdot s^{4}\cdot A^{2}$ \\
        $g$    & $S$      & $A\cdot V^{-1}$  &
            $kg^{-1}\cdot m^{-2}\cdot s^3\cdot A^2$ \\
        \hline
    \end{tabular}

    \end{center}
    \caption{Symbols and quantities.}
\end{table*}



%*************************************************
\bibliographystyle{abbrv}
\bibliography{bibliography}
%*************************************************

\end{document}



\chapter{The \texttt{student} command}

\input{../src/ladok3/student.tex}



\part{Other example applications}

\chapter{Transfer results from KTH Canvas to LADOK}

Here we provide an example program~\texttt{canvas2ladok.py} which exports 
results from KTH Canvas to LADOK for the introductory programming course~prgi 
(DD1315).
You can find an up-to-date version of this chapter at
\begin{center}
  \url{https://github.com/dbosk/intropy/tree/master/adm/reporting}.
\end{center}

\input{../examples/canvas2ladok.tex}


\backmatter
\printbibliography


\end{document}
