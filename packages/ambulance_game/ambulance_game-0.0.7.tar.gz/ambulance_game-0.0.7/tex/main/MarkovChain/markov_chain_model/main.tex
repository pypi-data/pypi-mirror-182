\section{Markov chain model}
The following Markov chain represents the transition between states of a service
centre 
while capturing the interactions between it and a buffer centre.
The service centre accepts two types of individuals; Class 1 and Class 2.  
Class 2 individuals are accepted until a pre-determined threshold \(T\) of 
individuals is reached.
When reached, all Class 2 individuals that arrive will remain \textit{``blocked''}
in the buffer centre until the number of people in the system is 
reduced below \(T\). 
Additionally, if the people in the service centre keep rising, they may exceed
the number 
of servers \(C\) available, which will in turn mean that every new person will 
have to wait for a server to become free. 
The states of the Markov chain are denoted by \((u,v)\) where:

\begin{itemize}
    \item \(u\) = number of Class 2 individuals blocked
    \item \(v\) = number of Class 1 individuals in the service centre
\end{itemize}

\begin{figure}
    \centering
    \begin{tikzpicture}[-, node distance = 0.9cm, auto, every node/.style={scale=0.7}]

        % Markov chain variables
        \tikzmath{
            let \initdist = 0.5cm;
            let \altdist = 1.2cm;
            let \minsz = 1.6cm;
        }

        % S_1 and S_2 rectangles
        \tikzmath{
            let \leftOne = -0.8;
            let \rightOne = 2.7;
            let \upOne = 0.8;
            let \downOne = -2.7;
            let \leftTwo = 2.8;
            let \rightTwo = 13;
            let \upTwo = -2.95;
            let \downTwo = -16.4;
        }

        % General case variables
        \tikzmath{
            let \GCsmallx = 8.3;
            let \GCsmally = -9.5;
            let \GCbigx = 4.1;
            let \GCbigy = -11.8;
        }

        % Rectangle for S1
        \draw[ultra thin, dashed] (\leftOne, \downOne) -- (\leftOne, \upOne);
        \draw[ultra thin, dashed] (\leftOne, \upOne) -- (\rightOne, \upOne);
        \draw[ultra thin, dashed] (\rightOne, \upOne) -- node 
        {\Huge{\( \quad S_1 \)}}(\rightOne, \downOne);
        \draw[ultra thin, dashed] (\rightOne, \downOne) -- (\leftOne, \downOne);

        % Rectangle for S2
        \draw[ultra thin, dashed] (\leftTwo, \downTwo) -- node 
        {\Huge{\( S_2 \quad \)}}(\leftTwo, \upTwo);
        \draw[ultra thin, dashed] (\leftTwo, \upTwo) -- (\rightTwo, \upTwo);
        \draw[ultra thin, dashed] (\rightTwo, \upTwo) -- (\rightTwo, \downTwo);
        \draw[ultra thin, dashed] (\rightTwo, \downTwo) -- (\leftTwo, \downTwo);

        % Small square of general case
        \draw [thick] (\GCsmallx, \GCsmally) -- node {} 
        (\GCsmallx + 0.4, \GCsmally);
        \draw [thick] (\GCsmallx + 0.4, \GCsmally) -- node {} 
        (\GCsmallx + 0.4, \GCsmally - 0.4);
        \draw [thick] (\GCsmallx + 0.4, \GCsmally - 0.4) -- node {} 
        (\GCsmallx, \GCsmally - 0.4);
        \draw [thick] (\GCsmallx, \GCsmally - 0.4) -- node {} 
        (\GCsmallx, \GCsmally);


        % Dashed lines to from small square to big one 
        \draw [ultra thin] (\GCsmallx, \GCsmally) -- node {} 
        (\GCbigx, \GCbigy);
        \draw [ultra thin] (\GCsmallx + 0.4, \GCsmally) -- node {} 
        (\GCbigx + 4, \GCbigy);
        \draw [ultra thin] (\GCsmallx, \GCsmally - 0.4) -- node {} (7, \GCbigy);
        \draw [ultra thin] (\GCsmallx + 0.4, \GCsmally - 0.4) -- node {} 
        (\GCbigx + 4, \GCbigy - 4);
        
        % Big Square of general case
        \draw [ultra thick] (\GCbigx, \GCbigy) -- node {} (\GCbigx + 4, \GCbigy);
        \draw [ultra thick] (\GCbigx + 4, \GCbigy) -- node {} 
        (\GCbigx + 4, \GCbigy - 4);
        \draw [ultra thick] (\GCbigx + 4, \GCbigy - 4) -- node {General Case} 
        (\GCbigx, \GCbigy - 4);
        \draw [ultra thick] (\GCbigx, \GCbigy - 4) -- node {} (\GCbigx, \GCbigy);

        % First Line
        \node[state, minimum size=1.5cm] (zero) {(0,0)};
        \node[state, node distance = \initdist, minimum size=\minsz, below right=of zero] 
        (one) {(0,1)};
        \node[draw=none, node distance = \initdist, minimum size=\minsz, below right=of one] 
        (two) {\textbf{\( \ddots \)}};
        \node[state, node distance = \initdist, minimum size=\minsz, below right=of two] 
        (three) {(0,T)};
        \node[state, node distance = \altdist, minimum size=\minsz, right=of three] 
        (four) {(0,T+1)};
        \node[draw=none, node distance = \altdist, minimum size=\minsz, right=of four] 
        (five) {\textbf{\dots}};
        \node[state, minimum size=\minsz, right=of five] (six) {(0,C)};
        \node[draw=none, minimum size=\minsz, right=of six] (seven) {\textbf{\dots}};

        % Second Line
        \node[state, minimum size=\minsz, below=of three] (three_one) {(1,T)};
        \node[state, minimum size=\minsz, below=of four] (four_one) {(1,T+1)};
        \node[draw=none, minimum size=\minsz, below=of five] (five_one) {\textbf{\dots}};
        \node[state, minimum size=\minsz, right=of five_one] (six_one) {(1,C)};
        \node[draw=none, minimum size=\minsz, right=of six_one] (seven_one) {\textbf{\dots}};
        
        % Third Line
        \node[state, minimum size=\minsz, below=of three_one] (three_two) {(2,T)};
        \node[state, minimum size=\minsz, below=of four_one] (four_two) {(2,T+1)};
        \node[draw=none, minimum size=\minsz, below=of five_one] (five_two) 
        {\textbf{\dots}};
        \node[state, minimum size=\minsz, right=of five_two] (six_two) {(2,C)};
        \node[draw=none, minimum size=\minsz, right=of six_two] (seven_two) 
        {\textbf{\dots}};

        % Fourth line
        \node[draw=none, node distance = \altdist, minimum size=\minsz, below=of three_two] 
        (three_three) {\textbf{\vdots}};
        \node[draw=none, node distance = \altdist, minimum size=\minsz, below=of four_two] 
        (four_three) {\textbf{\vdots}};
        \node[draw=none, node distance = 2cm, minimum size=\minsz, below=of five_two] 
        (five_three) {};
        \node[draw=none, node distance = \altdist, minimum size=\minsz, below=of six_two] 
        (six_three) {\textbf{\vdots}};

        % Fifth line
        % \node[state, node distance = \altdist, minimum size=\minsz, below=of five_three] 
        % (general_case_mid) {\( (u_i, v_i) \)};
        \node[draw=none, node distance = 0.3cm, minimum size=\minsz, below=of four_three] 
        (general_case_up) {};
        \node[state, node distance = \altdist, minimum size=\minsz, below=of general_case_up] 
        (general_case_mid) {\( (u_i, v_i) \)};

        \node[draw=none, node distance = \altdist, minimum size=\minsz, below=of general_case_mid] 
        (general_case_down) {};
        \node[draw=none, node distance = \altdist, minimum size=\minsz, left=of general_case_mid] 
        (general_case_left) {};
        \node[draw=none, node distance = \altdist, minimum size=\minsz, right=of general_case_mid] 
        (general_case_right) {};

        \draw[every loop]
            % First Horizontal Edges
            (zero) edge[bend left] node {\( \Lambda \)} (one)
            (one) edge[bend left] node {\( \mu \)} (zero)
            (one) edge[bend left] node {\( \Lambda \)} (two)
            (two) edge[bend left] node {\( 2 \mu \)} (one)
            (two) edge[bend left] node {\( \Lambda \)} (three)
            (three) edge[bend left] node {\( T \mu \)} (two)
            (three) edge[bend left] node {\( \lambda_1 \)} (four)
            (four) edge[bend left] node {\( (T+1) \mu \)} (three)
            (four) edge[bend left] node {\( \lambda_1 \)} (five)
            (five) edge[bend left] node {\( (T+2) \mu \)} (four)
            (five) edge[bend left] node {\( \lambda_1 \)} (six)
            (six) edge[bend left] node {\( C\mu \)} (five)
            (six) edge[bend left] node {\( \lambda_1 \)} (seven)
            (seven) edge[bend left] node {\( C\mu \)} (six)

            % Second Horizontal Edges
            (three_one) edge[bend left] node {\( \lambda_1 \)} (four_one)
            (four_one) edge[bend left] node {\( (T+1) \mu \)} (three_one)
            (four_one) edge[bend left] node {\( \lambda_1 \)} (five_one)
            (five_one) edge[bend left] node {\( (T+2) \mu \)} (four_one)
            (five_one) edge[bend left] node {\( \lambda_1 \)} (six_one)
            (six_one) edge[bend left] node {\( C\mu \)} (five_one)
            (six_one) edge[bend left] node {\( \lambda_1 \)} (seven_one)
            (seven_one) edge[bend left] node {\( C\mu \)} (six_one)

            % Third Horizontal Edges
            (three_two) edge[bend left] node {\( \lambda_1 \)} (four_two)
            (four_two) edge[bend left] node [below] {\( (T+1) \mu \)} (three_two)
            (four_two) edge[bend left] node {\( \lambda_1 \)} (five_two)
            (five_two) edge[bend left] node {\( (T+2) \mu \)} (four_two)
            (five_two) edge[bend left] node {\( \lambda_1 \)} (six_two)
            (six_two) edge[bend left] node {\( C\mu \)} (five_two)
            (six_two) edge[bend left] node {\( \lambda_1 \)} (seven_two)
            (seven_two) edge[bend left] node {\( C\mu \)} (six_two)

            % First Vertical Edges
            (three) edge[bend left] node {\( \lambda_2 \)} (three_one)
            (three_one) edge[bend left] node {\( T \mu \)} (three)
            (three_one) edge[bend left] node {\( \lambda_2 \)} (three_two)
            (three_two) edge[bend left] node {\( T\mu \)} (three_one)
            (three_two) edge[bend left] node {\( \lambda_2 \)} (three_three)
            (three_three) edge[bend left] node {\( T\mu \)} (three_two)

            % Second Vertical Edges
            (four) edge node {\( \lambda_2 \)} (four_one)
            (four_one) edge node {\( \lambda_2 \)} (four_two)
            (four_two) edge node {\( \lambda_2 \)} (four_three)

            % Fourth Vertical Edges
            (six) edge node {\( \lambda_2 \)} (six_one)
            (six_one) edge node {\( \lambda_2 \)} (six_two)
            (six_two) edge node {\( \lambda_2 \)} (six_three)

            % General Case
            (general_case_left) edge[bend left] node {\( \lambda_1 \)} (general_case_mid)
            (general_case_mid) edge[bend left] node {\( v_i \mu \)} (general_case_left)
            (general_case_right) edge[bend left] node {\( (v_i +1) \mu \)} (general_case_mid)
            (general_case_mid) edge[bend left] node {\( \lambda_1 \)} (general_case_right)
            % (five_three) edge node {\( \lambda_2 \)} (general_case_mid)
            (general_case_up) edge node {\( \lambda_2 \)} (general_case_mid)
            (general_case_mid) edge node {\( \lambda_2 \)} (general_case_down)
            ;
    \end{tikzpicture}
    \caption{Markov chain} 
    \label{markov_model}
\end{figure}


\subsection{Markov-chain state mapping function}
The transition matrix of the Markov-chain representation described above can be 
denoted by a state mapping function. 
The state space of this function is defined as:



\begin{align}
    S(T) =& S_1(T) \cup S_2(T) \text{ where:} \nonumber \\
    S_1(T) =& \left\{(0, v)\in\mathbb{N}_0^2 \; | \; v < T \right\} \label{eq:state_space} \\
    S_2(T) =& \{(u, v)\in\mathbb{N}_0^2 \; | \; v \geq T \} \nonumber
\end{align}

Therefore, the entries of the transition matrix \(Q\), can be given by 
\( q_{i,j} = q_{(u_i, v_i),(u_j, v_j)} \) which is the transition rate from state 
\( i = (u_i, v_i) \) to state \( j = (u_j , v_j) \) for all 
\( (u_i, v_i), (u_j, v_j) \in S \).

\begin{equation} \label{eq:markov_transition_rate}
    q_{i, j} = 
    \begin{cases}
        \Lambda, & \textbf{if } (u_i, v_i) - (u_j, v_j) = (0,-1) \textbf{ and } 
        v_i < \text{t} \\
        \lambda_1, & \textbf{if } (u_i, v_i) - (u_j, v_j) = (0,-1) \textbf{ and } 
        v_i \geq \text{t} \\
        \lambda_2, & \textbf{if } (u_i, v_i) - (u_j, v_j) = (-1,0) \\
        v_i \mu, & \textbf{if } (u_i, v_i) - (u_j, v_j) = (0,1) \textbf{ and } 
        v_i \leq C \textbf{ or} \\ & \hspace{0.37cm}(u_i, v_i) - (u_j, v_j) = (1,0) 
        \textbf{ and } v_i = T \leq C \\
        C \mu, & \textbf{if } (u_i, v_i) - (u_j, v_j) = (0,1) \textbf{ and } v_i > C 
        \textbf{ or} \\ & \hspace{0.37cm}(u_i, v_i) - (u_j, v_j) = (1,0) \textbf{ and } 
        v_i = T > C\\
        -\sum_{j=1}^{|Q|}{q_{i,j}} & \textbf{if } i = j \\
        0, & \textbf{otherwise}
    \end{cases}
\end{equation}

In order to acquire an exact solution of the problem a slight adjustment needs to 
be considered. 
The problem defined above assumes no upper boundary to the number of individuals 
that can wait for service or the ones that are blocked in the buffer centre. 
Therefore, a different state space \( \tilde S \) needs to be constructed where 
\( \tilde S \subseteq S \) and there is a maximum allowed number of people \( N \) 
that can be in the system and a maximum allowed number of people \( M \) that can
be blocked in the buffer centre:

\begin{equation}
    \tilde S = \left\{ (u, v) \in S\;| u \leq M, v\leq N \right\}
\end{equation}


\subsection{Steady State}
Having calculated the transition matrix \( Q \) for a given set of parameters the 
probability vector \( \pi \) needs to be considered. 
The vector \( \pi \) is commonly used to study such stochastic systems and it's 
main purpose is to keep track of the probability of being at any given state of 
the system. 
The term \textit{steady state} refers to the instance of the vector \( \pi \) where 
the probabilities of being at any state become stable over time. 
Thus, by considering the steady state vector \( \pi \) the relationship between 
it and \(Q \) is given by:

\[
\frac{d\pi}{dt} = \pi Q = 0
\]

There are numerous methods that can be used to solve problems of such kind. 
In this paper only numeric and algebraic approaches will be considered. 

\subsubsection{Numeric integration}
The first approach to be considered is to solve the differential equation numerically 
by observing the behaviour of the model over time. 
The solution is obtained via python's SciPy library. 
The functions odeint and solve\textunderscore ivp have been used in order to find 
a solution to the problem. 
Both of these functions can be used to solve any system of first order ODEs.

\subsubsection{Linear algebraic approach}
Another approach to be considered is the linear algebraic method. 
The steady state vector can be found algebraically by satisfying the following set 
of equations:
\[ \pi Q = 0 \]
\[ \sum_{i} \pi_i = 1 \]

These equations can be solved by slightly altering \( Q \) such that the final column 
is replaced by a vector of ones. 
Thus, the resultant solution occurs from solving the equation \( \tilde{Q}^T \pi = b \) 
where \( \tilde{Q} \) and \( b \) are defined as:

\[
\tilde{q_{i, j}} = 
\begin{cases}
    1, & \textbf{if } j = |Q| \\
    q_{i,j}, & \textbf{otherwise}
\end{cases}
\]

\[
b = 
\begin{bmatrix}
    x_{1} \\
    x_{2} \\
    \vdots \\
    x_{m}
\end{bmatrix}
\]


\subsubsection{Least Squares approach}
Finally, the last approach to be considered is the least squares method. 
This approach is considered because while the problem becomes more complex (in terms 
of input parameters) the computational time required to solve it increases exponentially. 
Thus, one may obtain the steady state vector \( \pi \) by solving the following 
equation.

\[
\pi = \text{argmin}_{x\in\mathbb{R}^{d}}\|Mx-b\|_2^2
\]
