\section{Markov chain VS Simulation}

\subsection{Example model}
Consider the Markov chain paradigm in figure \ref{Model_mini}. 
The illustrated model represents the unrealistically small system 
with a system capacity of five and a buffer capacity of three. 
The hospital in this particular example also has four servers and a threshold of 
three; meaning that every ambulance that arrives in a time that there are three or 
more individuals in the hospital, will proceed to the buffer centre.

\begin{figure}[h]
    \centering
    \section{Game Theory component:} 

\textbf{\underline{Players:}} 
\begin{itemize}
    \item Ambulance
    \item Hospital A
    \item Hospital B
\end{itemize}

\noindent 
\textbf{\underline{Strategies of players:}}
\begin{itemize}
    \item Hospital i:    
    \begin{enumerate} 
        \item Close doors at \(\hat{c_i} = 1\) 
        \item Close doors at \(\hat{c_i} = 2\)
        \item \dots
        \item Close doors at \(\hat{c_i} = C_i\)
    \end{enumerate}
    \item Ambulance:
    \begin{enumerate}
        \item Choose \(p_1 \in [0,1]\) 
    \end{enumerate}
\end{itemize}

\noindent 
\textbf{\underline{Cost Functions:}} Waiting times + the distance to each hospital. 





    \caption{Markov chains: number of servers=4} 
    \label{Model_mini}
\end{figure}

In addition to the Markov chain model a simulation model has also been built based 
on the same parameters. 
Comparing the results of the Markov model and the equivalent simulation model the 
resultant plots arose.

The heatmaps in figure \ref{Heatmap_mini} represent the state probabilities for 
the Markov chain model, the simulation model and the difference between the two. 
Each pixel of the heatmap corresponds to the equivalent state of figure \ref{Model_mini} 
and represents the probability of being at that state in any particular moment of time.

It can be observed that both Markov chain and simulation models' state probabilities 
vary from 5\% to 25\% and that states \( (0, 1) \) and \( (0, 2) \) are the most 
visited ones. 
Looking at the differences' heatmap, one may identify that the differences between 
the two are minimal.

\newpage

\begin{figure}[h]
    \includegraphics[width=\linewidth]{Comparisons/Example_model/Heatmap/main.pdf}
    \caption{Heatmaps of Simulation, Markov chains and differences of the two}
    \label{Heatmap_mini}
\end{figure}



