\section{Markov chain VS Simulation}

\subsection{Example model}
Consider the Markov chain paradigm in figure \ref{Model_mini}. 
The illustrated model represents the unrealistically small system 
with a system capacity of five and a buffer capacity of three. 
The hospital in this particular example also has four servers and a threshold of 
three; meaning that every ambulance that arrives in a time that there are three or 
more individuals in the hospital, will proceed to the buffer centre.

\begin{figure}[h]
    \centering
    \documentclass{article}

\usepackage{amsmath} % For writing mathematics (align, split environments etc.)
\usepackage{mathtools}
% \usepackage{amsthm} % For the proof environment
\usepackage{amsfonts} 
\usepackage{geometry}
\usepackage{float}
\usepackage{graphicx}
\usepackage{soul}
\usepackage{indentfirst}
\usepackage{multicol}
\usepackage{tikz}
\usepackage{cancel}

\usetikzlibrary{calc, automata, chains, arrows.meta, math}
\setcounter{MaxMatrixCols}{20}

\usepackage{biblatex}
\addbibresource{bibliography.bib}


\title{A game theoretic model of the behavioural gaming that takes place at the EMS - ED interface}
\author{}
\date{}

\begin{document}
\maketitle

\documentclass{article}

\usepackage{amsmath} % For writing mathematics (align, split environments etc.)
\usepackage{mathtools}
% \usepackage{amsthm} % For the proof environment
\usepackage{amsfonts} 
\usepackage{geometry}
\usepackage{float}
\usepackage{graphicx}
\usepackage{soul}
\usepackage{indentfirst}
\usepackage{multicol}
\usepackage{tikz}
\usepackage{cancel}

\usetikzlibrary{calc, automata, chains, arrows.meta, math}
\setcounter{MaxMatrixCols}{20}

\usepackage{biblatex}
\addbibresource{bibliography.bib}


\title{A game theoretic model of the behavioural gaming that takes place at the EMS - ED interface}
\author{}
\date{}

\begin{document}
\maketitle

\documentclass{article}

\usepackage{amsmath} % For writing mathematics (align, split environments etc.)
\usepackage{mathtools}
% \usepackage{amsthm} % For the proof environment
\usepackage{amsfonts} 
\usepackage{geometry}
\usepackage{float}
\usepackage{graphicx}
\usepackage{soul}
\usepackage{indentfirst}
\usepackage{multicol}
\usepackage{tikz}
\usepackage{cancel}

\usetikzlibrary{calc, automata, chains, arrows.meta, math}
\setcounter{MaxMatrixCols}{20}

\usepackage{biblatex}
\addbibresource{bibliography.bib}


\title{A game theoretic model of the behavioural gaming that takes place at the EMS - ED interface}
\author{}
\date{}

\begin{document}
\maketitle

\input{Abstract/main.tex}
\newpage
\tableofcontents
\newpage

% Introduction of the project
\input{Introduction/main.tex}

% Game Theoretic Component
\input{Game_theory_component/main.tex}


\newpage
% Quick representation of the steps of methodology
\input{Methodology/Quick/main.tex}

\newpage
% Proper methodology
\input{Methodology/Proper/main.tex}

% Markov Chains
\input{MarkovChain/markov_chain_model/main.tex}
\newpage
\input{MarkovChain/closed_form_state_probs/main.tex}
\newpage
\input{MarkovChain/expressions_from_pi/main.tex}

\newpage
% Heatmap comparisons
\input{Comparisons/Example_model/main.tex}


\newpage
\input{Miscellaneous/Useful_tikz/main.tex}


% Formulas used
\newpage
\input{Miscellaneous/Formulas/main.tex}

\newpage
\printbibliography[title={References}]

\end{document}

\newpage
\tableofcontents
\newpage

% Introduction of the project
\documentclass{article}

\usepackage{amsmath} % For writing mathematics (align, split environments etc.)
\usepackage{mathtools}
% \usepackage{amsthm} % For the proof environment
\usepackage{amsfonts} 
\usepackage{geometry}
\usepackage{float}
\usepackage{graphicx}
\usepackage{soul}
\usepackage{indentfirst}
\usepackage{multicol}
\usepackage{tikz}
\usepackage{cancel}

\usetikzlibrary{calc, automata, chains, arrows.meta, math}
\setcounter{MaxMatrixCols}{20}

\usepackage{biblatex}
\addbibresource{bibliography.bib}


\title{A game theoretic model of the behavioural gaming that takes place at the EMS - ED interface}
\author{}
\date{}

\begin{document}
\maketitle

\input{Abstract/main.tex}
\newpage
\tableofcontents
\newpage

% Introduction of the project
\input{Introduction/main.tex}

% Game Theoretic Component
\input{Game_theory_component/main.tex}


\newpage
% Quick representation of the steps of methodology
\input{Methodology/Quick/main.tex}

\newpage
% Proper methodology
\input{Methodology/Proper/main.tex}

% Markov Chains
\input{MarkovChain/markov_chain_model/main.tex}
\newpage
\input{MarkovChain/closed_form_state_probs/main.tex}
\newpage
\input{MarkovChain/expressions_from_pi/main.tex}

\newpage
% Heatmap comparisons
\input{Comparisons/Example_model/main.tex}


\newpage
\input{Miscellaneous/Useful_tikz/main.tex}


% Formulas used
\newpage
\input{Miscellaneous/Formulas/main.tex}

\newpage
\printbibliography[title={References}]

\end{document}


% Game Theoretic Component
\documentclass{article}

\usepackage{amsmath} % For writing mathematics (align, split environments etc.)
\usepackage{mathtools}
% \usepackage{amsthm} % For the proof environment
\usepackage{amsfonts} 
\usepackage{geometry}
\usepackage{float}
\usepackage{graphicx}
\usepackage{soul}
\usepackage{indentfirst}
\usepackage{multicol}
\usepackage{tikz}
\usepackage{cancel}

\usetikzlibrary{calc, automata, chains, arrows.meta, math}
\setcounter{MaxMatrixCols}{20}

\usepackage{biblatex}
\addbibresource{bibliography.bib}


\title{A game theoretic model of the behavioural gaming that takes place at the EMS - ED interface}
\author{}
\date{}

\begin{document}
\maketitle

\input{Abstract/main.tex}
\newpage
\tableofcontents
\newpage

% Introduction of the project
\input{Introduction/main.tex}

% Game Theoretic Component
\input{Game_theory_component/main.tex}


\newpage
% Quick representation of the steps of methodology
\input{Methodology/Quick/main.tex}

\newpage
% Proper methodology
\input{Methodology/Proper/main.tex}

% Markov Chains
\input{MarkovChain/markov_chain_model/main.tex}
\newpage
\input{MarkovChain/closed_form_state_probs/main.tex}
\newpage
\input{MarkovChain/expressions_from_pi/main.tex}

\newpage
% Heatmap comparisons
\input{Comparisons/Example_model/main.tex}


\newpage
\input{Miscellaneous/Useful_tikz/main.tex}


% Formulas used
\newpage
\input{Miscellaneous/Formulas/main.tex}

\newpage
\printbibliography[title={References}]

\end{document}



\newpage
% Quick representation of the steps of methodology
\documentclass{article}

\usepackage{amsmath} % For writing mathematics (align, split environments etc.)
\usepackage{mathtools}
% \usepackage{amsthm} % For the proof environment
\usepackage{amsfonts} 
\usepackage{geometry}
\usepackage{float}
\usepackage{graphicx}
\usepackage{soul}
\usepackage{indentfirst}
\usepackage{multicol}
\usepackage{tikz}
\usepackage{cancel}

\usetikzlibrary{calc, automata, chains, arrows.meta, math}
\setcounter{MaxMatrixCols}{20}

\usepackage{biblatex}
\addbibresource{bibliography.bib}


\title{A game theoretic model of the behavioural gaming that takes place at the EMS - ED interface}
\author{}
\date{}

\begin{document}
\maketitle

\input{Abstract/main.tex}
\newpage
\tableofcontents
\newpage

% Introduction of the project
\input{Introduction/main.tex}

% Game Theoretic Component
\input{Game_theory_component/main.tex}


\newpage
% Quick representation of the steps of methodology
\input{Methodology/Quick/main.tex}

\newpage
% Proper methodology
\input{Methodology/Proper/main.tex}

% Markov Chains
\input{MarkovChain/markov_chain_model/main.tex}
\newpage
\input{MarkovChain/closed_form_state_probs/main.tex}
\newpage
\input{MarkovChain/expressions_from_pi/main.tex}

\newpage
% Heatmap comparisons
\input{Comparisons/Example_model/main.tex}


\newpage
\input{Miscellaneous/Useful_tikz/main.tex}


% Formulas used
\newpage
\input{Miscellaneous/Formulas/main.tex}

\newpage
\printbibliography[title={References}]

\end{document}


\newpage
% Proper methodology
\documentclass{article}

\usepackage{amsmath} % For writing mathematics (align, split environments etc.)
\usepackage{mathtools}
% \usepackage{amsthm} % For the proof environment
\usepackage{amsfonts} 
\usepackage{geometry}
\usepackage{float}
\usepackage{graphicx}
\usepackage{soul}
\usepackage{indentfirst}
\usepackage{multicol}
\usepackage{tikz}
\usepackage{cancel}

\usetikzlibrary{calc, automata, chains, arrows.meta, math}
\setcounter{MaxMatrixCols}{20}

\usepackage{biblatex}
\addbibresource{bibliography.bib}


\title{A game theoretic model of the behavioural gaming that takes place at the EMS - ED interface}
\author{}
\date{}

\begin{document}
\maketitle

\input{Abstract/main.tex}
\newpage
\tableofcontents
\newpage

% Introduction of the project
\input{Introduction/main.tex}

% Game Theoretic Component
\input{Game_theory_component/main.tex}


\newpage
% Quick representation of the steps of methodology
\input{Methodology/Quick/main.tex}

\newpage
% Proper methodology
\input{Methodology/Proper/main.tex}

% Markov Chains
\input{MarkovChain/markov_chain_model/main.tex}
\newpage
\input{MarkovChain/closed_form_state_probs/main.tex}
\newpage
\input{MarkovChain/expressions_from_pi/main.tex}

\newpage
% Heatmap comparisons
\input{Comparisons/Example_model/main.tex}


\newpage
\input{Miscellaneous/Useful_tikz/main.tex}


% Formulas used
\newpage
\input{Miscellaneous/Formulas/main.tex}

\newpage
\printbibliography[title={References}]

\end{document}


% Markov Chains
\documentclass{article}

\usepackage{amsmath} % For writing mathematics (align, split environments etc.)
\usepackage{mathtools}
% \usepackage{amsthm} % For the proof environment
\usepackage{amsfonts} 
\usepackage{geometry}
\usepackage{float}
\usepackage{graphicx}
\usepackage{soul}
\usepackage{indentfirst}
\usepackage{multicol}
\usepackage{tikz}
\usepackage{cancel}

\usetikzlibrary{calc, automata, chains, arrows.meta, math}
\setcounter{MaxMatrixCols}{20}

\usepackage{biblatex}
\addbibresource{bibliography.bib}


\title{A game theoretic model of the behavioural gaming that takes place at the EMS - ED interface}
\author{}
\date{}

\begin{document}
\maketitle

\input{Abstract/main.tex}
\newpage
\tableofcontents
\newpage

% Introduction of the project
\input{Introduction/main.tex}

% Game Theoretic Component
\input{Game_theory_component/main.tex}


\newpage
% Quick representation of the steps of methodology
\input{Methodology/Quick/main.tex}

\newpage
% Proper methodology
\input{Methodology/Proper/main.tex}

% Markov Chains
\input{MarkovChain/markov_chain_model/main.tex}
\newpage
\input{MarkovChain/closed_form_state_probs/main.tex}
\newpage
\input{MarkovChain/expressions_from_pi/main.tex}

\newpage
% Heatmap comparisons
\input{Comparisons/Example_model/main.tex}


\newpage
\input{Miscellaneous/Useful_tikz/main.tex}


% Formulas used
\newpage
\input{Miscellaneous/Formulas/main.tex}

\newpage
\printbibliography[title={References}]

\end{document}

\newpage
\documentclass{article}

\usepackage{amsmath} % For writing mathematics (align, split environments etc.)
\usepackage{mathtools}
% \usepackage{amsthm} % For the proof environment
\usepackage{amsfonts} 
\usepackage{geometry}
\usepackage{float}
\usepackage{graphicx}
\usepackage{soul}
\usepackage{indentfirst}
\usepackage{multicol}
\usepackage{tikz}
\usepackage{cancel}

\usetikzlibrary{calc, automata, chains, arrows.meta, math}
\setcounter{MaxMatrixCols}{20}

\usepackage{biblatex}
\addbibresource{bibliography.bib}


\title{A game theoretic model of the behavioural gaming that takes place at the EMS - ED interface}
\author{}
\date{}

\begin{document}
\maketitle

\input{Abstract/main.tex}
\newpage
\tableofcontents
\newpage

% Introduction of the project
\input{Introduction/main.tex}

% Game Theoretic Component
\input{Game_theory_component/main.tex}


\newpage
% Quick representation of the steps of methodology
\input{Methodology/Quick/main.tex}

\newpage
% Proper methodology
\input{Methodology/Proper/main.tex}

% Markov Chains
\input{MarkovChain/markov_chain_model/main.tex}
\newpage
\input{MarkovChain/closed_form_state_probs/main.tex}
\newpage
\input{MarkovChain/expressions_from_pi/main.tex}

\newpage
% Heatmap comparisons
\input{Comparisons/Example_model/main.tex}


\newpage
\input{Miscellaneous/Useful_tikz/main.tex}


% Formulas used
\newpage
\input{Miscellaneous/Formulas/main.tex}

\newpage
\printbibliography[title={References}]

\end{document}

\newpage
\documentclass{article}

\usepackage{amsmath} % For writing mathematics (align, split environments etc.)
\usepackage{mathtools}
% \usepackage{amsthm} % For the proof environment
\usepackage{amsfonts} 
\usepackage{geometry}
\usepackage{float}
\usepackage{graphicx}
\usepackage{soul}
\usepackage{indentfirst}
\usepackage{multicol}
\usepackage{tikz}
\usepackage{cancel}

\usetikzlibrary{calc, automata, chains, arrows.meta, math}
\setcounter{MaxMatrixCols}{20}

\usepackage{biblatex}
\addbibresource{bibliography.bib}


\title{A game theoretic model of the behavioural gaming that takes place at the EMS - ED interface}
\author{}
\date{}

\begin{document}
\maketitle

\input{Abstract/main.tex}
\newpage
\tableofcontents
\newpage

% Introduction of the project
\input{Introduction/main.tex}

% Game Theoretic Component
\input{Game_theory_component/main.tex}


\newpage
% Quick representation of the steps of methodology
\input{Methodology/Quick/main.tex}

\newpage
% Proper methodology
\input{Methodology/Proper/main.tex}

% Markov Chains
\input{MarkovChain/markov_chain_model/main.tex}
\newpage
\input{MarkovChain/closed_form_state_probs/main.tex}
\newpage
\input{MarkovChain/expressions_from_pi/main.tex}

\newpage
% Heatmap comparisons
\input{Comparisons/Example_model/main.tex}


\newpage
\input{Miscellaneous/Useful_tikz/main.tex}


% Formulas used
\newpage
\input{Miscellaneous/Formulas/main.tex}

\newpage
\printbibliography[title={References}]

\end{document}


\newpage
% Heatmap comparisons
\documentclass{article}

\usepackage{amsmath} % For writing mathematics (align, split environments etc.)
\usepackage{mathtools}
% \usepackage{amsthm} % For the proof environment
\usepackage{amsfonts} 
\usepackage{geometry}
\usepackage{float}
\usepackage{graphicx}
\usepackage{soul}
\usepackage{indentfirst}
\usepackage{multicol}
\usepackage{tikz}
\usepackage{cancel}

\usetikzlibrary{calc, automata, chains, arrows.meta, math}
\setcounter{MaxMatrixCols}{20}

\usepackage{biblatex}
\addbibresource{bibliography.bib}


\title{A game theoretic model of the behavioural gaming that takes place at the EMS - ED interface}
\author{}
\date{}

\begin{document}
\maketitle

\input{Abstract/main.tex}
\newpage
\tableofcontents
\newpage

% Introduction of the project
\input{Introduction/main.tex}

% Game Theoretic Component
\input{Game_theory_component/main.tex}


\newpage
% Quick representation of the steps of methodology
\input{Methodology/Quick/main.tex}

\newpage
% Proper methodology
\input{Methodology/Proper/main.tex}

% Markov Chains
\input{MarkovChain/markov_chain_model/main.tex}
\newpage
\input{MarkovChain/closed_form_state_probs/main.tex}
\newpage
\input{MarkovChain/expressions_from_pi/main.tex}

\newpage
% Heatmap comparisons
\input{Comparisons/Example_model/main.tex}


\newpage
\input{Miscellaneous/Useful_tikz/main.tex}


% Formulas used
\newpage
\input{Miscellaneous/Formulas/main.tex}

\newpage
\printbibliography[title={References}]

\end{document}



\newpage
\documentclass{article}

\usepackage{amsmath} % For writing mathematics (align, split environments etc.)
\usepackage{mathtools}
% \usepackage{amsthm} % For the proof environment
\usepackage{amsfonts} 
\usepackage{geometry}
\usepackage{float}
\usepackage{graphicx}
\usepackage{soul}
\usepackage{indentfirst}
\usepackage{multicol}
\usepackage{tikz}
\usepackage{cancel}

\usetikzlibrary{calc, automata, chains, arrows.meta, math}
\setcounter{MaxMatrixCols}{20}

\usepackage{biblatex}
\addbibresource{bibliography.bib}


\title{A game theoretic model of the behavioural gaming that takes place at the EMS - ED interface}
\author{}
\date{}

\begin{document}
\maketitle

\input{Abstract/main.tex}
\newpage
\tableofcontents
\newpage

% Introduction of the project
\input{Introduction/main.tex}

% Game Theoretic Component
\input{Game_theory_component/main.tex}


\newpage
% Quick representation of the steps of methodology
\input{Methodology/Quick/main.tex}

\newpage
% Proper methodology
\input{Methodology/Proper/main.tex}

% Markov Chains
\input{MarkovChain/markov_chain_model/main.tex}
\newpage
\input{MarkovChain/closed_form_state_probs/main.tex}
\newpage
\input{MarkovChain/expressions_from_pi/main.tex}

\newpage
% Heatmap comparisons
\input{Comparisons/Example_model/main.tex}


\newpage
\input{Miscellaneous/Useful_tikz/main.tex}


% Formulas used
\newpage
\input{Miscellaneous/Formulas/main.tex}

\newpage
\printbibliography[title={References}]

\end{document}



% Formulas used
\newpage
\documentclass{article}

\usepackage{amsmath} % For writing mathematics (align, split environments etc.)
\usepackage{mathtools}
% \usepackage{amsthm} % For the proof environment
\usepackage{amsfonts} 
\usepackage{geometry}
\usepackage{float}
\usepackage{graphicx}
\usepackage{soul}
\usepackage{indentfirst}
\usepackage{multicol}
\usepackage{tikz}
\usepackage{cancel}

\usetikzlibrary{calc, automata, chains, arrows.meta, math}
\setcounter{MaxMatrixCols}{20}

\usepackage{biblatex}
\addbibresource{bibliography.bib}


\title{A game theoretic model of the behavioural gaming that takes place at the EMS - ED interface}
\author{}
\date{}

\begin{document}
\maketitle

\input{Abstract/main.tex}
\newpage
\tableofcontents
\newpage

% Introduction of the project
\input{Introduction/main.tex}

% Game Theoretic Component
\input{Game_theory_component/main.tex}


\newpage
% Quick representation of the steps of methodology
\input{Methodology/Quick/main.tex}

\newpage
% Proper methodology
\input{Methodology/Proper/main.tex}

% Markov Chains
\input{MarkovChain/markov_chain_model/main.tex}
\newpage
\input{MarkovChain/closed_form_state_probs/main.tex}
\newpage
\input{MarkovChain/expressions_from_pi/main.tex}

\newpage
% Heatmap comparisons
\input{Comparisons/Example_model/main.tex}


\newpage
\input{Miscellaneous/Useful_tikz/main.tex}


% Formulas used
\newpage
\input{Miscellaneous/Formulas/main.tex}

\newpage
\printbibliography[title={References}]

\end{document}


\newpage
\printbibliography[title={References}]

\end{document}

\newpage
\tableofcontents
\newpage

% Introduction of the project
\documentclass{article}

\usepackage{amsmath} % For writing mathematics (align, split environments etc.)
\usepackage{mathtools}
% \usepackage{amsthm} % For the proof environment
\usepackage{amsfonts} 
\usepackage{geometry}
\usepackage{float}
\usepackage{graphicx}
\usepackage{soul}
\usepackage{indentfirst}
\usepackage{multicol}
\usepackage{tikz}
\usepackage{cancel}

\usetikzlibrary{calc, automata, chains, arrows.meta, math}
\setcounter{MaxMatrixCols}{20}

\usepackage{biblatex}
\addbibresource{bibliography.bib}


\title{A game theoretic model of the behavioural gaming that takes place at the EMS - ED interface}
\author{}
\date{}

\begin{document}
\maketitle

\documentclass{article}

\usepackage{amsmath} % For writing mathematics (align, split environments etc.)
\usepackage{mathtools}
% \usepackage{amsthm} % For the proof environment
\usepackage{amsfonts} 
\usepackage{geometry}
\usepackage{float}
\usepackage{graphicx}
\usepackage{soul}
\usepackage{indentfirst}
\usepackage{multicol}
\usepackage{tikz}
\usepackage{cancel}

\usetikzlibrary{calc, automata, chains, arrows.meta, math}
\setcounter{MaxMatrixCols}{20}

\usepackage{biblatex}
\addbibresource{bibliography.bib}


\title{A game theoretic model of the behavioural gaming that takes place at the EMS - ED interface}
\author{}
\date{}

\begin{document}
\maketitle

\input{Abstract/main.tex}
\newpage
\tableofcontents
\newpage

% Introduction of the project
\input{Introduction/main.tex}

% Game Theoretic Component
\input{Game_theory_component/main.tex}


\newpage
% Quick representation of the steps of methodology
\input{Methodology/Quick/main.tex}

\newpage
% Proper methodology
\input{Methodology/Proper/main.tex}

% Markov Chains
\input{MarkovChain/markov_chain_model/main.tex}
\newpage
\input{MarkovChain/closed_form_state_probs/main.tex}
\newpage
\input{MarkovChain/expressions_from_pi/main.tex}

\newpage
% Heatmap comparisons
\input{Comparisons/Example_model/main.tex}


\newpage
\input{Miscellaneous/Useful_tikz/main.tex}


% Formulas used
\newpage
\input{Miscellaneous/Formulas/main.tex}

\newpage
\printbibliography[title={References}]

\end{document}

\newpage
\tableofcontents
\newpage

% Introduction of the project
\documentclass{article}

\usepackage{amsmath} % For writing mathematics (align, split environments etc.)
\usepackage{mathtools}
% \usepackage{amsthm} % For the proof environment
\usepackage{amsfonts} 
\usepackage{geometry}
\usepackage{float}
\usepackage{graphicx}
\usepackage{soul}
\usepackage{indentfirst}
\usepackage{multicol}
\usepackage{tikz}
\usepackage{cancel}

\usetikzlibrary{calc, automata, chains, arrows.meta, math}
\setcounter{MaxMatrixCols}{20}

\usepackage{biblatex}
\addbibresource{bibliography.bib}


\title{A game theoretic model of the behavioural gaming that takes place at the EMS - ED interface}
\author{}
\date{}

\begin{document}
\maketitle

\input{Abstract/main.tex}
\newpage
\tableofcontents
\newpage

% Introduction of the project
\input{Introduction/main.tex}

% Game Theoretic Component
\input{Game_theory_component/main.tex}


\newpage
% Quick representation of the steps of methodology
\input{Methodology/Quick/main.tex}

\newpage
% Proper methodology
\input{Methodology/Proper/main.tex}

% Markov Chains
\input{MarkovChain/markov_chain_model/main.tex}
\newpage
\input{MarkovChain/closed_form_state_probs/main.tex}
\newpage
\input{MarkovChain/expressions_from_pi/main.tex}

\newpage
% Heatmap comparisons
\input{Comparisons/Example_model/main.tex}


\newpage
\input{Miscellaneous/Useful_tikz/main.tex}


% Formulas used
\newpage
\input{Miscellaneous/Formulas/main.tex}

\newpage
\printbibliography[title={References}]

\end{document}


% Game Theoretic Component
\documentclass{article}

\usepackage{amsmath} % For writing mathematics (align, split environments etc.)
\usepackage{mathtools}
% \usepackage{amsthm} % For the proof environment
\usepackage{amsfonts} 
\usepackage{geometry}
\usepackage{float}
\usepackage{graphicx}
\usepackage{soul}
\usepackage{indentfirst}
\usepackage{multicol}
\usepackage{tikz}
\usepackage{cancel}

\usetikzlibrary{calc, automata, chains, arrows.meta, math}
\setcounter{MaxMatrixCols}{20}

\usepackage{biblatex}
\addbibresource{bibliography.bib}


\title{A game theoretic model of the behavioural gaming that takes place at the EMS - ED interface}
\author{}
\date{}

\begin{document}
\maketitle

\input{Abstract/main.tex}
\newpage
\tableofcontents
\newpage

% Introduction of the project
\input{Introduction/main.tex}

% Game Theoretic Component
\input{Game_theory_component/main.tex}


\newpage
% Quick representation of the steps of methodology
\input{Methodology/Quick/main.tex}

\newpage
% Proper methodology
\input{Methodology/Proper/main.tex}

% Markov Chains
\input{MarkovChain/markov_chain_model/main.tex}
\newpage
\input{MarkovChain/closed_form_state_probs/main.tex}
\newpage
\input{MarkovChain/expressions_from_pi/main.tex}

\newpage
% Heatmap comparisons
\input{Comparisons/Example_model/main.tex}


\newpage
\input{Miscellaneous/Useful_tikz/main.tex}


% Formulas used
\newpage
\input{Miscellaneous/Formulas/main.tex}

\newpage
\printbibliography[title={References}]

\end{document}



\newpage
% Quick representation of the steps of methodology
\documentclass{article}

\usepackage{amsmath} % For writing mathematics (align, split environments etc.)
\usepackage{mathtools}
% \usepackage{amsthm} % For the proof environment
\usepackage{amsfonts} 
\usepackage{geometry}
\usepackage{float}
\usepackage{graphicx}
\usepackage{soul}
\usepackage{indentfirst}
\usepackage{multicol}
\usepackage{tikz}
\usepackage{cancel}

\usetikzlibrary{calc, automata, chains, arrows.meta, math}
\setcounter{MaxMatrixCols}{20}

\usepackage{biblatex}
\addbibresource{bibliography.bib}


\title{A game theoretic model of the behavioural gaming that takes place at the EMS - ED interface}
\author{}
\date{}

\begin{document}
\maketitle

\input{Abstract/main.tex}
\newpage
\tableofcontents
\newpage

% Introduction of the project
\input{Introduction/main.tex}

% Game Theoretic Component
\input{Game_theory_component/main.tex}


\newpage
% Quick representation of the steps of methodology
\input{Methodology/Quick/main.tex}

\newpage
% Proper methodology
\input{Methodology/Proper/main.tex}

% Markov Chains
\input{MarkovChain/markov_chain_model/main.tex}
\newpage
\input{MarkovChain/closed_form_state_probs/main.tex}
\newpage
\input{MarkovChain/expressions_from_pi/main.tex}

\newpage
% Heatmap comparisons
\input{Comparisons/Example_model/main.tex}


\newpage
\input{Miscellaneous/Useful_tikz/main.tex}


% Formulas used
\newpage
\input{Miscellaneous/Formulas/main.tex}

\newpage
\printbibliography[title={References}]

\end{document}


\newpage
% Proper methodology
\documentclass{article}

\usepackage{amsmath} % For writing mathematics (align, split environments etc.)
\usepackage{mathtools}
% \usepackage{amsthm} % For the proof environment
\usepackage{amsfonts} 
\usepackage{geometry}
\usepackage{float}
\usepackage{graphicx}
\usepackage{soul}
\usepackage{indentfirst}
\usepackage{multicol}
\usepackage{tikz}
\usepackage{cancel}

\usetikzlibrary{calc, automata, chains, arrows.meta, math}
\setcounter{MaxMatrixCols}{20}

\usepackage{biblatex}
\addbibresource{bibliography.bib}


\title{A game theoretic model of the behavioural gaming that takes place at the EMS - ED interface}
\author{}
\date{}

\begin{document}
\maketitle

\input{Abstract/main.tex}
\newpage
\tableofcontents
\newpage

% Introduction of the project
\input{Introduction/main.tex}

% Game Theoretic Component
\input{Game_theory_component/main.tex}


\newpage
% Quick representation of the steps of methodology
\input{Methodology/Quick/main.tex}

\newpage
% Proper methodology
\input{Methodology/Proper/main.tex}

% Markov Chains
\input{MarkovChain/markov_chain_model/main.tex}
\newpage
\input{MarkovChain/closed_form_state_probs/main.tex}
\newpage
\input{MarkovChain/expressions_from_pi/main.tex}

\newpage
% Heatmap comparisons
\input{Comparisons/Example_model/main.tex}


\newpage
\input{Miscellaneous/Useful_tikz/main.tex}


% Formulas used
\newpage
\input{Miscellaneous/Formulas/main.tex}

\newpage
\printbibliography[title={References}]

\end{document}


% Markov Chains
\documentclass{article}

\usepackage{amsmath} % For writing mathematics (align, split environments etc.)
\usepackage{mathtools}
% \usepackage{amsthm} % For the proof environment
\usepackage{amsfonts} 
\usepackage{geometry}
\usepackage{float}
\usepackage{graphicx}
\usepackage{soul}
\usepackage{indentfirst}
\usepackage{multicol}
\usepackage{tikz}
\usepackage{cancel}

\usetikzlibrary{calc, automata, chains, arrows.meta, math}
\setcounter{MaxMatrixCols}{20}

\usepackage{biblatex}
\addbibresource{bibliography.bib}


\title{A game theoretic model of the behavioural gaming that takes place at the EMS - ED interface}
\author{}
\date{}

\begin{document}
\maketitle

\input{Abstract/main.tex}
\newpage
\tableofcontents
\newpage

% Introduction of the project
\input{Introduction/main.tex}

% Game Theoretic Component
\input{Game_theory_component/main.tex}


\newpage
% Quick representation of the steps of methodology
\input{Methodology/Quick/main.tex}

\newpage
% Proper methodology
\input{Methodology/Proper/main.tex}

% Markov Chains
\input{MarkovChain/markov_chain_model/main.tex}
\newpage
\input{MarkovChain/closed_form_state_probs/main.tex}
\newpage
\input{MarkovChain/expressions_from_pi/main.tex}

\newpage
% Heatmap comparisons
\input{Comparisons/Example_model/main.tex}


\newpage
\input{Miscellaneous/Useful_tikz/main.tex}


% Formulas used
\newpage
\input{Miscellaneous/Formulas/main.tex}

\newpage
\printbibliography[title={References}]

\end{document}

\newpage
\documentclass{article}

\usepackage{amsmath} % For writing mathematics (align, split environments etc.)
\usepackage{mathtools}
% \usepackage{amsthm} % For the proof environment
\usepackage{amsfonts} 
\usepackage{geometry}
\usepackage{float}
\usepackage{graphicx}
\usepackage{soul}
\usepackage{indentfirst}
\usepackage{multicol}
\usepackage{tikz}
\usepackage{cancel}

\usetikzlibrary{calc, automata, chains, arrows.meta, math}
\setcounter{MaxMatrixCols}{20}

\usepackage{biblatex}
\addbibresource{bibliography.bib}


\title{A game theoretic model of the behavioural gaming that takes place at the EMS - ED interface}
\author{}
\date{}

\begin{document}
\maketitle

\input{Abstract/main.tex}
\newpage
\tableofcontents
\newpage

% Introduction of the project
\input{Introduction/main.tex}

% Game Theoretic Component
\input{Game_theory_component/main.tex}


\newpage
% Quick representation of the steps of methodology
\input{Methodology/Quick/main.tex}

\newpage
% Proper methodology
\input{Methodology/Proper/main.tex}

% Markov Chains
\input{MarkovChain/markov_chain_model/main.tex}
\newpage
\input{MarkovChain/closed_form_state_probs/main.tex}
\newpage
\input{MarkovChain/expressions_from_pi/main.tex}

\newpage
% Heatmap comparisons
\input{Comparisons/Example_model/main.tex}


\newpage
\input{Miscellaneous/Useful_tikz/main.tex}


% Formulas used
\newpage
\input{Miscellaneous/Formulas/main.tex}

\newpage
\printbibliography[title={References}]

\end{document}

\newpage
\documentclass{article}

\usepackage{amsmath} % For writing mathematics (align, split environments etc.)
\usepackage{mathtools}
% \usepackage{amsthm} % For the proof environment
\usepackage{amsfonts} 
\usepackage{geometry}
\usepackage{float}
\usepackage{graphicx}
\usepackage{soul}
\usepackage{indentfirst}
\usepackage{multicol}
\usepackage{tikz}
\usepackage{cancel}

\usetikzlibrary{calc, automata, chains, arrows.meta, math}
\setcounter{MaxMatrixCols}{20}

\usepackage{biblatex}
\addbibresource{bibliography.bib}


\title{A game theoretic model of the behavioural gaming that takes place at the EMS - ED interface}
\author{}
\date{}

\begin{document}
\maketitle

\input{Abstract/main.tex}
\newpage
\tableofcontents
\newpage

% Introduction of the project
\input{Introduction/main.tex}

% Game Theoretic Component
\input{Game_theory_component/main.tex}


\newpage
% Quick representation of the steps of methodology
\input{Methodology/Quick/main.tex}

\newpage
% Proper methodology
\input{Methodology/Proper/main.tex}

% Markov Chains
\input{MarkovChain/markov_chain_model/main.tex}
\newpage
\input{MarkovChain/closed_form_state_probs/main.tex}
\newpage
\input{MarkovChain/expressions_from_pi/main.tex}

\newpage
% Heatmap comparisons
\input{Comparisons/Example_model/main.tex}


\newpage
\input{Miscellaneous/Useful_tikz/main.tex}


% Formulas used
\newpage
\input{Miscellaneous/Formulas/main.tex}

\newpage
\printbibliography[title={References}]

\end{document}


\newpage
% Heatmap comparisons
\documentclass{article}

\usepackage{amsmath} % For writing mathematics (align, split environments etc.)
\usepackage{mathtools}
% \usepackage{amsthm} % For the proof environment
\usepackage{amsfonts} 
\usepackage{geometry}
\usepackage{float}
\usepackage{graphicx}
\usepackage{soul}
\usepackage{indentfirst}
\usepackage{multicol}
\usepackage{tikz}
\usepackage{cancel}

\usetikzlibrary{calc, automata, chains, arrows.meta, math}
\setcounter{MaxMatrixCols}{20}

\usepackage{biblatex}
\addbibresource{bibliography.bib}


\title{A game theoretic model of the behavioural gaming that takes place at the EMS - ED interface}
\author{}
\date{}

\begin{document}
\maketitle

\input{Abstract/main.tex}
\newpage
\tableofcontents
\newpage

% Introduction of the project
\input{Introduction/main.tex}

% Game Theoretic Component
\input{Game_theory_component/main.tex}


\newpage
% Quick representation of the steps of methodology
\input{Methodology/Quick/main.tex}

\newpage
% Proper methodology
\input{Methodology/Proper/main.tex}

% Markov Chains
\input{MarkovChain/markov_chain_model/main.tex}
\newpage
\input{MarkovChain/closed_form_state_probs/main.tex}
\newpage
\input{MarkovChain/expressions_from_pi/main.tex}

\newpage
% Heatmap comparisons
\input{Comparisons/Example_model/main.tex}


\newpage
\input{Miscellaneous/Useful_tikz/main.tex}


% Formulas used
\newpage
\input{Miscellaneous/Formulas/main.tex}

\newpage
\printbibliography[title={References}]

\end{document}



\newpage
\documentclass{article}

\usepackage{amsmath} % For writing mathematics (align, split environments etc.)
\usepackage{mathtools}
% \usepackage{amsthm} % For the proof environment
\usepackage{amsfonts} 
\usepackage{geometry}
\usepackage{float}
\usepackage{graphicx}
\usepackage{soul}
\usepackage{indentfirst}
\usepackage{multicol}
\usepackage{tikz}
\usepackage{cancel}

\usetikzlibrary{calc, automata, chains, arrows.meta, math}
\setcounter{MaxMatrixCols}{20}

\usepackage{biblatex}
\addbibresource{bibliography.bib}


\title{A game theoretic model of the behavioural gaming that takes place at the EMS - ED interface}
\author{}
\date{}

\begin{document}
\maketitle

\input{Abstract/main.tex}
\newpage
\tableofcontents
\newpage

% Introduction of the project
\input{Introduction/main.tex}

% Game Theoretic Component
\input{Game_theory_component/main.tex}


\newpage
% Quick representation of the steps of methodology
\input{Methodology/Quick/main.tex}

\newpage
% Proper methodology
\input{Methodology/Proper/main.tex}

% Markov Chains
\input{MarkovChain/markov_chain_model/main.tex}
\newpage
\input{MarkovChain/closed_form_state_probs/main.tex}
\newpage
\input{MarkovChain/expressions_from_pi/main.tex}

\newpage
% Heatmap comparisons
\input{Comparisons/Example_model/main.tex}


\newpage
\input{Miscellaneous/Useful_tikz/main.tex}


% Formulas used
\newpage
\input{Miscellaneous/Formulas/main.tex}

\newpage
\printbibliography[title={References}]

\end{document}



% Formulas used
\newpage
\documentclass{article}

\usepackage{amsmath} % For writing mathematics (align, split environments etc.)
\usepackage{mathtools}
% \usepackage{amsthm} % For the proof environment
\usepackage{amsfonts} 
\usepackage{geometry}
\usepackage{float}
\usepackage{graphicx}
\usepackage{soul}
\usepackage{indentfirst}
\usepackage{multicol}
\usepackage{tikz}
\usepackage{cancel}

\usetikzlibrary{calc, automata, chains, arrows.meta, math}
\setcounter{MaxMatrixCols}{20}

\usepackage{biblatex}
\addbibresource{bibliography.bib}


\title{A game theoretic model of the behavioural gaming that takes place at the EMS - ED interface}
\author{}
\date{}

\begin{document}
\maketitle

\input{Abstract/main.tex}
\newpage
\tableofcontents
\newpage

% Introduction of the project
\input{Introduction/main.tex}

% Game Theoretic Component
\input{Game_theory_component/main.tex}


\newpage
% Quick representation of the steps of methodology
\input{Methodology/Quick/main.tex}

\newpage
% Proper methodology
\input{Methodology/Proper/main.tex}

% Markov Chains
\input{MarkovChain/markov_chain_model/main.tex}
\newpage
\input{MarkovChain/closed_form_state_probs/main.tex}
\newpage
\input{MarkovChain/expressions_from_pi/main.tex}

\newpage
% Heatmap comparisons
\input{Comparisons/Example_model/main.tex}


\newpage
\input{Miscellaneous/Useful_tikz/main.tex}


% Formulas used
\newpage
\input{Miscellaneous/Formulas/main.tex}

\newpage
\printbibliography[title={References}]

\end{document}


\newpage
\printbibliography[title={References}]

\end{document}


% Game Theoretic Component
\documentclass{article}

\usepackage{amsmath} % For writing mathematics (align, split environments etc.)
\usepackage{mathtools}
% \usepackage{amsthm} % For the proof environment
\usepackage{amsfonts} 
\usepackage{geometry}
\usepackage{float}
\usepackage{graphicx}
\usepackage{soul}
\usepackage{indentfirst}
\usepackage{multicol}
\usepackage{tikz}
\usepackage{cancel}

\usetikzlibrary{calc, automata, chains, arrows.meta, math}
\setcounter{MaxMatrixCols}{20}

\usepackage{biblatex}
\addbibresource{bibliography.bib}


\title{A game theoretic model of the behavioural gaming that takes place at the EMS - ED interface}
\author{}
\date{}

\begin{document}
\maketitle

\documentclass{article}

\usepackage{amsmath} % For writing mathematics (align, split environments etc.)
\usepackage{mathtools}
% \usepackage{amsthm} % For the proof environment
\usepackage{amsfonts} 
\usepackage{geometry}
\usepackage{float}
\usepackage{graphicx}
\usepackage{soul}
\usepackage{indentfirst}
\usepackage{multicol}
\usepackage{tikz}
\usepackage{cancel}

\usetikzlibrary{calc, automata, chains, arrows.meta, math}
\setcounter{MaxMatrixCols}{20}

\usepackage{biblatex}
\addbibresource{bibliography.bib}


\title{A game theoretic model of the behavioural gaming that takes place at the EMS - ED interface}
\author{}
\date{}

\begin{document}
\maketitle

\input{Abstract/main.tex}
\newpage
\tableofcontents
\newpage

% Introduction of the project
\input{Introduction/main.tex}

% Game Theoretic Component
\input{Game_theory_component/main.tex}


\newpage
% Quick representation of the steps of methodology
\input{Methodology/Quick/main.tex}

\newpage
% Proper methodology
\input{Methodology/Proper/main.tex}

% Markov Chains
\input{MarkovChain/markov_chain_model/main.tex}
\newpage
\input{MarkovChain/closed_form_state_probs/main.tex}
\newpage
\input{MarkovChain/expressions_from_pi/main.tex}

\newpage
% Heatmap comparisons
\input{Comparisons/Example_model/main.tex}


\newpage
\input{Miscellaneous/Useful_tikz/main.tex}


% Formulas used
\newpage
\input{Miscellaneous/Formulas/main.tex}

\newpage
\printbibliography[title={References}]

\end{document}

\newpage
\tableofcontents
\newpage

% Introduction of the project
\documentclass{article}

\usepackage{amsmath} % For writing mathematics (align, split environments etc.)
\usepackage{mathtools}
% \usepackage{amsthm} % For the proof environment
\usepackage{amsfonts} 
\usepackage{geometry}
\usepackage{float}
\usepackage{graphicx}
\usepackage{soul}
\usepackage{indentfirst}
\usepackage{multicol}
\usepackage{tikz}
\usepackage{cancel}

\usetikzlibrary{calc, automata, chains, arrows.meta, math}
\setcounter{MaxMatrixCols}{20}

\usepackage{biblatex}
\addbibresource{bibliography.bib}


\title{A game theoretic model of the behavioural gaming that takes place at the EMS - ED interface}
\author{}
\date{}

\begin{document}
\maketitle

\input{Abstract/main.tex}
\newpage
\tableofcontents
\newpage

% Introduction of the project
\input{Introduction/main.tex}

% Game Theoretic Component
\input{Game_theory_component/main.tex}


\newpage
% Quick representation of the steps of methodology
\input{Methodology/Quick/main.tex}

\newpage
% Proper methodology
\input{Methodology/Proper/main.tex}

% Markov Chains
\input{MarkovChain/markov_chain_model/main.tex}
\newpage
\input{MarkovChain/closed_form_state_probs/main.tex}
\newpage
\input{MarkovChain/expressions_from_pi/main.tex}

\newpage
% Heatmap comparisons
\input{Comparisons/Example_model/main.tex}


\newpage
\input{Miscellaneous/Useful_tikz/main.tex}


% Formulas used
\newpage
\input{Miscellaneous/Formulas/main.tex}

\newpage
\printbibliography[title={References}]

\end{document}


% Game Theoretic Component
\documentclass{article}

\usepackage{amsmath} % For writing mathematics (align, split environments etc.)
\usepackage{mathtools}
% \usepackage{amsthm} % For the proof environment
\usepackage{amsfonts} 
\usepackage{geometry}
\usepackage{float}
\usepackage{graphicx}
\usepackage{soul}
\usepackage{indentfirst}
\usepackage{multicol}
\usepackage{tikz}
\usepackage{cancel}

\usetikzlibrary{calc, automata, chains, arrows.meta, math}
\setcounter{MaxMatrixCols}{20}

\usepackage{biblatex}
\addbibresource{bibliography.bib}


\title{A game theoretic model of the behavioural gaming that takes place at the EMS - ED interface}
\author{}
\date{}

\begin{document}
\maketitle

\input{Abstract/main.tex}
\newpage
\tableofcontents
\newpage

% Introduction of the project
\input{Introduction/main.tex}

% Game Theoretic Component
\input{Game_theory_component/main.tex}


\newpage
% Quick representation of the steps of methodology
\input{Methodology/Quick/main.tex}

\newpage
% Proper methodology
\input{Methodology/Proper/main.tex}

% Markov Chains
\input{MarkovChain/markov_chain_model/main.tex}
\newpage
\input{MarkovChain/closed_form_state_probs/main.tex}
\newpage
\input{MarkovChain/expressions_from_pi/main.tex}

\newpage
% Heatmap comparisons
\input{Comparisons/Example_model/main.tex}


\newpage
\input{Miscellaneous/Useful_tikz/main.tex}


% Formulas used
\newpage
\input{Miscellaneous/Formulas/main.tex}

\newpage
\printbibliography[title={References}]

\end{document}



\newpage
% Quick representation of the steps of methodology
\documentclass{article}

\usepackage{amsmath} % For writing mathematics (align, split environments etc.)
\usepackage{mathtools}
% \usepackage{amsthm} % For the proof environment
\usepackage{amsfonts} 
\usepackage{geometry}
\usepackage{float}
\usepackage{graphicx}
\usepackage{soul}
\usepackage{indentfirst}
\usepackage{multicol}
\usepackage{tikz}
\usepackage{cancel}

\usetikzlibrary{calc, automata, chains, arrows.meta, math}
\setcounter{MaxMatrixCols}{20}

\usepackage{biblatex}
\addbibresource{bibliography.bib}


\title{A game theoretic model of the behavioural gaming that takes place at the EMS - ED interface}
\author{}
\date{}

\begin{document}
\maketitle

\input{Abstract/main.tex}
\newpage
\tableofcontents
\newpage

% Introduction of the project
\input{Introduction/main.tex}

% Game Theoretic Component
\input{Game_theory_component/main.tex}


\newpage
% Quick representation of the steps of methodology
\input{Methodology/Quick/main.tex}

\newpage
% Proper methodology
\input{Methodology/Proper/main.tex}

% Markov Chains
\input{MarkovChain/markov_chain_model/main.tex}
\newpage
\input{MarkovChain/closed_form_state_probs/main.tex}
\newpage
\input{MarkovChain/expressions_from_pi/main.tex}

\newpage
% Heatmap comparisons
\input{Comparisons/Example_model/main.tex}


\newpage
\input{Miscellaneous/Useful_tikz/main.tex}


% Formulas used
\newpage
\input{Miscellaneous/Formulas/main.tex}

\newpage
\printbibliography[title={References}]

\end{document}


\newpage
% Proper methodology
\documentclass{article}

\usepackage{amsmath} % For writing mathematics (align, split environments etc.)
\usepackage{mathtools}
% \usepackage{amsthm} % For the proof environment
\usepackage{amsfonts} 
\usepackage{geometry}
\usepackage{float}
\usepackage{graphicx}
\usepackage{soul}
\usepackage{indentfirst}
\usepackage{multicol}
\usepackage{tikz}
\usepackage{cancel}

\usetikzlibrary{calc, automata, chains, arrows.meta, math}
\setcounter{MaxMatrixCols}{20}

\usepackage{biblatex}
\addbibresource{bibliography.bib}


\title{A game theoretic model of the behavioural gaming that takes place at the EMS - ED interface}
\author{}
\date{}

\begin{document}
\maketitle

\input{Abstract/main.tex}
\newpage
\tableofcontents
\newpage

% Introduction of the project
\input{Introduction/main.tex}

% Game Theoretic Component
\input{Game_theory_component/main.tex}


\newpage
% Quick representation of the steps of methodology
\input{Methodology/Quick/main.tex}

\newpage
% Proper methodology
\input{Methodology/Proper/main.tex}

% Markov Chains
\input{MarkovChain/markov_chain_model/main.tex}
\newpage
\input{MarkovChain/closed_form_state_probs/main.tex}
\newpage
\input{MarkovChain/expressions_from_pi/main.tex}

\newpage
% Heatmap comparisons
\input{Comparisons/Example_model/main.tex}


\newpage
\input{Miscellaneous/Useful_tikz/main.tex}


% Formulas used
\newpage
\input{Miscellaneous/Formulas/main.tex}

\newpage
\printbibliography[title={References}]

\end{document}


% Markov Chains
\documentclass{article}

\usepackage{amsmath} % For writing mathematics (align, split environments etc.)
\usepackage{mathtools}
% \usepackage{amsthm} % For the proof environment
\usepackage{amsfonts} 
\usepackage{geometry}
\usepackage{float}
\usepackage{graphicx}
\usepackage{soul}
\usepackage{indentfirst}
\usepackage{multicol}
\usepackage{tikz}
\usepackage{cancel}

\usetikzlibrary{calc, automata, chains, arrows.meta, math}
\setcounter{MaxMatrixCols}{20}

\usepackage{biblatex}
\addbibresource{bibliography.bib}


\title{A game theoretic model of the behavioural gaming that takes place at the EMS - ED interface}
\author{}
\date{}

\begin{document}
\maketitle

\input{Abstract/main.tex}
\newpage
\tableofcontents
\newpage

% Introduction of the project
\input{Introduction/main.tex}

% Game Theoretic Component
\input{Game_theory_component/main.tex}


\newpage
% Quick representation of the steps of methodology
\input{Methodology/Quick/main.tex}

\newpage
% Proper methodology
\input{Methodology/Proper/main.tex}

% Markov Chains
\input{MarkovChain/markov_chain_model/main.tex}
\newpage
\input{MarkovChain/closed_form_state_probs/main.tex}
\newpage
\input{MarkovChain/expressions_from_pi/main.tex}

\newpage
% Heatmap comparisons
\input{Comparisons/Example_model/main.tex}


\newpage
\input{Miscellaneous/Useful_tikz/main.tex}


% Formulas used
\newpage
\input{Miscellaneous/Formulas/main.tex}

\newpage
\printbibliography[title={References}]

\end{document}

\newpage
\documentclass{article}

\usepackage{amsmath} % For writing mathematics (align, split environments etc.)
\usepackage{mathtools}
% \usepackage{amsthm} % For the proof environment
\usepackage{amsfonts} 
\usepackage{geometry}
\usepackage{float}
\usepackage{graphicx}
\usepackage{soul}
\usepackage{indentfirst}
\usepackage{multicol}
\usepackage{tikz}
\usepackage{cancel}

\usetikzlibrary{calc, automata, chains, arrows.meta, math}
\setcounter{MaxMatrixCols}{20}

\usepackage{biblatex}
\addbibresource{bibliography.bib}


\title{A game theoretic model of the behavioural gaming that takes place at the EMS - ED interface}
\author{}
\date{}

\begin{document}
\maketitle

\input{Abstract/main.tex}
\newpage
\tableofcontents
\newpage

% Introduction of the project
\input{Introduction/main.tex}

% Game Theoretic Component
\input{Game_theory_component/main.tex}


\newpage
% Quick representation of the steps of methodology
\input{Methodology/Quick/main.tex}

\newpage
% Proper methodology
\input{Methodology/Proper/main.tex}

% Markov Chains
\input{MarkovChain/markov_chain_model/main.tex}
\newpage
\input{MarkovChain/closed_form_state_probs/main.tex}
\newpage
\input{MarkovChain/expressions_from_pi/main.tex}

\newpage
% Heatmap comparisons
\input{Comparisons/Example_model/main.tex}


\newpage
\input{Miscellaneous/Useful_tikz/main.tex}


% Formulas used
\newpage
\input{Miscellaneous/Formulas/main.tex}

\newpage
\printbibliography[title={References}]

\end{document}

\newpage
\documentclass{article}

\usepackage{amsmath} % For writing mathematics (align, split environments etc.)
\usepackage{mathtools}
% \usepackage{amsthm} % For the proof environment
\usepackage{amsfonts} 
\usepackage{geometry}
\usepackage{float}
\usepackage{graphicx}
\usepackage{soul}
\usepackage{indentfirst}
\usepackage{multicol}
\usepackage{tikz}
\usepackage{cancel}

\usetikzlibrary{calc, automata, chains, arrows.meta, math}
\setcounter{MaxMatrixCols}{20}

\usepackage{biblatex}
\addbibresource{bibliography.bib}


\title{A game theoretic model of the behavioural gaming that takes place at the EMS - ED interface}
\author{}
\date{}

\begin{document}
\maketitle

\input{Abstract/main.tex}
\newpage
\tableofcontents
\newpage

% Introduction of the project
\input{Introduction/main.tex}

% Game Theoretic Component
\input{Game_theory_component/main.tex}


\newpage
% Quick representation of the steps of methodology
\input{Methodology/Quick/main.tex}

\newpage
% Proper methodology
\input{Methodology/Proper/main.tex}

% Markov Chains
\input{MarkovChain/markov_chain_model/main.tex}
\newpage
\input{MarkovChain/closed_form_state_probs/main.tex}
\newpage
\input{MarkovChain/expressions_from_pi/main.tex}

\newpage
% Heatmap comparisons
\input{Comparisons/Example_model/main.tex}


\newpage
\input{Miscellaneous/Useful_tikz/main.tex}


% Formulas used
\newpage
\input{Miscellaneous/Formulas/main.tex}

\newpage
\printbibliography[title={References}]

\end{document}


\newpage
% Heatmap comparisons
\documentclass{article}

\usepackage{amsmath} % For writing mathematics (align, split environments etc.)
\usepackage{mathtools}
% \usepackage{amsthm} % For the proof environment
\usepackage{amsfonts} 
\usepackage{geometry}
\usepackage{float}
\usepackage{graphicx}
\usepackage{soul}
\usepackage{indentfirst}
\usepackage{multicol}
\usepackage{tikz}
\usepackage{cancel}

\usetikzlibrary{calc, automata, chains, arrows.meta, math}
\setcounter{MaxMatrixCols}{20}

\usepackage{biblatex}
\addbibresource{bibliography.bib}


\title{A game theoretic model of the behavioural gaming that takes place at the EMS - ED interface}
\author{}
\date{}

\begin{document}
\maketitle

\input{Abstract/main.tex}
\newpage
\tableofcontents
\newpage

% Introduction of the project
\input{Introduction/main.tex}

% Game Theoretic Component
\input{Game_theory_component/main.tex}


\newpage
% Quick representation of the steps of methodology
\input{Methodology/Quick/main.tex}

\newpage
% Proper methodology
\input{Methodology/Proper/main.tex}

% Markov Chains
\input{MarkovChain/markov_chain_model/main.tex}
\newpage
\input{MarkovChain/closed_form_state_probs/main.tex}
\newpage
\input{MarkovChain/expressions_from_pi/main.tex}

\newpage
% Heatmap comparisons
\input{Comparisons/Example_model/main.tex}


\newpage
\input{Miscellaneous/Useful_tikz/main.tex}


% Formulas used
\newpage
\input{Miscellaneous/Formulas/main.tex}

\newpage
\printbibliography[title={References}]

\end{document}



\newpage
\documentclass{article}

\usepackage{amsmath} % For writing mathematics (align, split environments etc.)
\usepackage{mathtools}
% \usepackage{amsthm} % For the proof environment
\usepackage{amsfonts} 
\usepackage{geometry}
\usepackage{float}
\usepackage{graphicx}
\usepackage{soul}
\usepackage{indentfirst}
\usepackage{multicol}
\usepackage{tikz}
\usepackage{cancel}

\usetikzlibrary{calc, automata, chains, arrows.meta, math}
\setcounter{MaxMatrixCols}{20}

\usepackage{biblatex}
\addbibresource{bibliography.bib}


\title{A game theoretic model of the behavioural gaming that takes place at the EMS - ED interface}
\author{}
\date{}

\begin{document}
\maketitle

\input{Abstract/main.tex}
\newpage
\tableofcontents
\newpage

% Introduction of the project
\input{Introduction/main.tex}

% Game Theoretic Component
\input{Game_theory_component/main.tex}


\newpage
% Quick representation of the steps of methodology
\input{Methodology/Quick/main.tex}

\newpage
% Proper methodology
\input{Methodology/Proper/main.tex}

% Markov Chains
\input{MarkovChain/markov_chain_model/main.tex}
\newpage
\input{MarkovChain/closed_form_state_probs/main.tex}
\newpage
\input{MarkovChain/expressions_from_pi/main.tex}

\newpage
% Heatmap comparisons
\input{Comparisons/Example_model/main.tex}


\newpage
\input{Miscellaneous/Useful_tikz/main.tex}


% Formulas used
\newpage
\input{Miscellaneous/Formulas/main.tex}

\newpage
\printbibliography[title={References}]

\end{document}



% Formulas used
\newpage
\documentclass{article}

\usepackage{amsmath} % For writing mathematics (align, split environments etc.)
\usepackage{mathtools}
% \usepackage{amsthm} % For the proof environment
\usepackage{amsfonts} 
\usepackage{geometry}
\usepackage{float}
\usepackage{graphicx}
\usepackage{soul}
\usepackage{indentfirst}
\usepackage{multicol}
\usepackage{tikz}
\usepackage{cancel}

\usetikzlibrary{calc, automata, chains, arrows.meta, math}
\setcounter{MaxMatrixCols}{20}

\usepackage{biblatex}
\addbibresource{bibliography.bib}


\title{A game theoretic model of the behavioural gaming that takes place at the EMS - ED interface}
\author{}
\date{}

\begin{document}
\maketitle

\input{Abstract/main.tex}
\newpage
\tableofcontents
\newpage

% Introduction of the project
\input{Introduction/main.tex}

% Game Theoretic Component
\input{Game_theory_component/main.tex}


\newpage
% Quick representation of the steps of methodology
\input{Methodology/Quick/main.tex}

\newpage
% Proper methodology
\input{Methodology/Proper/main.tex}

% Markov Chains
\input{MarkovChain/markov_chain_model/main.tex}
\newpage
\input{MarkovChain/closed_form_state_probs/main.tex}
\newpage
\input{MarkovChain/expressions_from_pi/main.tex}

\newpage
% Heatmap comparisons
\input{Comparisons/Example_model/main.tex}


\newpage
\input{Miscellaneous/Useful_tikz/main.tex}


% Formulas used
\newpage
\input{Miscellaneous/Formulas/main.tex}

\newpage
\printbibliography[title={References}]

\end{document}


\newpage
\printbibliography[title={References}]

\end{document}



\newpage
% Quick representation of the steps of methodology
\documentclass{article}

\usepackage{amsmath} % For writing mathematics (align, split environments etc.)
\usepackage{mathtools}
% \usepackage{amsthm} % For the proof environment
\usepackage{amsfonts} 
\usepackage{geometry}
\usepackage{float}
\usepackage{graphicx}
\usepackage{soul}
\usepackage{indentfirst}
\usepackage{multicol}
\usepackage{tikz}
\usepackage{cancel}

\usetikzlibrary{calc, automata, chains, arrows.meta, math}
\setcounter{MaxMatrixCols}{20}

\usepackage{biblatex}
\addbibresource{bibliography.bib}


\title{A game theoretic model of the behavioural gaming that takes place at the EMS - ED interface}
\author{}
\date{}

\begin{document}
\maketitle

\documentclass{article}

\usepackage{amsmath} % For writing mathematics (align, split environments etc.)
\usepackage{mathtools}
% \usepackage{amsthm} % For the proof environment
\usepackage{amsfonts} 
\usepackage{geometry}
\usepackage{float}
\usepackage{graphicx}
\usepackage{soul}
\usepackage{indentfirst}
\usepackage{multicol}
\usepackage{tikz}
\usepackage{cancel}

\usetikzlibrary{calc, automata, chains, arrows.meta, math}
\setcounter{MaxMatrixCols}{20}

\usepackage{biblatex}
\addbibresource{bibliography.bib}


\title{A game theoretic model of the behavioural gaming that takes place at the EMS - ED interface}
\author{}
\date{}

\begin{document}
\maketitle

\input{Abstract/main.tex}
\newpage
\tableofcontents
\newpage

% Introduction of the project
\input{Introduction/main.tex}

% Game Theoretic Component
\input{Game_theory_component/main.tex}


\newpage
% Quick representation of the steps of methodology
\input{Methodology/Quick/main.tex}

\newpage
% Proper methodology
\input{Methodology/Proper/main.tex}

% Markov Chains
\input{MarkovChain/markov_chain_model/main.tex}
\newpage
\input{MarkovChain/closed_form_state_probs/main.tex}
\newpage
\input{MarkovChain/expressions_from_pi/main.tex}

\newpage
% Heatmap comparisons
\input{Comparisons/Example_model/main.tex}


\newpage
\input{Miscellaneous/Useful_tikz/main.tex}


% Formulas used
\newpage
\input{Miscellaneous/Formulas/main.tex}

\newpage
\printbibliography[title={References}]

\end{document}

\newpage
\tableofcontents
\newpage

% Introduction of the project
\documentclass{article}

\usepackage{amsmath} % For writing mathematics (align, split environments etc.)
\usepackage{mathtools}
% \usepackage{amsthm} % For the proof environment
\usepackage{amsfonts} 
\usepackage{geometry}
\usepackage{float}
\usepackage{graphicx}
\usepackage{soul}
\usepackage{indentfirst}
\usepackage{multicol}
\usepackage{tikz}
\usepackage{cancel}

\usetikzlibrary{calc, automata, chains, arrows.meta, math}
\setcounter{MaxMatrixCols}{20}

\usepackage{biblatex}
\addbibresource{bibliography.bib}


\title{A game theoretic model of the behavioural gaming that takes place at the EMS - ED interface}
\author{}
\date{}

\begin{document}
\maketitle

\input{Abstract/main.tex}
\newpage
\tableofcontents
\newpage

% Introduction of the project
\input{Introduction/main.tex}

% Game Theoretic Component
\input{Game_theory_component/main.tex}


\newpage
% Quick representation of the steps of methodology
\input{Methodology/Quick/main.tex}

\newpage
% Proper methodology
\input{Methodology/Proper/main.tex}

% Markov Chains
\input{MarkovChain/markov_chain_model/main.tex}
\newpage
\input{MarkovChain/closed_form_state_probs/main.tex}
\newpage
\input{MarkovChain/expressions_from_pi/main.tex}

\newpage
% Heatmap comparisons
\input{Comparisons/Example_model/main.tex}


\newpage
\input{Miscellaneous/Useful_tikz/main.tex}


% Formulas used
\newpage
\input{Miscellaneous/Formulas/main.tex}

\newpage
\printbibliography[title={References}]

\end{document}


% Game Theoretic Component
\documentclass{article}

\usepackage{amsmath} % For writing mathematics (align, split environments etc.)
\usepackage{mathtools}
% \usepackage{amsthm} % For the proof environment
\usepackage{amsfonts} 
\usepackage{geometry}
\usepackage{float}
\usepackage{graphicx}
\usepackage{soul}
\usepackage{indentfirst}
\usepackage{multicol}
\usepackage{tikz}
\usepackage{cancel}

\usetikzlibrary{calc, automata, chains, arrows.meta, math}
\setcounter{MaxMatrixCols}{20}

\usepackage{biblatex}
\addbibresource{bibliography.bib}


\title{A game theoretic model of the behavioural gaming that takes place at the EMS - ED interface}
\author{}
\date{}

\begin{document}
\maketitle

\input{Abstract/main.tex}
\newpage
\tableofcontents
\newpage

% Introduction of the project
\input{Introduction/main.tex}

% Game Theoretic Component
\input{Game_theory_component/main.tex}


\newpage
% Quick representation of the steps of methodology
\input{Methodology/Quick/main.tex}

\newpage
% Proper methodology
\input{Methodology/Proper/main.tex}

% Markov Chains
\input{MarkovChain/markov_chain_model/main.tex}
\newpage
\input{MarkovChain/closed_form_state_probs/main.tex}
\newpage
\input{MarkovChain/expressions_from_pi/main.tex}

\newpage
% Heatmap comparisons
\input{Comparisons/Example_model/main.tex}


\newpage
\input{Miscellaneous/Useful_tikz/main.tex}


% Formulas used
\newpage
\input{Miscellaneous/Formulas/main.tex}

\newpage
\printbibliography[title={References}]

\end{document}



\newpage
% Quick representation of the steps of methodology
\documentclass{article}

\usepackage{amsmath} % For writing mathematics (align, split environments etc.)
\usepackage{mathtools}
% \usepackage{amsthm} % For the proof environment
\usepackage{amsfonts} 
\usepackage{geometry}
\usepackage{float}
\usepackage{graphicx}
\usepackage{soul}
\usepackage{indentfirst}
\usepackage{multicol}
\usepackage{tikz}
\usepackage{cancel}

\usetikzlibrary{calc, automata, chains, arrows.meta, math}
\setcounter{MaxMatrixCols}{20}

\usepackage{biblatex}
\addbibresource{bibliography.bib}


\title{A game theoretic model of the behavioural gaming that takes place at the EMS - ED interface}
\author{}
\date{}

\begin{document}
\maketitle

\input{Abstract/main.tex}
\newpage
\tableofcontents
\newpage

% Introduction of the project
\input{Introduction/main.tex}

% Game Theoretic Component
\input{Game_theory_component/main.tex}


\newpage
% Quick representation of the steps of methodology
\input{Methodology/Quick/main.tex}

\newpage
% Proper methodology
\input{Methodology/Proper/main.tex}

% Markov Chains
\input{MarkovChain/markov_chain_model/main.tex}
\newpage
\input{MarkovChain/closed_form_state_probs/main.tex}
\newpage
\input{MarkovChain/expressions_from_pi/main.tex}

\newpage
% Heatmap comparisons
\input{Comparisons/Example_model/main.tex}


\newpage
\input{Miscellaneous/Useful_tikz/main.tex}


% Formulas used
\newpage
\input{Miscellaneous/Formulas/main.tex}

\newpage
\printbibliography[title={References}]

\end{document}


\newpage
% Proper methodology
\documentclass{article}

\usepackage{amsmath} % For writing mathematics (align, split environments etc.)
\usepackage{mathtools}
% \usepackage{amsthm} % For the proof environment
\usepackage{amsfonts} 
\usepackage{geometry}
\usepackage{float}
\usepackage{graphicx}
\usepackage{soul}
\usepackage{indentfirst}
\usepackage{multicol}
\usepackage{tikz}
\usepackage{cancel}

\usetikzlibrary{calc, automata, chains, arrows.meta, math}
\setcounter{MaxMatrixCols}{20}

\usepackage{biblatex}
\addbibresource{bibliography.bib}


\title{A game theoretic model of the behavioural gaming that takes place at the EMS - ED interface}
\author{}
\date{}

\begin{document}
\maketitle

\input{Abstract/main.tex}
\newpage
\tableofcontents
\newpage

% Introduction of the project
\input{Introduction/main.tex}

% Game Theoretic Component
\input{Game_theory_component/main.tex}


\newpage
% Quick representation of the steps of methodology
\input{Methodology/Quick/main.tex}

\newpage
% Proper methodology
\input{Methodology/Proper/main.tex}

% Markov Chains
\input{MarkovChain/markov_chain_model/main.tex}
\newpage
\input{MarkovChain/closed_form_state_probs/main.tex}
\newpage
\input{MarkovChain/expressions_from_pi/main.tex}

\newpage
% Heatmap comparisons
\input{Comparisons/Example_model/main.tex}


\newpage
\input{Miscellaneous/Useful_tikz/main.tex}


% Formulas used
\newpage
\input{Miscellaneous/Formulas/main.tex}

\newpage
\printbibliography[title={References}]

\end{document}


% Markov Chains
\documentclass{article}

\usepackage{amsmath} % For writing mathematics (align, split environments etc.)
\usepackage{mathtools}
% \usepackage{amsthm} % For the proof environment
\usepackage{amsfonts} 
\usepackage{geometry}
\usepackage{float}
\usepackage{graphicx}
\usepackage{soul}
\usepackage{indentfirst}
\usepackage{multicol}
\usepackage{tikz}
\usepackage{cancel}

\usetikzlibrary{calc, automata, chains, arrows.meta, math}
\setcounter{MaxMatrixCols}{20}

\usepackage{biblatex}
\addbibresource{bibliography.bib}


\title{A game theoretic model of the behavioural gaming that takes place at the EMS - ED interface}
\author{}
\date{}

\begin{document}
\maketitle

\input{Abstract/main.tex}
\newpage
\tableofcontents
\newpage

% Introduction of the project
\input{Introduction/main.tex}

% Game Theoretic Component
\input{Game_theory_component/main.tex}


\newpage
% Quick representation of the steps of methodology
\input{Methodology/Quick/main.tex}

\newpage
% Proper methodology
\input{Methodology/Proper/main.tex}

% Markov Chains
\input{MarkovChain/markov_chain_model/main.tex}
\newpage
\input{MarkovChain/closed_form_state_probs/main.tex}
\newpage
\input{MarkovChain/expressions_from_pi/main.tex}

\newpage
% Heatmap comparisons
\input{Comparisons/Example_model/main.tex}


\newpage
\input{Miscellaneous/Useful_tikz/main.tex}


% Formulas used
\newpage
\input{Miscellaneous/Formulas/main.tex}

\newpage
\printbibliography[title={References}]

\end{document}

\newpage
\documentclass{article}

\usepackage{amsmath} % For writing mathematics (align, split environments etc.)
\usepackage{mathtools}
% \usepackage{amsthm} % For the proof environment
\usepackage{amsfonts} 
\usepackage{geometry}
\usepackage{float}
\usepackage{graphicx}
\usepackage{soul}
\usepackage{indentfirst}
\usepackage{multicol}
\usepackage{tikz}
\usepackage{cancel}

\usetikzlibrary{calc, automata, chains, arrows.meta, math}
\setcounter{MaxMatrixCols}{20}

\usepackage{biblatex}
\addbibresource{bibliography.bib}


\title{A game theoretic model of the behavioural gaming that takes place at the EMS - ED interface}
\author{}
\date{}

\begin{document}
\maketitle

\input{Abstract/main.tex}
\newpage
\tableofcontents
\newpage

% Introduction of the project
\input{Introduction/main.tex}

% Game Theoretic Component
\input{Game_theory_component/main.tex}


\newpage
% Quick representation of the steps of methodology
\input{Methodology/Quick/main.tex}

\newpage
% Proper methodology
\input{Methodology/Proper/main.tex}

% Markov Chains
\input{MarkovChain/markov_chain_model/main.tex}
\newpage
\input{MarkovChain/closed_form_state_probs/main.tex}
\newpage
\input{MarkovChain/expressions_from_pi/main.tex}

\newpage
% Heatmap comparisons
\input{Comparisons/Example_model/main.tex}


\newpage
\input{Miscellaneous/Useful_tikz/main.tex}


% Formulas used
\newpage
\input{Miscellaneous/Formulas/main.tex}

\newpage
\printbibliography[title={References}]

\end{document}

\newpage
\documentclass{article}

\usepackage{amsmath} % For writing mathematics (align, split environments etc.)
\usepackage{mathtools}
% \usepackage{amsthm} % For the proof environment
\usepackage{amsfonts} 
\usepackage{geometry}
\usepackage{float}
\usepackage{graphicx}
\usepackage{soul}
\usepackage{indentfirst}
\usepackage{multicol}
\usepackage{tikz}
\usepackage{cancel}

\usetikzlibrary{calc, automata, chains, arrows.meta, math}
\setcounter{MaxMatrixCols}{20}

\usepackage{biblatex}
\addbibresource{bibliography.bib}


\title{A game theoretic model of the behavioural gaming that takes place at the EMS - ED interface}
\author{}
\date{}

\begin{document}
\maketitle

\input{Abstract/main.tex}
\newpage
\tableofcontents
\newpage

% Introduction of the project
\input{Introduction/main.tex}

% Game Theoretic Component
\input{Game_theory_component/main.tex}


\newpage
% Quick representation of the steps of methodology
\input{Methodology/Quick/main.tex}

\newpage
% Proper methodology
\input{Methodology/Proper/main.tex}

% Markov Chains
\input{MarkovChain/markov_chain_model/main.tex}
\newpage
\input{MarkovChain/closed_form_state_probs/main.tex}
\newpage
\input{MarkovChain/expressions_from_pi/main.tex}

\newpage
% Heatmap comparisons
\input{Comparisons/Example_model/main.tex}


\newpage
\input{Miscellaneous/Useful_tikz/main.tex}


% Formulas used
\newpage
\input{Miscellaneous/Formulas/main.tex}

\newpage
\printbibliography[title={References}]

\end{document}


\newpage
% Heatmap comparisons
\documentclass{article}

\usepackage{amsmath} % For writing mathematics (align, split environments etc.)
\usepackage{mathtools}
% \usepackage{amsthm} % For the proof environment
\usepackage{amsfonts} 
\usepackage{geometry}
\usepackage{float}
\usepackage{graphicx}
\usepackage{soul}
\usepackage{indentfirst}
\usepackage{multicol}
\usepackage{tikz}
\usepackage{cancel}

\usetikzlibrary{calc, automata, chains, arrows.meta, math}
\setcounter{MaxMatrixCols}{20}

\usepackage{biblatex}
\addbibresource{bibliography.bib}


\title{A game theoretic model of the behavioural gaming that takes place at the EMS - ED interface}
\author{}
\date{}

\begin{document}
\maketitle

\input{Abstract/main.tex}
\newpage
\tableofcontents
\newpage

% Introduction of the project
\input{Introduction/main.tex}

% Game Theoretic Component
\input{Game_theory_component/main.tex}


\newpage
% Quick representation of the steps of methodology
\input{Methodology/Quick/main.tex}

\newpage
% Proper methodology
\input{Methodology/Proper/main.tex}

% Markov Chains
\input{MarkovChain/markov_chain_model/main.tex}
\newpage
\input{MarkovChain/closed_form_state_probs/main.tex}
\newpage
\input{MarkovChain/expressions_from_pi/main.tex}

\newpage
% Heatmap comparisons
\input{Comparisons/Example_model/main.tex}


\newpage
\input{Miscellaneous/Useful_tikz/main.tex}


% Formulas used
\newpage
\input{Miscellaneous/Formulas/main.tex}

\newpage
\printbibliography[title={References}]

\end{document}



\newpage
\documentclass{article}

\usepackage{amsmath} % For writing mathematics (align, split environments etc.)
\usepackage{mathtools}
% \usepackage{amsthm} % For the proof environment
\usepackage{amsfonts} 
\usepackage{geometry}
\usepackage{float}
\usepackage{graphicx}
\usepackage{soul}
\usepackage{indentfirst}
\usepackage{multicol}
\usepackage{tikz}
\usepackage{cancel}

\usetikzlibrary{calc, automata, chains, arrows.meta, math}
\setcounter{MaxMatrixCols}{20}

\usepackage{biblatex}
\addbibresource{bibliography.bib}


\title{A game theoretic model of the behavioural gaming that takes place at the EMS - ED interface}
\author{}
\date{}

\begin{document}
\maketitle

\input{Abstract/main.tex}
\newpage
\tableofcontents
\newpage

% Introduction of the project
\input{Introduction/main.tex}

% Game Theoretic Component
\input{Game_theory_component/main.tex}


\newpage
% Quick representation of the steps of methodology
\input{Methodology/Quick/main.tex}

\newpage
% Proper methodology
\input{Methodology/Proper/main.tex}

% Markov Chains
\input{MarkovChain/markov_chain_model/main.tex}
\newpage
\input{MarkovChain/closed_form_state_probs/main.tex}
\newpage
\input{MarkovChain/expressions_from_pi/main.tex}

\newpage
% Heatmap comparisons
\input{Comparisons/Example_model/main.tex}


\newpage
\input{Miscellaneous/Useful_tikz/main.tex}


% Formulas used
\newpage
\input{Miscellaneous/Formulas/main.tex}

\newpage
\printbibliography[title={References}]

\end{document}



% Formulas used
\newpage
\documentclass{article}

\usepackage{amsmath} % For writing mathematics (align, split environments etc.)
\usepackage{mathtools}
% \usepackage{amsthm} % For the proof environment
\usepackage{amsfonts} 
\usepackage{geometry}
\usepackage{float}
\usepackage{graphicx}
\usepackage{soul}
\usepackage{indentfirst}
\usepackage{multicol}
\usepackage{tikz}
\usepackage{cancel}

\usetikzlibrary{calc, automata, chains, arrows.meta, math}
\setcounter{MaxMatrixCols}{20}

\usepackage{biblatex}
\addbibresource{bibliography.bib}


\title{A game theoretic model of the behavioural gaming that takes place at the EMS - ED interface}
\author{}
\date{}

\begin{document}
\maketitle

\input{Abstract/main.tex}
\newpage
\tableofcontents
\newpage

% Introduction of the project
\input{Introduction/main.tex}

% Game Theoretic Component
\input{Game_theory_component/main.tex}


\newpage
% Quick representation of the steps of methodology
\input{Methodology/Quick/main.tex}

\newpage
% Proper methodology
\input{Methodology/Proper/main.tex}

% Markov Chains
\input{MarkovChain/markov_chain_model/main.tex}
\newpage
\input{MarkovChain/closed_form_state_probs/main.tex}
\newpage
\input{MarkovChain/expressions_from_pi/main.tex}

\newpage
% Heatmap comparisons
\input{Comparisons/Example_model/main.tex}


\newpage
\input{Miscellaneous/Useful_tikz/main.tex}


% Formulas used
\newpage
\input{Miscellaneous/Formulas/main.tex}

\newpage
\printbibliography[title={References}]

\end{document}


\newpage
\printbibliography[title={References}]

\end{document}


\newpage
% Proper methodology
\documentclass{article}

\usepackage{amsmath} % For writing mathematics (align, split environments etc.)
\usepackage{mathtools}
% \usepackage{amsthm} % For the proof environment
\usepackage{amsfonts} 
\usepackage{geometry}
\usepackage{float}
\usepackage{graphicx}
\usepackage{soul}
\usepackage{indentfirst}
\usepackage{multicol}
\usepackage{tikz}
\usepackage{cancel}

\usetikzlibrary{calc, automata, chains, arrows.meta, math}
\setcounter{MaxMatrixCols}{20}

\usepackage{biblatex}
\addbibresource{bibliography.bib}


\title{A game theoretic model of the behavioural gaming that takes place at the EMS - ED interface}
\author{}
\date{}

\begin{document}
\maketitle

\documentclass{article}

\usepackage{amsmath} % For writing mathematics (align, split environments etc.)
\usepackage{mathtools}
% \usepackage{amsthm} % For the proof environment
\usepackage{amsfonts} 
\usepackage{geometry}
\usepackage{float}
\usepackage{graphicx}
\usepackage{soul}
\usepackage{indentfirst}
\usepackage{multicol}
\usepackage{tikz}
\usepackage{cancel}

\usetikzlibrary{calc, automata, chains, arrows.meta, math}
\setcounter{MaxMatrixCols}{20}

\usepackage{biblatex}
\addbibresource{bibliography.bib}


\title{A game theoretic model of the behavioural gaming that takes place at the EMS - ED interface}
\author{}
\date{}

\begin{document}
\maketitle

\input{Abstract/main.tex}
\newpage
\tableofcontents
\newpage

% Introduction of the project
\input{Introduction/main.tex}

% Game Theoretic Component
\input{Game_theory_component/main.tex}


\newpage
% Quick representation of the steps of methodology
\input{Methodology/Quick/main.tex}

\newpage
% Proper methodology
\input{Methodology/Proper/main.tex}

% Markov Chains
\input{MarkovChain/markov_chain_model/main.tex}
\newpage
\input{MarkovChain/closed_form_state_probs/main.tex}
\newpage
\input{MarkovChain/expressions_from_pi/main.tex}

\newpage
% Heatmap comparisons
\input{Comparisons/Example_model/main.tex}


\newpage
\input{Miscellaneous/Useful_tikz/main.tex}


% Formulas used
\newpage
\input{Miscellaneous/Formulas/main.tex}

\newpage
\printbibliography[title={References}]

\end{document}

\newpage
\tableofcontents
\newpage

% Introduction of the project
\documentclass{article}

\usepackage{amsmath} % For writing mathematics (align, split environments etc.)
\usepackage{mathtools}
% \usepackage{amsthm} % For the proof environment
\usepackage{amsfonts} 
\usepackage{geometry}
\usepackage{float}
\usepackage{graphicx}
\usepackage{soul}
\usepackage{indentfirst}
\usepackage{multicol}
\usepackage{tikz}
\usepackage{cancel}

\usetikzlibrary{calc, automata, chains, arrows.meta, math}
\setcounter{MaxMatrixCols}{20}

\usepackage{biblatex}
\addbibresource{bibliography.bib}


\title{A game theoretic model of the behavioural gaming that takes place at the EMS - ED interface}
\author{}
\date{}

\begin{document}
\maketitle

\input{Abstract/main.tex}
\newpage
\tableofcontents
\newpage

% Introduction of the project
\input{Introduction/main.tex}

% Game Theoretic Component
\input{Game_theory_component/main.tex}


\newpage
% Quick representation of the steps of methodology
\input{Methodology/Quick/main.tex}

\newpage
% Proper methodology
\input{Methodology/Proper/main.tex}

% Markov Chains
\input{MarkovChain/markov_chain_model/main.tex}
\newpage
\input{MarkovChain/closed_form_state_probs/main.tex}
\newpage
\input{MarkovChain/expressions_from_pi/main.tex}

\newpage
% Heatmap comparisons
\input{Comparisons/Example_model/main.tex}


\newpage
\input{Miscellaneous/Useful_tikz/main.tex}


% Formulas used
\newpage
\input{Miscellaneous/Formulas/main.tex}

\newpage
\printbibliography[title={References}]

\end{document}


% Game Theoretic Component
\documentclass{article}

\usepackage{amsmath} % For writing mathematics (align, split environments etc.)
\usepackage{mathtools}
% \usepackage{amsthm} % For the proof environment
\usepackage{amsfonts} 
\usepackage{geometry}
\usepackage{float}
\usepackage{graphicx}
\usepackage{soul}
\usepackage{indentfirst}
\usepackage{multicol}
\usepackage{tikz}
\usepackage{cancel}

\usetikzlibrary{calc, automata, chains, arrows.meta, math}
\setcounter{MaxMatrixCols}{20}

\usepackage{biblatex}
\addbibresource{bibliography.bib}


\title{A game theoretic model of the behavioural gaming that takes place at the EMS - ED interface}
\author{}
\date{}

\begin{document}
\maketitle

\input{Abstract/main.tex}
\newpage
\tableofcontents
\newpage

% Introduction of the project
\input{Introduction/main.tex}

% Game Theoretic Component
\input{Game_theory_component/main.tex}


\newpage
% Quick representation of the steps of methodology
\input{Methodology/Quick/main.tex}

\newpage
% Proper methodology
\input{Methodology/Proper/main.tex}

% Markov Chains
\input{MarkovChain/markov_chain_model/main.tex}
\newpage
\input{MarkovChain/closed_form_state_probs/main.tex}
\newpage
\input{MarkovChain/expressions_from_pi/main.tex}

\newpage
% Heatmap comparisons
\input{Comparisons/Example_model/main.tex}


\newpage
\input{Miscellaneous/Useful_tikz/main.tex}


% Formulas used
\newpage
\input{Miscellaneous/Formulas/main.tex}

\newpage
\printbibliography[title={References}]

\end{document}



\newpage
% Quick representation of the steps of methodology
\documentclass{article}

\usepackage{amsmath} % For writing mathematics (align, split environments etc.)
\usepackage{mathtools}
% \usepackage{amsthm} % For the proof environment
\usepackage{amsfonts} 
\usepackage{geometry}
\usepackage{float}
\usepackage{graphicx}
\usepackage{soul}
\usepackage{indentfirst}
\usepackage{multicol}
\usepackage{tikz}
\usepackage{cancel}

\usetikzlibrary{calc, automata, chains, arrows.meta, math}
\setcounter{MaxMatrixCols}{20}

\usepackage{biblatex}
\addbibresource{bibliography.bib}


\title{A game theoretic model of the behavioural gaming that takes place at the EMS - ED interface}
\author{}
\date{}

\begin{document}
\maketitle

\input{Abstract/main.tex}
\newpage
\tableofcontents
\newpage

% Introduction of the project
\input{Introduction/main.tex}

% Game Theoretic Component
\input{Game_theory_component/main.tex}


\newpage
% Quick representation of the steps of methodology
\input{Methodology/Quick/main.tex}

\newpage
% Proper methodology
\input{Methodology/Proper/main.tex}

% Markov Chains
\input{MarkovChain/markov_chain_model/main.tex}
\newpage
\input{MarkovChain/closed_form_state_probs/main.tex}
\newpage
\input{MarkovChain/expressions_from_pi/main.tex}

\newpage
% Heatmap comparisons
\input{Comparisons/Example_model/main.tex}


\newpage
\input{Miscellaneous/Useful_tikz/main.tex}


% Formulas used
\newpage
\input{Miscellaneous/Formulas/main.tex}

\newpage
\printbibliography[title={References}]

\end{document}


\newpage
% Proper methodology
\documentclass{article}

\usepackage{amsmath} % For writing mathematics (align, split environments etc.)
\usepackage{mathtools}
% \usepackage{amsthm} % For the proof environment
\usepackage{amsfonts} 
\usepackage{geometry}
\usepackage{float}
\usepackage{graphicx}
\usepackage{soul}
\usepackage{indentfirst}
\usepackage{multicol}
\usepackage{tikz}
\usepackage{cancel}

\usetikzlibrary{calc, automata, chains, arrows.meta, math}
\setcounter{MaxMatrixCols}{20}

\usepackage{biblatex}
\addbibresource{bibliography.bib}


\title{A game theoretic model of the behavioural gaming that takes place at the EMS - ED interface}
\author{}
\date{}

\begin{document}
\maketitle

\input{Abstract/main.tex}
\newpage
\tableofcontents
\newpage

% Introduction of the project
\input{Introduction/main.tex}

% Game Theoretic Component
\input{Game_theory_component/main.tex}


\newpage
% Quick representation of the steps of methodology
\input{Methodology/Quick/main.tex}

\newpage
% Proper methodology
\input{Methodology/Proper/main.tex}

% Markov Chains
\input{MarkovChain/markov_chain_model/main.tex}
\newpage
\input{MarkovChain/closed_form_state_probs/main.tex}
\newpage
\input{MarkovChain/expressions_from_pi/main.tex}

\newpage
% Heatmap comparisons
\input{Comparisons/Example_model/main.tex}


\newpage
\input{Miscellaneous/Useful_tikz/main.tex}


% Formulas used
\newpage
\input{Miscellaneous/Formulas/main.tex}

\newpage
\printbibliography[title={References}]

\end{document}


% Markov Chains
\documentclass{article}

\usepackage{amsmath} % For writing mathematics (align, split environments etc.)
\usepackage{mathtools}
% \usepackage{amsthm} % For the proof environment
\usepackage{amsfonts} 
\usepackage{geometry}
\usepackage{float}
\usepackage{graphicx}
\usepackage{soul}
\usepackage{indentfirst}
\usepackage{multicol}
\usepackage{tikz}
\usepackage{cancel}

\usetikzlibrary{calc, automata, chains, arrows.meta, math}
\setcounter{MaxMatrixCols}{20}

\usepackage{biblatex}
\addbibresource{bibliography.bib}


\title{A game theoretic model of the behavioural gaming that takes place at the EMS - ED interface}
\author{}
\date{}

\begin{document}
\maketitle

\input{Abstract/main.tex}
\newpage
\tableofcontents
\newpage

% Introduction of the project
\input{Introduction/main.tex}

% Game Theoretic Component
\input{Game_theory_component/main.tex}


\newpage
% Quick representation of the steps of methodology
\input{Methodology/Quick/main.tex}

\newpage
% Proper methodology
\input{Methodology/Proper/main.tex}

% Markov Chains
\input{MarkovChain/markov_chain_model/main.tex}
\newpage
\input{MarkovChain/closed_form_state_probs/main.tex}
\newpage
\input{MarkovChain/expressions_from_pi/main.tex}

\newpage
% Heatmap comparisons
\input{Comparisons/Example_model/main.tex}


\newpage
\input{Miscellaneous/Useful_tikz/main.tex}


% Formulas used
\newpage
\input{Miscellaneous/Formulas/main.tex}

\newpage
\printbibliography[title={References}]

\end{document}

\newpage
\documentclass{article}

\usepackage{amsmath} % For writing mathematics (align, split environments etc.)
\usepackage{mathtools}
% \usepackage{amsthm} % For the proof environment
\usepackage{amsfonts} 
\usepackage{geometry}
\usepackage{float}
\usepackage{graphicx}
\usepackage{soul}
\usepackage{indentfirst}
\usepackage{multicol}
\usepackage{tikz}
\usepackage{cancel}

\usetikzlibrary{calc, automata, chains, arrows.meta, math}
\setcounter{MaxMatrixCols}{20}

\usepackage{biblatex}
\addbibresource{bibliography.bib}


\title{A game theoretic model of the behavioural gaming that takes place at the EMS - ED interface}
\author{}
\date{}

\begin{document}
\maketitle

\input{Abstract/main.tex}
\newpage
\tableofcontents
\newpage

% Introduction of the project
\input{Introduction/main.tex}

% Game Theoretic Component
\input{Game_theory_component/main.tex}


\newpage
% Quick representation of the steps of methodology
\input{Methodology/Quick/main.tex}

\newpage
% Proper methodology
\input{Methodology/Proper/main.tex}

% Markov Chains
\input{MarkovChain/markov_chain_model/main.tex}
\newpage
\input{MarkovChain/closed_form_state_probs/main.tex}
\newpage
\input{MarkovChain/expressions_from_pi/main.tex}

\newpage
% Heatmap comparisons
\input{Comparisons/Example_model/main.tex}


\newpage
\input{Miscellaneous/Useful_tikz/main.tex}


% Formulas used
\newpage
\input{Miscellaneous/Formulas/main.tex}

\newpage
\printbibliography[title={References}]

\end{document}

\newpage
\documentclass{article}

\usepackage{amsmath} % For writing mathematics (align, split environments etc.)
\usepackage{mathtools}
% \usepackage{amsthm} % For the proof environment
\usepackage{amsfonts} 
\usepackage{geometry}
\usepackage{float}
\usepackage{graphicx}
\usepackage{soul}
\usepackage{indentfirst}
\usepackage{multicol}
\usepackage{tikz}
\usepackage{cancel}

\usetikzlibrary{calc, automata, chains, arrows.meta, math}
\setcounter{MaxMatrixCols}{20}

\usepackage{biblatex}
\addbibresource{bibliography.bib}


\title{A game theoretic model of the behavioural gaming that takes place at the EMS - ED interface}
\author{}
\date{}

\begin{document}
\maketitle

\input{Abstract/main.tex}
\newpage
\tableofcontents
\newpage

% Introduction of the project
\input{Introduction/main.tex}

% Game Theoretic Component
\input{Game_theory_component/main.tex}


\newpage
% Quick representation of the steps of methodology
\input{Methodology/Quick/main.tex}

\newpage
% Proper methodology
\input{Methodology/Proper/main.tex}

% Markov Chains
\input{MarkovChain/markov_chain_model/main.tex}
\newpage
\input{MarkovChain/closed_form_state_probs/main.tex}
\newpage
\input{MarkovChain/expressions_from_pi/main.tex}

\newpage
% Heatmap comparisons
\input{Comparisons/Example_model/main.tex}


\newpage
\input{Miscellaneous/Useful_tikz/main.tex}


% Formulas used
\newpage
\input{Miscellaneous/Formulas/main.tex}

\newpage
\printbibliography[title={References}]

\end{document}


\newpage
% Heatmap comparisons
\documentclass{article}

\usepackage{amsmath} % For writing mathematics (align, split environments etc.)
\usepackage{mathtools}
% \usepackage{amsthm} % For the proof environment
\usepackage{amsfonts} 
\usepackage{geometry}
\usepackage{float}
\usepackage{graphicx}
\usepackage{soul}
\usepackage{indentfirst}
\usepackage{multicol}
\usepackage{tikz}
\usepackage{cancel}

\usetikzlibrary{calc, automata, chains, arrows.meta, math}
\setcounter{MaxMatrixCols}{20}

\usepackage{biblatex}
\addbibresource{bibliography.bib}


\title{A game theoretic model of the behavioural gaming that takes place at the EMS - ED interface}
\author{}
\date{}

\begin{document}
\maketitle

\input{Abstract/main.tex}
\newpage
\tableofcontents
\newpage

% Introduction of the project
\input{Introduction/main.tex}

% Game Theoretic Component
\input{Game_theory_component/main.tex}


\newpage
% Quick representation of the steps of methodology
\input{Methodology/Quick/main.tex}

\newpage
% Proper methodology
\input{Methodology/Proper/main.tex}

% Markov Chains
\input{MarkovChain/markov_chain_model/main.tex}
\newpage
\input{MarkovChain/closed_form_state_probs/main.tex}
\newpage
\input{MarkovChain/expressions_from_pi/main.tex}

\newpage
% Heatmap comparisons
\input{Comparisons/Example_model/main.tex}


\newpage
\input{Miscellaneous/Useful_tikz/main.tex}


% Formulas used
\newpage
\input{Miscellaneous/Formulas/main.tex}

\newpage
\printbibliography[title={References}]

\end{document}



\newpage
\documentclass{article}

\usepackage{amsmath} % For writing mathematics (align, split environments etc.)
\usepackage{mathtools}
% \usepackage{amsthm} % For the proof environment
\usepackage{amsfonts} 
\usepackage{geometry}
\usepackage{float}
\usepackage{graphicx}
\usepackage{soul}
\usepackage{indentfirst}
\usepackage{multicol}
\usepackage{tikz}
\usepackage{cancel}

\usetikzlibrary{calc, automata, chains, arrows.meta, math}
\setcounter{MaxMatrixCols}{20}

\usepackage{biblatex}
\addbibresource{bibliography.bib}


\title{A game theoretic model of the behavioural gaming that takes place at the EMS - ED interface}
\author{}
\date{}

\begin{document}
\maketitle

\input{Abstract/main.tex}
\newpage
\tableofcontents
\newpage

% Introduction of the project
\input{Introduction/main.tex}

% Game Theoretic Component
\input{Game_theory_component/main.tex}


\newpage
% Quick representation of the steps of methodology
\input{Methodology/Quick/main.tex}

\newpage
% Proper methodology
\input{Methodology/Proper/main.tex}

% Markov Chains
\input{MarkovChain/markov_chain_model/main.tex}
\newpage
\input{MarkovChain/closed_form_state_probs/main.tex}
\newpage
\input{MarkovChain/expressions_from_pi/main.tex}

\newpage
% Heatmap comparisons
\input{Comparisons/Example_model/main.tex}


\newpage
\input{Miscellaneous/Useful_tikz/main.tex}


% Formulas used
\newpage
\input{Miscellaneous/Formulas/main.tex}

\newpage
\printbibliography[title={References}]

\end{document}



% Formulas used
\newpage
\documentclass{article}

\usepackage{amsmath} % For writing mathematics (align, split environments etc.)
\usepackage{mathtools}
% \usepackage{amsthm} % For the proof environment
\usepackage{amsfonts} 
\usepackage{geometry}
\usepackage{float}
\usepackage{graphicx}
\usepackage{soul}
\usepackage{indentfirst}
\usepackage{multicol}
\usepackage{tikz}
\usepackage{cancel}

\usetikzlibrary{calc, automata, chains, arrows.meta, math}
\setcounter{MaxMatrixCols}{20}

\usepackage{biblatex}
\addbibresource{bibliography.bib}


\title{A game theoretic model of the behavioural gaming that takes place at the EMS - ED interface}
\author{}
\date{}

\begin{document}
\maketitle

\input{Abstract/main.tex}
\newpage
\tableofcontents
\newpage

% Introduction of the project
\input{Introduction/main.tex}

% Game Theoretic Component
\input{Game_theory_component/main.tex}


\newpage
% Quick representation of the steps of methodology
\input{Methodology/Quick/main.tex}

\newpage
% Proper methodology
\input{Methodology/Proper/main.tex}

% Markov Chains
\input{MarkovChain/markov_chain_model/main.tex}
\newpage
\input{MarkovChain/closed_form_state_probs/main.tex}
\newpage
\input{MarkovChain/expressions_from_pi/main.tex}

\newpage
% Heatmap comparisons
\input{Comparisons/Example_model/main.tex}


\newpage
\input{Miscellaneous/Useful_tikz/main.tex}


% Formulas used
\newpage
\input{Miscellaneous/Formulas/main.tex}

\newpage
\printbibliography[title={References}]

\end{document}


\newpage
\printbibliography[title={References}]

\end{document}


% Markov Chains
\documentclass{article}

\usepackage{amsmath} % For writing mathematics (align, split environments etc.)
\usepackage{mathtools}
% \usepackage{amsthm} % For the proof environment
\usepackage{amsfonts} 
\usepackage{geometry}
\usepackage{float}
\usepackage{graphicx}
\usepackage{soul}
\usepackage{indentfirst}
\usepackage{multicol}
\usepackage{tikz}
\usepackage{cancel}

\usetikzlibrary{calc, automata, chains, arrows.meta, math}
\setcounter{MaxMatrixCols}{20}

\usepackage{biblatex}
\addbibresource{bibliography.bib}


\title{A game theoretic model of the behavioural gaming that takes place at the EMS - ED interface}
\author{}
\date{}

\begin{document}
\maketitle

\documentclass{article}

\usepackage{amsmath} % For writing mathematics (align, split environments etc.)
\usepackage{mathtools}
% \usepackage{amsthm} % For the proof environment
\usepackage{amsfonts} 
\usepackage{geometry}
\usepackage{float}
\usepackage{graphicx}
\usepackage{soul}
\usepackage{indentfirst}
\usepackage{multicol}
\usepackage{tikz}
\usepackage{cancel}

\usetikzlibrary{calc, automata, chains, arrows.meta, math}
\setcounter{MaxMatrixCols}{20}

\usepackage{biblatex}
\addbibresource{bibliography.bib}


\title{A game theoretic model of the behavioural gaming that takes place at the EMS - ED interface}
\author{}
\date{}

\begin{document}
\maketitle

\input{Abstract/main.tex}
\newpage
\tableofcontents
\newpage

% Introduction of the project
\input{Introduction/main.tex}

% Game Theoretic Component
\input{Game_theory_component/main.tex}


\newpage
% Quick representation of the steps of methodology
\input{Methodology/Quick/main.tex}

\newpage
% Proper methodology
\input{Methodology/Proper/main.tex}

% Markov Chains
\input{MarkovChain/markov_chain_model/main.tex}
\newpage
\input{MarkovChain/closed_form_state_probs/main.tex}
\newpage
\input{MarkovChain/expressions_from_pi/main.tex}

\newpage
% Heatmap comparisons
\input{Comparisons/Example_model/main.tex}


\newpage
\input{Miscellaneous/Useful_tikz/main.tex}


% Formulas used
\newpage
\input{Miscellaneous/Formulas/main.tex}

\newpage
\printbibliography[title={References}]

\end{document}

\newpage
\tableofcontents
\newpage

% Introduction of the project
\documentclass{article}

\usepackage{amsmath} % For writing mathematics (align, split environments etc.)
\usepackage{mathtools}
% \usepackage{amsthm} % For the proof environment
\usepackage{amsfonts} 
\usepackage{geometry}
\usepackage{float}
\usepackage{graphicx}
\usepackage{soul}
\usepackage{indentfirst}
\usepackage{multicol}
\usepackage{tikz}
\usepackage{cancel}

\usetikzlibrary{calc, automata, chains, arrows.meta, math}
\setcounter{MaxMatrixCols}{20}

\usepackage{biblatex}
\addbibresource{bibliography.bib}


\title{A game theoretic model of the behavioural gaming that takes place at the EMS - ED interface}
\author{}
\date{}

\begin{document}
\maketitle

\input{Abstract/main.tex}
\newpage
\tableofcontents
\newpage

% Introduction of the project
\input{Introduction/main.tex}

% Game Theoretic Component
\input{Game_theory_component/main.tex}


\newpage
% Quick representation of the steps of methodology
\input{Methodology/Quick/main.tex}

\newpage
% Proper methodology
\input{Methodology/Proper/main.tex}

% Markov Chains
\input{MarkovChain/markov_chain_model/main.tex}
\newpage
\input{MarkovChain/closed_form_state_probs/main.tex}
\newpage
\input{MarkovChain/expressions_from_pi/main.tex}

\newpage
% Heatmap comparisons
\input{Comparisons/Example_model/main.tex}


\newpage
\input{Miscellaneous/Useful_tikz/main.tex}


% Formulas used
\newpage
\input{Miscellaneous/Formulas/main.tex}

\newpage
\printbibliography[title={References}]

\end{document}


% Game Theoretic Component
\documentclass{article}

\usepackage{amsmath} % For writing mathematics (align, split environments etc.)
\usepackage{mathtools}
% \usepackage{amsthm} % For the proof environment
\usepackage{amsfonts} 
\usepackage{geometry}
\usepackage{float}
\usepackage{graphicx}
\usepackage{soul}
\usepackage{indentfirst}
\usepackage{multicol}
\usepackage{tikz}
\usepackage{cancel}

\usetikzlibrary{calc, automata, chains, arrows.meta, math}
\setcounter{MaxMatrixCols}{20}

\usepackage{biblatex}
\addbibresource{bibliography.bib}


\title{A game theoretic model of the behavioural gaming that takes place at the EMS - ED interface}
\author{}
\date{}

\begin{document}
\maketitle

\input{Abstract/main.tex}
\newpage
\tableofcontents
\newpage

% Introduction of the project
\input{Introduction/main.tex}

% Game Theoretic Component
\input{Game_theory_component/main.tex}


\newpage
% Quick representation of the steps of methodology
\input{Methodology/Quick/main.tex}

\newpage
% Proper methodology
\input{Methodology/Proper/main.tex}

% Markov Chains
\input{MarkovChain/markov_chain_model/main.tex}
\newpage
\input{MarkovChain/closed_form_state_probs/main.tex}
\newpage
\input{MarkovChain/expressions_from_pi/main.tex}

\newpage
% Heatmap comparisons
\input{Comparisons/Example_model/main.tex}


\newpage
\input{Miscellaneous/Useful_tikz/main.tex}


% Formulas used
\newpage
\input{Miscellaneous/Formulas/main.tex}

\newpage
\printbibliography[title={References}]

\end{document}



\newpage
% Quick representation of the steps of methodology
\documentclass{article}

\usepackage{amsmath} % For writing mathematics (align, split environments etc.)
\usepackage{mathtools}
% \usepackage{amsthm} % For the proof environment
\usepackage{amsfonts} 
\usepackage{geometry}
\usepackage{float}
\usepackage{graphicx}
\usepackage{soul}
\usepackage{indentfirst}
\usepackage{multicol}
\usepackage{tikz}
\usepackage{cancel}

\usetikzlibrary{calc, automata, chains, arrows.meta, math}
\setcounter{MaxMatrixCols}{20}

\usepackage{biblatex}
\addbibresource{bibliography.bib}


\title{A game theoretic model of the behavioural gaming that takes place at the EMS - ED interface}
\author{}
\date{}

\begin{document}
\maketitle

\input{Abstract/main.tex}
\newpage
\tableofcontents
\newpage

% Introduction of the project
\input{Introduction/main.tex}

% Game Theoretic Component
\input{Game_theory_component/main.tex}


\newpage
% Quick representation of the steps of methodology
\input{Methodology/Quick/main.tex}

\newpage
% Proper methodology
\input{Methodology/Proper/main.tex}

% Markov Chains
\input{MarkovChain/markov_chain_model/main.tex}
\newpage
\input{MarkovChain/closed_form_state_probs/main.tex}
\newpage
\input{MarkovChain/expressions_from_pi/main.tex}

\newpage
% Heatmap comparisons
\input{Comparisons/Example_model/main.tex}


\newpage
\input{Miscellaneous/Useful_tikz/main.tex}


% Formulas used
\newpage
\input{Miscellaneous/Formulas/main.tex}

\newpage
\printbibliography[title={References}]

\end{document}


\newpage
% Proper methodology
\documentclass{article}

\usepackage{amsmath} % For writing mathematics (align, split environments etc.)
\usepackage{mathtools}
% \usepackage{amsthm} % For the proof environment
\usepackage{amsfonts} 
\usepackage{geometry}
\usepackage{float}
\usepackage{graphicx}
\usepackage{soul}
\usepackage{indentfirst}
\usepackage{multicol}
\usepackage{tikz}
\usepackage{cancel}

\usetikzlibrary{calc, automata, chains, arrows.meta, math}
\setcounter{MaxMatrixCols}{20}

\usepackage{biblatex}
\addbibresource{bibliography.bib}


\title{A game theoretic model of the behavioural gaming that takes place at the EMS - ED interface}
\author{}
\date{}

\begin{document}
\maketitle

\input{Abstract/main.tex}
\newpage
\tableofcontents
\newpage

% Introduction of the project
\input{Introduction/main.tex}

% Game Theoretic Component
\input{Game_theory_component/main.tex}


\newpage
% Quick representation of the steps of methodology
\input{Methodology/Quick/main.tex}

\newpage
% Proper methodology
\input{Methodology/Proper/main.tex}

% Markov Chains
\input{MarkovChain/markov_chain_model/main.tex}
\newpage
\input{MarkovChain/closed_form_state_probs/main.tex}
\newpage
\input{MarkovChain/expressions_from_pi/main.tex}

\newpage
% Heatmap comparisons
\input{Comparisons/Example_model/main.tex}


\newpage
\input{Miscellaneous/Useful_tikz/main.tex}


% Formulas used
\newpage
\input{Miscellaneous/Formulas/main.tex}

\newpage
\printbibliography[title={References}]

\end{document}


% Markov Chains
\documentclass{article}

\usepackage{amsmath} % For writing mathematics (align, split environments etc.)
\usepackage{mathtools}
% \usepackage{amsthm} % For the proof environment
\usepackage{amsfonts} 
\usepackage{geometry}
\usepackage{float}
\usepackage{graphicx}
\usepackage{soul}
\usepackage{indentfirst}
\usepackage{multicol}
\usepackage{tikz}
\usepackage{cancel}

\usetikzlibrary{calc, automata, chains, arrows.meta, math}
\setcounter{MaxMatrixCols}{20}

\usepackage{biblatex}
\addbibresource{bibliography.bib}


\title{A game theoretic model of the behavioural gaming that takes place at the EMS - ED interface}
\author{}
\date{}

\begin{document}
\maketitle

\input{Abstract/main.tex}
\newpage
\tableofcontents
\newpage

% Introduction of the project
\input{Introduction/main.tex}

% Game Theoretic Component
\input{Game_theory_component/main.tex}


\newpage
% Quick representation of the steps of methodology
\input{Methodology/Quick/main.tex}

\newpage
% Proper methodology
\input{Methodology/Proper/main.tex}

% Markov Chains
\input{MarkovChain/markov_chain_model/main.tex}
\newpage
\input{MarkovChain/closed_form_state_probs/main.tex}
\newpage
\input{MarkovChain/expressions_from_pi/main.tex}

\newpage
% Heatmap comparisons
\input{Comparisons/Example_model/main.tex}


\newpage
\input{Miscellaneous/Useful_tikz/main.tex}


% Formulas used
\newpage
\input{Miscellaneous/Formulas/main.tex}

\newpage
\printbibliography[title={References}]

\end{document}

\newpage
\documentclass{article}

\usepackage{amsmath} % For writing mathematics (align, split environments etc.)
\usepackage{mathtools}
% \usepackage{amsthm} % For the proof environment
\usepackage{amsfonts} 
\usepackage{geometry}
\usepackage{float}
\usepackage{graphicx}
\usepackage{soul}
\usepackage{indentfirst}
\usepackage{multicol}
\usepackage{tikz}
\usepackage{cancel}

\usetikzlibrary{calc, automata, chains, arrows.meta, math}
\setcounter{MaxMatrixCols}{20}

\usepackage{biblatex}
\addbibresource{bibliography.bib}


\title{A game theoretic model of the behavioural gaming that takes place at the EMS - ED interface}
\author{}
\date{}

\begin{document}
\maketitle

\input{Abstract/main.tex}
\newpage
\tableofcontents
\newpage

% Introduction of the project
\input{Introduction/main.tex}

% Game Theoretic Component
\input{Game_theory_component/main.tex}


\newpage
% Quick representation of the steps of methodology
\input{Methodology/Quick/main.tex}

\newpage
% Proper methodology
\input{Methodology/Proper/main.tex}

% Markov Chains
\input{MarkovChain/markov_chain_model/main.tex}
\newpage
\input{MarkovChain/closed_form_state_probs/main.tex}
\newpage
\input{MarkovChain/expressions_from_pi/main.tex}

\newpage
% Heatmap comparisons
\input{Comparisons/Example_model/main.tex}


\newpage
\input{Miscellaneous/Useful_tikz/main.tex}


% Formulas used
\newpage
\input{Miscellaneous/Formulas/main.tex}

\newpage
\printbibliography[title={References}]

\end{document}

\newpage
\documentclass{article}

\usepackage{amsmath} % For writing mathematics (align, split environments etc.)
\usepackage{mathtools}
% \usepackage{amsthm} % For the proof environment
\usepackage{amsfonts} 
\usepackage{geometry}
\usepackage{float}
\usepackage{graphicx}
\usepackage{soul}
\usepackage{indentfirst}
\usepackage{multicol}
\usepackage{tikz}
\usepackage{cancel}

\usetikzlibrary{calc, automata, chains, arrows.meta, math}
\setcounter{MaxMatrixCols}{20}

\usepackage{biblatex}
\addbibresource{bibliography.bib}


\title{A game theoretic model of the behavioural gaming that takes place at the EMS - ED interface}
\author{}
\date{}

\begin{document}
\maketitle

\input{Abstract/main.tex}
\newpage
\tableofcontents
\newpage

% Introduction of the project
\input{Introduction/main.tex}

% Game Theoretic Component
\input{Game_theory_component/main.tex}


\newpage
% Quick representation of the steps of methodology
\input{Methodology/Quick/main.tex}

\newpage
% Proper methodology
\input{Methodology/Proper/main.tex}

% Markov Chains
\input{MarkovChain/markov_chain_model/main.tex}
\newpage
\input{MarkovChain/closed_form_state_probs/main.tex}
\newpage
\input{MarkovChain/expressions_from_pi/main.tex}

\newpage
% Heatmap comparisons
\input{Comparisons/Example_model/main.tex}


\newpage
\input{Miscellaneous/Useful_tikz/main.tex}


% Formulas used
\newpage
\input{Miscellaneous/Formulas/main.tex}

\newpage
\printbibliography[title={References}]

\end{document}


\newpage
% Heatmap comparisons
\documentclass{article}

\usepackage{amsmath} % For writing mathematics (align, split environments etc.)
\usepackage{mathtools}
% \usepackage{amsthm} % For the proof environment
\usepackage{amsfonts} 
\usepackage{geometry}
\usepackage{float}
\usepackage{graphicx}
\usepackage{soul}
\usepackage{indentfirst}
\usepackage{multicol}
\usepackage{tikz}
\usepackage{cancel}

\usetikzlibrary{calc, automata, chains, arrows.meta, math}
\setcounter{MaxMatrixCols}{20}

\usepackage{biblatex}
\addbibresource{bibliography.bib}


\title{A game theoretic model of the behavioural gaming that takes place at the EMS - ED interface}
\author{}
\date{}

\begin{document}
\maketitle

\input{Abstract/main.tex}
\newpage
\tableofcontents
\newpage

% Introduction of the project
\input{Introduction/main.tex}

% Game Theoretic Component
\input{Game_theory_component/main.tex}


\newpage
% Quick representation of the steps of methodology
\input{Methodology/Quick/main.tex}

\newpage
% Proper methodology
\input{Methodology/Proper/main.tex}

% Markov Chains
\input{MarkovChain/markov_chain_model/main.tex}
\newpage
\input{MarkovChain/closed_form_state_probs/main.tex}
\newpage
\input{MarkovChain/expressions_from_pi/main.tex}

\newpage
% Heatmap comparisons
\input{Comparisons/Example_model/main.tex}


\newpage
\input{Miscellaneous/Useful_tikz/main.tex}


% Formulas used
\newpage
\input{Miscellaneous/Formulas/main.tex}

\newpage
\printbibliography[title={References}]

\end{document}



\newpage
\documentclass{article}

\usepackage{amsmath} % For writing mathematics (align, split environments etc.)
\usepackage{mathtools}
% \usepackage{amsthm} % For the proof environment
\usepackage{amsfonts} 
\usepackage{geometry}
\usepackage{float}
\usepackage{graphicx}
\usepackage{soul}
\usepackage{indentfirst}
\usepackage{multicol}
\usepackage{tikz}
\usepackage{cancel}

\usetikzlibrary{calc, automata, chains, arrows.meta, math}
\setcounter{MaxMatrixCols}{20}

\usepackage{biblatex}
\addbibresource{bibliography.bib}


\title{A game theoretic model of the behavioural gaming that takes place at the EMS - ED interface}
\author{}
\date{}

\begin{document}
\maketitle

\input{Abstract/main.tex}
\newpage
\tableofcontents
\newpage

% Introduction of the project
\input{Introduction/main.tex}

% Game Theoretic Component
\input{Game_theory_component/main.tex}


\newpage
% Quick representation of the steps of methodology
\input{Methodology/Quick/main.tex}

\newpage
% Proper methodology
\input{Methodology/Proper/main.tex}

% Markov Chains
\input{MarkovChain/markov_chain_model/main.tex}
\newpage
\input{MarkovChain/closed_form_state_probs/main.tex}
\newpage
\input{MarkovChain/expressions_from_pi/main.tex}

\newpage
% Heatmap comparisons
\input{Comparisons/Example_model/main.tex}


\newpage
\input{Miscellaneous/Useful_tikz/main.tex}


% Formulas used
\newpage
\input{Miscellaneous/Formulas/main.tex}

\newpage
\printbibliography[title={References}]

\end{document}



% Formulas used
\newpage
\documentclass{article}

\usepackage{amsmath} % For writing mathematics (align, split environments etc.)
\usepackage{mathtools}
% \usepackage{amsthm} % For the proof environment
\usepackage{amsfonts} 
\usepackage{geometry}
\usepackage{float}
\usepackage{graphicx}
\usepackage{soul}
\usepackage{indentfirst}
\usepackage{multicol}
\usepackage{tikz}
\usepackage{cancel}

\usetikzlibrary{calc, automata, chains, arrows.meta, math}
\setcounter{MaxMatrixCols}{20}

\usepackage{biblatex}
\addbibresource{bibliography.bib}


\title{A game theoretic model of the behavioural gaming that takes place at the EMS - ED interface}
\author{}
\date{}

\begin{document}
\maketitle

\input{Abstract/main.tex}
\newpage
\tableofcontents
\newpage

% Introduction of the project
\input{Introduction/main.tex}

% Game Theoretic Component
\input{Game_theory_component/main.tex}


\newpage
% Quick representation of the steps of methodology
\input{Methodology/Quick/main.tex}

\newpage
% Proper methodology
\input{Methodology/Proper/main.tex}

% Markov Chains
\input{MarkovChain/markov_chain_model/main.tex}
\newpage
\input{MarkovChain/closed_form_state_probs/main.tex}
\newpage
\input{MarkovChain/expressions_from_pi/main.tex}

\newpage
% Heatmap comparisons
\input{Comparisons/Example_model/main.tex}


\newpage
\input{Miscellaneous/Useful_tikz/main.tex}


% Formulas used
\newpage
\input{Miscellaneous/Formulas/main.tex}

\newpage
\printbibliography[title={References}]

\end{document}


\newpage
\printbibliography[title={References}]

\end{document}

\newpage
\documentclass{article}

\usepackage{amsmath} % For writing mathematics (align, split environments etc.)
\usepackage{mathtools}
% \usepackage{amsthm} % For the proof environment
\usepackage{amsfonts} 
\usepackage{geometry}
\usepackage{float}
\usepackage{graphicx}
\usepackage{soul}
\usepackage{indentfirst}
\usepackage{multicol}
\usepackage{tikz}
\usepackage{cancel}

\usetikzlibrary{calc, automata, chains, arrows.meta, math}
\setcounter{MaxMatrixCols}{20}

\usepackage{biblatex}
\addbibresource{bibliography.bib}


\title{A game theoretic model of the behavioural gaming that takes place at the EMS - ED interface}
\author{}
\date{}

\begin{document}
\maketitle

\documentclass{article}

\usepackage{amsmath} % For writing mathematics (align, split environments etc.)
\usepackage{mathtools}
% \usepackage{amsthm} % For the proof environment
\usepackage{amsfonts} 
\usepackage{geometry}
\usepackage{float}
\usepackage{graphicx}
\usepackage{soul}
\usepackage{indentfirst}
\usepackage{multicol}
\usepackage{tikz}
\usepackage{cancel}

\usetikzlibrary{calc, automata, chains, arrows.meta, math}
\setcounter{MaxMatrixCols}{20}

\usepackage{biblatex}
\addbibresource{bibliography.bib}


\title{A game theoretic model of the behavioural gaming that takes place at the EMS - ED interface}
\author{}
\date{}

\begin{document}
\maketitle

\input{Abstract/main.tex}
\newpage
\tableofcontents
\newpage

% Introduction of the project
\input{Introduction/main.tex}

% Game Theoretic Component
\input{Game_theory_component/main.tex}


\newpage
% Quick representation of the steps of methodology
\input{Methodology/Quick/main.tex}

\newpage
% Proper methodology
\input{Methodology/Proper/main.tex}

% Markov Chains
\input{MarkovChain/markov_chain_model/main.tex}
\newpage
\input{MarkovChain/closed_form_state_probs/main.tex}
\newpage
\input{MarkovChain/expressions_from_pi/main.tex}

\newpage
% Heatmap comparisons
\input{Comparisons/Example_model/main.tex}


\newpage
\input{Miscellaneous/Useful_tikz/main.tex}


% Formulas used
\newpage
\input{Miscellaneous/Formulas/main.tex}

\newpage
\printbibliography[title={References}]

\end{document}

\newpage
\tableofcontents
\newpage

% Introduction of the project
\documentclass{article}

\usepackage{amsmath} % For writing mathematics (align, split environments etc.)
\usepackage{mathtools}
% \usepackage{amsthm} % For the proof environment
\usepackage{amsfonts} 
\usepackage{geometry}
\usepackage{float}
\usepackage{graphicx}
\usepackage{soul}
\usepackage{indentfirst}
\usepackage{multicol}
\usepackage{tikz}
\usepackage{cancel}

\usetikzlibrary{calc, automata, chains, arrows.meta, math}
\setcounter{MaxMatrixCols}{20}

\usepackage{biblatex}
\addbibresource{bibliography.bib}


\title{A game theoretic model of the behavioural gaming that takes place at the EMS - ED interface}
\author{}
\date{}

\begin{document}
\maketitle

\input{Abstract/main.tex}
\newpage
\tableofcontents
\newpage

% Introduction of the project
\input{Introduction/main.tex}

% Game Theoretic Component
\input{Game_theory_component/main.tex}


\newpage
% Quick representation of the steps of methodology
\input{Methodology/Quick/main.tex}

\newpage
% Proper methodology
\input{Methodology/Proper/main.tex}

% Markov Chains
\input{MarkovChain/markov_chain_model/main.tex}
\newpage
\input{MarkovChain/closed_form_state_probs/main.tex}
\newpage
\input{MarkovChain/expressions_from_pi/main.tex}

\newpage
% Heatmap comparisons
\input{Comparisons/Example_model/main.tex}


\newpage
\input{Miscellaneous/Useful_tikz/main.tex}


% Formulas used
\newpage
\input{Miscellaneous/Formulas/main.tex}

\newpage
\printbibliography[title={References}]

\end{document}


% Game Theoretic Component
\documentclass{article}

\usepackage{amsmath} % For writing mathematics (align, split environments etc.)
\usepackage{mathtools}
% \usepackage{amsthm} % For the proof environment
\usepackage{amsfonts} 
\usepackage{geometry}
\usepackage{float}
\usepackage{graphicx}
\usepackage{soul}
\usepackage{indentfirst}
\usepackage{multicol}
\usepackage{tikz}
\usepackage{cancel}

\usetikzlibrary{calc, automata, chains, arrows.meta, math}
\setcounter{MaxMatrixCols}{20}

\usepackage{biblatex}
\addbibresource{bibliography.bib}


\title{A game theoretic model of the behavioural gaming that takes place at the EMS - ED interface}
\author{}
\date{}

\begin{document}
\maketitle

\input{Abstract/main.tex}
\newpage
\tableofcontents
\newpage

% Introduction of the project
\input{Introduction/main.tex}

% Game Theoretic Component
\input{Game_theory_component/main.tex}


\newpage
% Quick representation of the steps of methodology
\input{Methodology/Quick/main.tex}

\newpage
% Proper methodology
\input{Methodology/Proper/main.tex}

% Markov Chains
\input{MarkovChain/markov_chain_model/main.tex}
\newpage
\input{MarkovChain/closed_form_state_probs/main.tex}
\newpage
\input{MarkovChain/expressions_from_pi/main.tex}

\newpage
% Heatmap comparisons
\input{Comparisons/Example_model/main.tex}


\newpage
\input{Miscellaneous/Useful_tikz/main.tex}


% Formulas used
\newpage
\input{Miscellaneous/Formulas/main.tex}

\newpage
\printbibliography[title={References}]

\end{document}



\newpage
% Quick representation of the steps of methodology
\documentclass{article}

\usepackage{amsmath} % For writing mathematics (align, split environments etc.)
\usepackage{mathtools}
% \usepackage{amsthm} % For the proof environment
\usepackage{amsfonts} 
\usepackage{geometry}
\usepackage{float}
\usepackage{graphicx}
\usepackage{soul}
\usepackage{indentfirst}
\usepackage{multicol}
\usepackage{tikz}
\usepackage{cancel}

\usetikzlibrary{calc, automata, chains, arrows.meta, math}
\setcounter{MaxMatrixCols}{20}

\usepackage{biblatex}
\addbibresource{bibliography.bib}


\title{A game theoretic model of the behavioural gaming that takes place at the EMS - ED interface}
\author{}
\date{}

\begin{document}
\maketitle

\input{Abstract/main.tex}
\newpage
\tableofcontents
\newpage

% Introduction of the project
\input{Introduction/main.tex}

% Game Theoretic Component
\input{Game_theory_component/main.tex}


\newpage
% Quick representation of the steps of methodology
\input{Methodology/Quick/main.tex}

\newpage
% Proper methodology
\input{Methodology/Proper/main.tex}

% Markov Chains
\input{MarkovChain/markov_chain_model/main.tex}
\newpage
\input{MarkovChain/closed_form_state_probs/main.tex}
\newpage
\input{MarkovChain/expressions_from_pi/main.tex}

\newpage
% Heatmap comparisons
\input{Comparisons/Example_model/main.tex}


\newpage
\input{Miscellaneous/Useful_tikz/main.tex}


% Formulas used
\newpage
\input{Miscellaneous/Formulas/main.tex}

\newpage
\printbibliography[title={References}]

\end{document}


\newpage
% Proper methodology
\documentclass{article}

\usepackage{amsmath} % For writing mathematics (align, split environments etc.)
\usepackage{mathtools}
% \usepackage{amsthm} % For the proof environment
\usepackage{amsfonts} 
\usepackage{geometry}
\usepackage{float}
\usepackage{graphicx}
\usepackage{soul}
\usepackage{indentfirst}
\usepackage{multicol}
\usepackage{tikz}
\usepackage{cancel}

\usetikzlibrary{calc, automata, chains, arrows.meta, math}
\setcounter{MaxMatrixCols}{20}

\usepackage{biblatex}
\addbibresource{bibliography.bib}


\title{A game theoretic model of the behavioural gaming that takes place at the EMS - ED interface}
\author{}
\date{}

\begin{document}
\maketitle

\input{Abstract/main.tex}
\newpage
\tableofcontents
\newpage

% Introduction of the project
\input{Introduction/main.tex}

% Game Theoretic Component
\input{Game_theory_component/main.tex}


\newpage
% Quick representation of the steps of methodology
\input{Methodology/Quick/main.tex}

\newpage
% Proper methodology
\input{Methodology/Proper/main.tex}

% Markov Chains
\input{MarkovChain/markov_chain_model/main.tex}
\newpage
\input{MarkovChain/closed_form_state_probs/main.tex}
\newpage
\input{MarkovChain/expressions_from_pi/main.tex}

\newpage
% Heatmap comparisons
\input{Comparisons/Example_model/main.tex}


\newpage
\input{Miscellaneous/Useful_tikz/main.tex}


% Formulas used
\newpage
\input{Miscellaneous/Formulas/main.tex}

\newpage
\printbibliography[title={References}]

\end{document}


% Markov Chains
\documentclass{article}

\usepackage{amsmath} % For writing mathematics (align, split environments etc.)
\usepackage{mathtools}
% \usepackage{amsthm} % For the proof environment
\usepackage{amsfonts} 
\usepackage{geometry}
\usepackage{float}
\usepackage{graphicx}
\usepackage{soul}
\usepackage{indentfirst}
\usepackage{multicol}
\usepackage{tikz}
\usepackage{cancel}

\usetikzlibrary{calc, automata, chains, arrows.meta, math}
\setcounter{MaxMatrixCols}{20}

\usepackage{biblatex}
\addbibresource{bibliography.bib}


\title{A game theoretic model of the behavioural gaming that takes place at the EMS - ED interface}
\author{}
\date{}

\begin{document}
\maketitle

\input{Abstract/main.tex}
\newpage
\tableofcontents
\newpage

% Introduction of the project
\input{Introduction/main.tex}

% Game Theoretic Component
\input{Game_theory_component/main.tex}


\newpage
% Quick representation of the steps of methodology
\input{Methodology/Quick/main.tex}

\newpage
% Proper methodology
\input{Methodology/Proper/main.tex}

% Markov Chains
\input{MarkovChain/markov_chain_model/main.tex}
\newpage
\input{MarkovChain/closed_form_state_probs/main.tex}
\newpage
\input{MarkovChain/expressions_from_pi/main.tex}

\newpage
% Heatmap comparisons
\input{Comparisons/Example_model/main.tex}


\newpage
\input{Miscellaneous/Useful_tikz/main.tex}


% Formulas used
\newpage
\input{Miscellaneous/Formulas/main.tex}

\newpage
\printbibliography[title={References}]

\end{document}

\newpage
\documentclass{article}

\usepackage{amsmath} % For writing mathematics (align, split environments etc.)
\usepackage{mathtools}
% \usepackage{amsthm} % For the proof environment
\usepackage{amsfonts} 
\usepackage{geometry}
\usepackage{float}
\usepackage{graphicx}
\usepackage{soul}
\usepackage{indentfirst}
\usepackage{multicol}
\usepackage{tikz}
\usepackage{cancel}

\usetikzlibrary{calc, automata, chains, arrows.meta, math}
\setcounter{MaxMatrixCols}{20}

\usepackage{biblatex}
\addbibresource{bibliography.bib}


\title{A game theoretic model of the behavioural gaming that takes place at the EMS - ED interface}
\author{}
\date{}

\begin{document}
\maketitle

\input{Abstract/main.tex}
\newpage
\tableofcontents
\newpage

% Introduction of the project
\input{Introduction/main.tex}

% Game Theoretic Component
\input{Game_theory_component/main.tex}


\newpage
% Quick representation of the steps of methodology
\input{Methodology/Quick/main.tex}

\newpage
% Proper methodology
\input{Methodology/Proper/main.tex}

% Markov Chains
\input{MarkovChain/markov_chain_model/main.tex}
\newpage
\input{MarkovChain/closed_form_state_probs/main.tex}
\newpage
\input{MarkovChain/expressions_from_pi/main.tex}

\newpage
% Heatmap comparisons
\input{Comparisons/Example_model/main.tex}


\newpage
\input{Miscellaneous/Useful_tikz/main.tex}


% Formulas used
\newpage
\input{Miscellaneous/Formulas/main.tex}

\newpage
\printbibliography[title={References}]

\end{document}

\newpage
\documentclass{article}

\usepackage{amsmath} % For writing mathematics (align, split environments etc.)
\usepackage{mathtools}
% \usepackage{amsthm} % For the proof environment
\usepackage{amsfonts} 
\usepackage{geometry}
\usepackage{float}
\usepackage{graphicx}
\usepackage{soul}
\usepackage{indentfirst}
\usepackage{multicol}
\usepackage{tikz}
\usepackage{cancel}

\usetikzlibrary{calc, automata, chains, arrows.meta, math}
\setcounter{MaxMatrixCols}{20}

\usepackage{biblatex}
\addbibresource{bibliography.bib}


\title{A game theoretic model of the behavioural gaming that takes place at the EMS - ED interface}
\author{}
\date{}

\begin{document}
\maketitle

\input{Abstract/main.tex}
\newpage
\tableofcontents
\newpage

% Introduction of the project
\input{Introduction/main.tex}

% Game Theoretic Component
\input{Game_theory_component/main.tex}


\newpage
% Quick representation of the steps of methodology
\input{Methodology/Quick/main.tex}

\newpage
% Proper methodology
\input{Methodology/Proper/main.tex}

% Markov Chains
\input{MarkovChain/markov_chain_model/main.tex}
\newpage
\input{MarkovChain/closed_form_state_probs/main.tex}
\newpage
\input{MarkovChain/expressions_from_pi/main.tex}

\newpage
% Heatmap comparisons
\input{Comparisons/Example_model/main.tex}


\newpage
\input{Miscellaneous/Useful_tikz/main.tex}


% Formulas used
\newpage
\input{Miscellaneous/Formulas/main.tex}

\newpage
\printbibliography[title={References}]

\end{document}


\newpage
% Heatmap comparisons
\documentclass{article}

\usepackage{amsmath} % For writing mathematics (align, split environments etc.)
\usepackage{mathtools}
% \usepackage{amsthm} % For the proof environment
\usepackage{amsfonts} 
\usepackage{geometry}
\usepackage{float}
\usepackage{graphicx}
\usepackage{soul}
\usepackage{indentfirst}
\usepackage{multicol}
\usepackage{tikz}
\usepackage{cancel}

\usetikzlibrary{calc, automata, chains, arrows.meta, math}
\setcounter{MaxMatrixCols}{20}

\usepackage{biblatex}
\addbibresource{bibliography.bib}


\title{A game theoretic model of the behavioural gaming that takes place at the EMS - ED interface}
\author{}
\date{}

\begin{document}
\maketitle

\input{Abstract/main.tex}
\newpage
\tableofcontents
\newpage

% Introduction of the project
\input{Introduction/main.tex}

% Game Theoretic Component
\input{Game_theory_component/main.tex}


\newpage
% Quick representation of the steps of methodology
\input{Methodology/Quick/main.tex}

\newpage
% Proper methodology
\input{Methodology/Proper/main.tex}

% Markov Chains
\input{MarkovChain/markov_chain_model/main.tex}
\newpage
\input{MarkovChain/closed_form_state_probs/main.tex}
\newpage
\input{MarkovChain/expressions_from_pi/main.tex}

\newpage
% Heatmap comparisons
\input{Comparisons/Example_model/main.tex}


\newpage
\input{Miscellaneous/Useful_tikz/main.tex}


% Formulas used
\newpage
\input{Miscellaneous/Formulas/main.tex}

\newpage
\printbibliography[title={References}]

\end{document}



\newpage
\documentclass{article}

\usepackage{amsmath} % For writing mathematics (align, split environments etc.)
\usepackage{mathtools}
% \usepackage{amsthm} % For the proof environment
\usepackage{amsfonts} 
\usepackage{geometry}
\usepackage{float}
\usepackage{graphicx}
\usepackage{soul}
\usepackage{indentfirst}
\usepackage{multicol}
\usepackage{tikz}
\usepackage{cancel}

\usetikzlibrary{calc, automata, chains, arrows.meta, math}
\setcounter{MaxMatrixCols}{20}

\usepackage{biblatex}
\addbibresource{bibliography.bib}


\title{A game theoretic model of the behavioural gaming that takes place at the EMS - ED interface}
\author{}
\date{}

\begin{document}
\maketitle

\input{Abstract/main.tex}
\newpage
\tableofcontents
\newpage

% Introduction of the project
\input{Introduction/main.tex}

% Game Theoretic Component
\input{Game_theory_component/main.tex}


\newpage
% Quick representation of the steps of methodology
\input{Methodology/Quick/main.tex}

\newpage
% Proper methodology
\input{Methodology/Proper/main.tex}

% Markov Chains
\input{MarkovChain/markov_chain_model/main.tex}
\newpage
\input{MarkovChain/closed_form_state_probs/main.tex}
\newpage
\input{MarkovChain/expressions_from_pi/main.tex}

\newpage
% Heatmap comparisons
\input{Comparisons/Example_model/main.tex}


\newpage
\input{Miscellaneous/Useful_tikz/main.tex}


% Formulas used
\newpage
\input{Miscellaneous/Formulas/main.tex}

\newpage
\printbibliography[title={References}]

\end{document}



% Formulas used
\newpage
\documentclass{article}

\usepackage{amsmath} % For writing mathematics (align, split environments etc.)
\usepackage{mathtools}
% \usepackage{amsthm} % For the proof environment
\usepackage{amsfonts} 
\usepackage{geometry}
\usepackage{float}
\usepackage{graphicx}
\usepackage{soul}
\usepackage{indentfirst}
\usepackage{multicol}
\usepackage{tikz}
\usepackage{cancel}

\usetikzlibrary{calc, automata, chains, arrows.meta, math}
\setcounter{MaxMatrixCols}{20}

\usepackage{biblatex}
\addbibresource{bibliography.bib}


\title{A game theoretic model of the behavioural gaming that takes place at the EMS - ED interface}
\author{}
\date{}

\begin{document}
\maketitle

\input{Abstract/main.tex}
\newpage
\tableofcontents
\newpage

% Introduction of the project
\input{Introduction/main.tex}

% Game Theoretic Component
\input{Game_theory_component/main.tex}


\newpage
% Quick representation of the steps of methodology
\input{Methodology/Quick/main.tex}

\newpage
% Proper methodology
\input{Methodology/Proper/main.tex}

% Markov Chains
\input{MarkovChain/markov_chain_model/main.tex}
\newpage
\input{MarkovChain/closed_form_state_probs/main.tex}
\newpage
\input{MarkovChain/expressions_from_pi/main.tex}

\newpage
% Heatmap comparisons
\input{Comparisons/Example_model/main.tex}


\newpage
\input{Miscellaneous/Useful_tikz/main.tex}


% Formulas used
\newpage
\input{Miscellaneous/Formulas/main.tex}

\newpage
\printbibliography[title={References}]

\end{document}


\newpage
\printbibliography[title={References}]

\end{document}

\newpage
\documentclass{article}

\usepackage{amsmath} % For writing mathematics (align, split environments etc.)
\usepackage{mathtools}
% \usepackage{amsthm} % For the proof environment
\usepackage{amsfonts} 
\usepackage{geometry}
\usepackage{float}
\usepackage{graphicx}
\usepackage{soul}
\usepackage{indentfirst}
\usepackage{multicol}
\usepackage{tikz}
\usepackage{cancel}

\usetikzlibrary{calc, automata, chains, arrows.meta, math}
\setcounter{MaxMatrixCols}{20}

\usepackage{biblatex}
\addbibresource{bibliography.bib}


\title{A game theoretic model of the behavioural gaming that takes place at the EMS - ED interface}
\author{}
\date{}

\begin{document}
\maketitle

\documentclass{article}

\usepackage{amsmath} % For writing mathematics (align, split environments etc.)
\usepackage{mathtools}
% \usepackage{amsthm} % For the proof environment
\usepackage{amsfonts} 
\usepackage{geometry}
\usepackage{float}
\usepackage{graphicx}
\usepackage{soul}
\usepackage{indentfirst}
\usepackage{multicol}
\usepackage{tikz}
\usepackage{cancel}

\usetikzlibrary{calc, automata, chains, arrows.meta, math}
\setcounter{MaxMatrixCols}{20}

\usepackage{biblatex}
\addbibresource{bibliography.bib}


\title{A game theoretic model of the behavioural gaming that takes place at the EMS - ED interface}
\author{}
\date{}

\begin{document}
\maketitle

\input{Abstract/main.tex}
\newpage
\tableofcontents
\newpage

% Introduction of the project
\input{Introduction/main.tex}

% Game Theoretic Component
\input{Game_theory_component/main.tex}


\newpage
% Quick representation of the steps of methodology
\input{Methodology/Quick/main.tex}

\newpage
% Proper methodology
\input{Methodology/Proper/main.tex}

% Markov Chains
\input{MarkovChain/markov_chain_model/main.tex}
\newpage
\input{MarkovChain/closed_form_state_probs/main.tex}
\newpage
\input{MarkovChain/expressions_from_pi/main.tex}

\newpage
% Heatmap comparisons
\input{Comparisons/Example_model/main.tex}


\newpage
\input{Miscellaneous/Useful_tikz/main.tex}


% Formulas used
\newpage
\input{Miscellaneous/Formulas/main.tex}

\newpage
\printbibliography[title={References}]

\end{document}

\newpage
\tableofcontents
\newpage

% Introduction of the project
\documentclass{article}

\usepackage{amsmath} % For writing mathematics (align, split environments etc.)
\usepackage{mathtools}
% \usepackage{amsthm} % For the proof environment
\usepackage{amsfonts} 
\usepackage{geometry}
\usepackage{float}
\usepackage{graphicx}
\usepackage{soul}
\usepackage{indentfirst}
\usepackage{multicol}
\usepackage{tikz}
\usepackage{cancel}

\usetikzlibrary{calc, automata, chains, arrows.meta, math}
\setcounter{MaxMatrixCols}{20}

\usepackage{biblatex}
\addbibresource{bibliography.bib}


\title{A game theoretic model of the behavioural gaming that takes place at the EMS - ED interface}
\author{}
\date{}

\begin{document}
\maketitle

\input{Abstract/main.tex}
\newpage
\tableofcontents
\newpage

% Introduction of the project
\input{Introduction/main.tex}

% Game Theoretic Component
\input{Game_theory_component/main.tex}


\newpage
% Quick representation of the steps of methodology
\input{Methodology/Quick/main.tex}

\newpage
% Proper methodology
\input{Methodology/Proper/main.tex}

% Markov Chains
\input{MarkovChain/markov_chain_model/main.tex}
\newpage
\input{MarkovChain/closed_form_state_probs/main.tex}
\newpage
\input{MarkovChain/expressions_from_pi/main.tex}

\newpage
% Heatmap comparisons
\input{Comparisons/Example_model/main.tex}


\newpage
\input{Miscellaneous/Useful_tikz/main.tex}


% Formulas used
\newpage
\input{Miscellaneous/Formulas/main.tex}

\newpage
\printbibliography[title={References}]

\end{document}


% Game Theoretic Component
\documentclass{article}

\usepackage{amsmath} % For writing mathematics (align, split environments etc.)
\usepackage{mathtools}
% \usepackage{amsthm} % For the proof environment
\usepackage{amsfonts} 
\usepackage{geometry}
\usepackage{float}
\usepackage{graphicx}
\usepackage{soul}
\usepackage{indentfirst}
\usepackage{multicol}
\usepackage{tikz}
\usepackage{cancel}

\usetikzlibrary{calc, automata, chains, arrows.meta, math}
\setcounter{MaxMatrixCols}{20}

\usepackage{biblatex}
\addbibresource{bibliography.bib}


\title{A game theoretic model of the behavioural gaming that takes place at the EMS - ED interface}
\author{}
\date{}

\begin{document}
\maketitle

\input{Abstract/main.tex}
\newpage
\tableofcontents
\newpage

% Introduction of the project
\input{Introduction/main.tex}

% Game Theoretic Component
\input{Game_theory_component/main.tex}


\newpage
% Quick representation of the steps of methodology
\input{Methodology/Quick/main.tex}

\newpage
% Proper methodology
\input{Methodology/Proper/main.tex}

% Markov Chains
\input{MarkovChain/markov_chain_model/main.tex}
\newpage
\input{MarkovChain/closed_form_state_probs/main.tex}
\newpage
\input{MarkovChain/expressions_from_pi/main.tex}

\newpage
% Heatmap comparisons
\input{Comparisons/Example_model/main.tex}


\newpage
\input{Miscellaneous/Useful_tikz/main.tex}


% Formulas used
\newpage
\input{Miscellaneous/Formulas/main.tex}

\newpage
\printbibliography[title={References}]

\end{document}



\newpage
% Quick representation of the steps of methodology
\documentclass{article}

\usepackage{amsmath} % For writing mathematics (align, split environments etc.)
\usepackage{mathtools}
% \usepackage{amsthm} % For the proof environment
\usepackage{amsfonts} 
\usepackage{geometry}
\usepackage{float}
\usepackage{graphicx}
\usepackage{soul}
\usepackage{indentfirst}
\usepackage{multicol}
\usepackage{tikz}
\usepackage{cancel}

\usetikzlibrary{calc, automata, chains, arrows.meta, math}
\setcounter{MaxMatrixCols}{20}

\usepackage{biblatex}
\addbibresource{bibliography.bib}


\title{A game theoretic model of the behavioural gaming that takes place at the EMS - ED interface}
\author{}
\date{}

\begin{document}
\maketitle

\input{Abstract/main.tex}
\newpage
\tableofcontents
\newpage

% Introduction of the project
\input{Introduction/main.tex}

% Game Theoretic Component
\input{Game_theory_component/main.tex}


\newpage
% Quick representation of the steps of methodology
\input{Methodology/Quick/main.tex}

\newpage
% Proper methodology
\input{Methodology/Proper/main.tex}

% Markov Chains
\input{MarkovChain/markov_chain_model/main.tex}
\newpage
\input{MarkovChain/closed_form_state_probs/main.tex}
\newpage
\input{MarkovChain/expressions_from_pi/main.tex}

\newpage
% Heatmap comparisons
\input{Comparisons/Example_model/main.tex}


\newpage
\input{Miscellaneous/Useful_tikz/main.tex}


% Formulas used
\newpage
\input{Miscellaneous/Formulas/main.tex}

\newpage
\printbibliography[title={References}]

\end{document}


\newpage
% Proper methodology
\documentclass{article}

\usepackage{amsmath} % For writing mathematics (align, split environments etc.)
\usepackage{mathtools}
% \usepackage{amsthm} % For the proof environment
\usepackage{amsfonts} 
\usepackage{geometry}
\usepackage{float}
\usepackage{graphicx}
\usepackage{soul}
\usepackage{indentfirst}
\usepackage{multicol}
\usepackage{tikz}
\usepackage{cancel}

\usetikzlibrary{calc, automata, chains, arrows.meta, math}
\setcounter{MaxMatrixCols}{20}

\usepackage{biblatex}
\addbibresource{bibliography.bib}


\title{A game theoretic model of the behavioural gaming that takes place at the EMS - ED interface}
\author{}
\date{}

\begin{document}
\maketitle

\input{Abstract/main.tex}
\newpage
\tableofcontents
\newpage

% Introduction of the project
\input{Introduction/main.tex}

% Game Theoretic Component
\input{Game_theory_component/main.tex}


\newpage
% Quick representation of the steps of methodology
\input{Methodology/Quick/main.tex}

\newpage
% Proper methodology
\input{Methodology/Proper/main.tex}

% Markov Chains
\input{MarkovChain/markov_chain_model/main.tex}
\newpage
\input{MarkovChain/closed_form_state_probs/main.tex}
\newpage
\input{MarkovChain/expressions_from_pi/main.tex}

\newpage
% Heatmap comparisons
\input{Comparisons/Example_model/main.tex}


\newpage
\input{Miscellaneous/Useful_tikz/main.tex}


% Formulas used
\newpage
\input{Miscellaneous/Formulas/main.tex}

\newpage
\printbibliography[title={References}]

\end{document}


% Markov Chains
\documentclass{article}

\usepackage{amsmath} % For writing mathematics (align, split environments etc.)
\usepackage{mathtools}
% \usepackage{amsthm} % For the proof environment
\usepackage{amsfonts} 
\usepackage{geometry}
\usepackage{float}
\usepackage{graphicx}
\usepackage{soul}
\usepackage{indentfirst}
\usepackage{multicol}
\usepackage{tikz}
\usepackage{cancel}

\usetikzlibrary{calc, automata, chains, arrows.meta, math}
\setcounter{MaxMatrixCols}{20}

\usepackage{biblatex}
\addbibresource{bibliography.bib}


\title{A game theoretic model of the behavioural gaming that takes place at the EMS - ED interface}
\author{}
\date{}

\begin{document}
\maketitle

\input{Abstract/main.tex}
\newpage
\tableofcontents
\newpage

% Introduction of the project
\input{Introduction/main.tex}

% Game Theoretic Component
\input{Game_theory_component/main.tex}


\newpage
% Quick representation of the steps of methodology
\input{Methodology/Quick/main.tex}

\newpage
% Proper methodology
\input{Methodology/Proper/main.tex}

% Markov Chains
\input{MarkovChain/markov_chain_model/main.tex}
\newpage
\input{MarkovChain/closed_form_state_probs/main.tex}
\newpage
\input{MarkovChain/expressions_from_pi/main.tex}

\newpage
% Heatmap comparisons
\input{Comparisons/Example_model/main.tex}


\newpage
\input{Miscellaneous/Useful_tikz/main.tex}


% Formulas used
\newpage
\input{Miscellaneous/Formulas/main.tex}

\newpage
\printbibliography[title={References}]

\end{document}

\newpage
\documentclass{article}

\usepackage{amsmath} % For writing mathematics (align, split environments etc.)
\usepackage{mathtools}
% \usepackage{amsthm} % For the proof environment
\usepackage{amsfonts} 
\usepackage{geometry}
\usepackage{float}
\usepackage{graphicx}
\usepackage{soul}
\usepackage{indentfirst}
\usepackage{multicol}
\usepackage{tikz}
\usepackage{cancel}

\usetikzlibrary{calc, automata, chains, arrows.meta, math}
\setcounter{MaxMatrixCols}{20}

\usepackage{biblatex}
\addbibresource{bibliography.bib}


\title{A game theoretic model of the behavioural gaming that takes place at the EMS - ED interface}
\author{}
\date{}

\begin{document}
\maketitle

\input{Abstract/main.tex}
\newpage
\tableofcontents
\newpage

% Introduction of the project
\input{Introduction/main.tex}

% Game Theoretic Component
\input{Game_theory_component/main.tex}


\newpage
% Quick representation of the steps of methodology
\input{Methodology/Quick/main.tex}

\newpage
% Proper methodology
\input{Methodology/Proper/main.tex}

% Markov Chains
\input{MarkovChain/markov_chain_model/main.tex}
\newpage
\input{MarkovChain/closed_form_state_probs/main.tex}
\newpage
\input{MarkovChain/expressions_from_pi/main.tex}

\newpage
% Heatmap comparisons
\input{Comparisons/Example_model/main.tex}


\newpage
\input{Miscellaneous/Useful_tikz/main.tex}


% Formulas used
\newpage
\input{Miscellaneous/Formulas/main.tex}

\newpage
\printbibliography[title={References}]

\end{document}

\newpage
\documentclass{article}

\usepackage{amsmath} % For writing mathematics (align, split environments etc.)
\usepackage{mathtools}
% \usepackage{amsthm} % For the proof environment
\usepackage{amsfonts} 
\usepackage{geometry}
\usepackage{float}
\usepackage{graphicx}
\usepackage{soul}
\usepackage{indentfirst}
\usepackage{multicol}
\usepackage{tikz}
\usepackage{cancel}

\usetikzlibrary{calc, automata, chains, arrows.meta, math}
\setcounter{MaxMatrixCols}{20}

\usepackage{biblatex}
\addbibresource{bibliography.bib}


\title{A game theoretic model of the behavioural gaming that takes place at the EMS - ED interface}
\author{}
\date{}

\begin{document}
\maketitle

\input{Abstract/main.tex}
\newpage
\tableofcontents
\newpage

% Introduction of the project
\input{Introduction/main.tex}

% Game Theoretic Component
\input{Game_theory_component/main.tex}


\newpage
% Quick representation of the steps of methodology
\input{Methodology/Quick/main.tex}

\newpage
% Proper methodology
\input{Methodology/Proper/main.tex}

% Markov Chains
\input{MarkovChain/markov_chain_model/main.tex}
\newpage
\input{MarkovChain/closed_form_state_probs/main.tex}
\newpage
\input{MarkovChain/expressions_from_pi/main.tex}

\newpage
% Heatmap comparisons
\input{Comparisons/Example_model/main.tex}


\newpage
\input{Miscellaneous/Useful_tikz/main.tex}


% Formulas used
\newpage
\input{Miscellaneous/Formulas/main.tex}

\newpage
\printbibliography[title={References}]

\end{document}


\newpage
% Heatmap comparisons
\documentclass{article}

\usepackage{amsmath} % For writing mathematics (align, split environments etc.)
\usepackage{mathtools}
% \usepackage{amsthm} % For the proof environment
\usepackage{amsfonts} 
\usepackage{geometry}
\usepackage{float}
\usepackage{graphicx}
\usepackage{soul}
\usepackage{indentfirst}
\usepackage{multicol}
\usepackage{tikz}
\usepackage{cancel}

\usetikzlibrary{calc, automata, chains, arrows.meta, math}
\setcounter{MaxMatrixCols}{20}

\usepackage{biblatex}
\addbibresource{bibliography.bib}


\title{A game theoretic model of the behavioural gaming that takes place at the EMS - ED interface}
\author{}
\date{}

\begin{document}
\maketitle

\input{Abstract/main.tex}
\newpage
\tableofcontents
\newpage

% Introduction of the project
\input{Introduction/main.tex}

% Game Theoretic Component
\input{Game_theory_component/main.tex}


\newpage
% Quick representation of the steps of methodology
\input{Methodology/Quick/main.tex}

\newpage
% Proper methodology
\input{Methodology/Proper/main.tex}

% Markov Chains
\input{MarkovChain/markov_chain_model/main.tex}
\newpage
\input{MarkovChain/closed_form_state_probs/main.tex}
\newpage
\input{MarkovChain/expressions_from_pi/main.tex}

\newpage
% Heatmap comparisons
\input{Comparisons/Example_model/main.tex}


\newpage
\input{Miscellaneous/Useful_tikz/main.tex}


% Formulas used
\newpage
\input{Miscellaneous/Formulas/main.tex}

\newpage
\printbibliography[title={References}]

\end{document}



\newpage
\documentclass{article}

\usepackage{amsmath} % For writing mathematics (align, split environments etc.)
\usepackage{mathtools}
% \usepackage{amsthm} % For the proof environment
\usepackage{amsfonts} 
\usepackage{geometry}
\usepackage{float}
\usepackage{graphicx}
\usepackage{soul}
\usepackage{indentfirst}
\usepackage{multicol}
\usepackage{tikz}
\usepackage{cancel}

\usetikzlibrary{calc, automata, chains, arrows.meta, math}
\setcounter{MaxMatrixCols}{20}

\usepackage{biblatex}
\addbibresource{bibliography.bib}


\title{A game theoretic model of the behavioural gaming that takes place at the EMS - ED interface}
\author{}
\date{}

\begin{document}
\maketitle

\input{Abstract/main.tex}
\newpage
\tableofcontents
\newpage

% Introduction of the project
\input{Introduction/main.tex}

% Game Theoretic Component
\input{Game_theory_component/main.tex}


\newpage
% Quick representation of the steps of methodology
\input{Methodology/Quick/main.tex}

\newpage
% Proper methodology
\input{Methodology/Proper/main.tex}

% Markov Chains
\input{MarkovChain/markov_chain_model/main.tex}
\newpage
\input{MarkovChain/closed_form_state_probs/main.tex}
\newpage
\input{MarkovChain/expressions_from_pi/main.tex}

\newpage
% Heatmap comparisons
\input{Comparisons/Example_model/main.tex}


\newpage
\input{Miscellaneous/Useful_tikz/main.tex}


% Formulas used
\newpage
\input{Miscellaneous/Formulas/main.tex}

\newpage
\printbibliography[title={References}]

\end{document}



% Formulas used
\newpage
\documentclass{article}

\usepackage{amsmath} % For writing mathematics (align, split environments etc.)
\usepackage{mathtools}
% \usepackage{amsthm} % For the proof environment
\usepackage{amsfonts} 
\usepackage{geometry}
\usepackage{float}
\usepackage{graphicx}
\usepackage{soul}
\usepackage{indentfirst}
\usepackage{multicol}
\usepackage{tikz}
\usepackage{cancel}

\usetikzlibrary{calc, automata, chains, arrows.meta, math}
\setcounter{MaxMatrixCols}{20}

\usepackage{biblatex}
\addbibresource{bibliography.bib}


\title{A game theoretic model of the behavioural gaming that takes place at the EMS - ED interface}
\author{}
\date{}

\begin{document}
\maketitle

\input{Abstract/main.tex}
\newpage
\tableofcontents
\newpage

% Introduction of the project
\input{Introduction/main.tex}

% Game Theoretic Component
\input{Game_theory_component/main.tex}


\newpage
% Quick representation of the steps of methodology
\input{Methodology/Quick/main.tex}

\newpage
% Proper methodology
\input{Methodology/Proper/main.tex}

% Markov Chains
\input{MarkovChain/markov_chain_model/main.tex}
\newpage
\input{MarkovChain/closed_form_state_probs/main.tex}
\newpage
\input{MarkovChain/expressions_from_pi/main.tex}

\newpage
% Heatmap comparisons
\input{Comparisons/Example_model/main.tex}


\newpage
\input{Miscellaneous/Useful_tikz/main.tex}


% Formulas used
\newpage
\input{Miscellaneous/Formulas/main.tex}

\newpage
\printbibliography[title={References}]

\end{document}


\newpage
\printbibliography[title={References}]

\end{document}


\newpage
% Heatmap comparisons
\documentclass{article}

\usepackage{amsmath} % For writing mathematics (align, split environments etc.)
\usepackage{mathtools}
% \usepackage{amsthm} % For the proof environment
\usepackage{amsfonts} 
\usepackage{geometry}
\usepackage{float}
\usepackage{graphicx}
\usepackage{soul}
\usepackage{indentfirst}
\usepackage{multicol}
\usepackage{tikz}
\usepackage{cancel}

\usetikzlibrary{calc, automata, chains, arrows.meta, math}
\setcounter{MaxMatrixCols}{20}

\usepackage{biblatex}
\addbibresource{bibliography.bib}


\title{A game theoretic model of the behavioural gaming that takes place at the EMS - ED interface}
\author{}
\date{}

\begin{document}
\maketitle

\documentclass{article}

\usepackage{amsmath} % For writing mathematics (align, split environments etc.)
\usepackage{mathtools}
% \usepackage{amsthm} % For the proof environment
\usepackage{amsfonts} 
\usepackage{geometry}
\usepackage{float}
\usepackage{graphicx}
\usepackage{soul}
\usepackage{indentfirst}
\usepackage{multicol}
\usepackage{tikz}
\usepackage{cancel}

\usetikzlibrary{calc, automata, chains, arrows.meta, math}
\setcounter{MaxMatrixCols}{20}

\usepackage{biblatex}
\addbibresource{bibliography.bib}


\title{A game theoretic model of the behavioural gaming that takes place at the EMS - ED interface}
\author{}
\date{}

\begin{document}
\maketitle

\input{Abstract/main.tex}
\newpage
\tableofcontents
\newpage

% Introduction of the project
\input{Introduction/main.tex}

% Game Theoretic Component
\input{Game_theory_component/main.tex}


\newpage
% Quick representation of the steps of methodology
\input{Methodology/Quick/main.tex}

\newpage
% Proper methodology
\input{Methodology/Proper/main.tex}

% Markov Chains
\input{MarkovChain/markov_chain_model/main.tex}
\newpage
\input{MarkovChain/closed_form_state_probs/main.tex}
\newpage
\input{MarkovChain/expressions_from_pi/main.tex}

\newpage
% Heatmap comparisons
\input{Comparisons/Example_model/main.tex}


\newpage
\input{Miscellaneous/Useful_tikz/main.tex}


% Formulas used
\newpage
\input{Miscellaneous/Formulas/main.tex}

\newpage
\printbibliography[title={References}]

\end{document}

\newpage
\tableofcontents
\newpage

% Introduction of the project
\documentclass{article}

\usepackage{amsmath} % For writing mathematics (align, split environments etc.)
\usepackage{mathtools}
% \usepackage{amsthm} % For the proof environment
\usepackage{amsfonts} 
\usepackage{geometry}
\usepackage{float}
\usepackage{graphicx}
\usepackage{soul}
\usepackage{indentfirst}
\usepackage{multicol}
\usepackage{tikz}
\usepackage{cancel}

\usetikzlibrary{calc, automata, chains, arrows.meta, math}
\setcounter{MaxMatrixCols}{20}

\usepackage{biblatex}
\addbibresource{bibliography.bib}


\title{A game theoretic model of the behavioural gaming that takes place at the EMS - ED interface}
\author{}
\date{}

\begin{document}
\maketitle

\input{Abstract/main.tex}
\newpage
\tableofcontents
\newpage

% Introduction of the project
\input{Introduction/main.tex}

% Game Theoretic Component
\input{Game_theory_component/main.tex}


\newpage
% Quick representation of the steps of methodology
\input{Methodology/Quick/main.tex}

\newpage
% Proper methodology
\input{Methodology/Proper/main.tex}

% Markov Chains
\input{MarkovChain/markov_chain_model/main.tex}
\newpage
\input{MarkovChain/closed_form_state_probs/main.tex}
\newpage
\input{MarkovChain/expressions_from_pi/main.tex}

\newpage
% Heatmap comparisons
\input{Comparisons/Example_model/main.tex}


\newpage
\input{Miscellaneous/Useful_tikz/main.tex}


% Formulas used
\newpage
\input{Miscellaneous/Formulas/main.tex}

\newpage
\printbibliography[title={References}]

\end{document}


% Game Theoretic Component
\documentclass{article}

\usepackage{amsmath} % For writing mathematics (align, split environments etc.)
\usepackage{mathtools}
% \usepackage{amsthm} % For the proof environment
\usepackage{amsfonts} 
\usepackage{geometry}
\usepackage{float}
\usepackage{graphicx}
\usepackage{soul}
\usepackage{indentfirst}
\usepackage{multicol}
\usepackage{tikz}
\usepackage{cancel}

\usetikzlibrary{calc, automata, chains, arrows.meta, math}
\setcounter{MaxMatrixCols}{20}

\usepackage{biblatex}
\addbibresource{bibliography.bib}


\title{A game theoretic model of the behavioural gaming that takes place at the EMS - ED interface}
\author{}
\date{}

\begin{document}
\maketitle

\input{Abstract/main.tex}
\newpage
\tableofcontents
\newpage

% Introduction of the project
\input{Introduction/main.tex}

% Game Theoretic Component
\input{Game_theory_component/main.tex}


\newpage
% Quick representation of the steps of methodology
\input{Methodology/Quick/main.tex}

\newpage
% Proper methodology
\input{Methodology/Proper/main.tex}

% Markov Chains
\input{MarkovChain/markov_chain_model/main.tex}
\newpage
\input{MarkovChain/closed_form_state_probs/main.tex}
\newpage
\input{MarkovChain/expressions_from_pi/main.tex}

\newpage
% Heatmap comparisons
\input{Comparisons/Example_model/main.tex}


\newpage
\input{Miscellaneous/Useful_tikz/main.tex}


% Formulas used
\newpage
\input{Miscellaneous/Formulas/main.tex}

\newpage
\printbibliography[title={References}]

\end{document}



\newpage
% Quick representation of the steps of methodology
\documentclass{article}

\usepackage{amsmath} % For writing mathematics (align, split environments etc.)
\usepackage{mathtools}
% \usepackage{amsthm} % For the proof environment
\usepackage{amsfonts} 
\usepackage{geometry}
\usepackage{float}
\usepackage{graphicx}
\usepackage{soul}
\usepackage{indentfirst}
\usepackage{multicol}
\usepackage{tikz}
\usepackage{cancel}

\usetikzlibrary{calc, automata, chains, arrows.meta, math}
\setcounter{MaxMatrixCols}{20}

\usepackage{biblatex}
\addbibresource{bibliography.bib}


\title{A game theoretic model of the behavioural gaming that takes place at the EMS - ED interface}
\author{}
\date{}

\begin{document}
\maketitle

\input{Abstract/main.tex}
\newpage
\tableofcontents
\newpage

% Introduction of the project
\input{Introduction/main.tex}

% Game Theoretic Component
\input{Game_theory_component/main.tex}


\newpage
% Quick representation of the steps of methodology
\input{Methodology/Quick/main.tex}

\newpage
% Proper methodology
\input{Methodology/Proper/main.tex}

% Markov Chains
\input{MarkovChain/markov_chain_model/main.tex}
\newpage
\input{MarkovChain/closed_form_state_probs/main.tex}
\newpage
\input{MarkovChain/expressions_from_pi/main.tex}

\newpage
% Heatmap comparisons
\input{Comparisons/Example_model/main.tex}


\newpage
\input{Miscellaneous/Useful_tikz/main.tex}


% Formulas used
\newpage
\input{Miscellaneous/Formulas/main.tex}

\newpage
\printbibliography[title={References}]

\end{document}


\newpage
% Proper methodology
\documentclass{article}

\usepackage{amsmath} % For writing mathematics (align, split environments etc.)
\usepackage{mathtools}
% \usepackage{amsthm} % For the proof environment
\usepackage{amsfonts} 
\usepackage{geometry}
\usepackage{float}
\usepackage{graphicx}
\usepackage{soul}
\usepackage{indentfirst}
\usepackage{multicol}
\usepackage{tikz}
\usepackage{cancel}

\usetikzlibrary{calc, automata, chains, arrows.meta, math}
\setcounter{MaxMatrixCols}{20}

\usepackage{biblatex}
\addbibresource{bibliography.bib}


\title{A game theoretic model of the behavioural gaming that takes place at the EMS - ED interface}
\author{}
\date{}

\begin{document}
\maketitle

\input{Abstract/main.tex}
\newpage
\tableofcontents
\newpage

% Introduction of the project
\input{Introduction/main.tex}

% Game Theoretic Component
\input{Game_theory_component/main.tex}


\newpage
% Quick representation of the steps of methodology
\input{Methodology/Quick/main.tex}

\newpage
% Proper methodology
\input{Methodology/Proper/main.tex}

% Markov Chains
\input{MarkovChain/markov_chain_model/main.tex}
\newpage
\input{MarkovChain/closed_form_state_probs/main.tex}
\newpage
\input{MarkovChain/expressions_from_pi/main.tex}

\newpage
% Heatmap comparisons
\input{Comparisons/Example_model/main.tex}


\newpage
\input{Miscellaneous/Useful_tikz/main.tex}


% Formulas used
\newpage
\input{Miscellaneous/Formulas/main.tex}

\newpage
\printbibliography[title={References}]

\end{document}


% Markov Chains
\documentclass{article}

\usepackage{amsmath} % For writing mathematics (align, split environments etc.)
\usepackage{mathtools}
% \usepackage{amsthm} % For the proof environment
\usepackage{amsfonts} 
\usepackage{geometry}
\usepackage{float}
\usepackage{graphicx}
\usepackage{soul}
\usepackage{indentfirst}
\usepackage{multicol}
\usepackage{tikz}
\usepackage{cancel}

\usetikzlibrary{calc, automata, chains, arrows.meta, math}
\setcounter{MaxMatrixCols}{20}

\usepackage{biblatex}
\addbibresource{bibliography.bib}


\title{A game theoretic model of the behavioural gaming that takes place at the EMS - ED interface}
\author{}
\date{}

\begin{document}
\maketitle

\input{Abstract/main.tex}
\newpage
\tableofcontents
\newpage

% Introduction of the project
\input{Introduction/main.tex}

% Game Theoretic Component
\input{Game_theory_component/main.tex}


\newpage
% Quick representation of the steps of methodology
\input{Methodology/Quick/main.tex}

\newpage
% Proper methodology
\input{Methodology/Proper/main.tex}

% Markov Chains
\input{MarkovChain/markov_chain_model/main.tex}
\newpage
\input{MarkovChain/closed_form_state_probs/main.tex}
\newpage
\input{MarkovChain/expressions_from_pi/main.tex}

\newpage
% Heatmap comparisons
\input{Comparisons/Example_model/main.tex}


\newpage
\input{Miscellaneous/Useful_tikz/main.tex}


% Formulas used
\newpage
\input{Miscellaneous/Formulas/main.tex}

\newpage
\printbibliography[title={References}]

\end{document}

\newpage
\documentclass{article}

\usepackage{amsmath} % For writing mathematics (align, split environments etc.)
\usepackage{mathtools}
% \usepackage{amsthm} % For the proof environment
\usepackage{amsfonts} 
\usepackage{geometry}
\usepackage{float}
\usepackage{graphicx}
\usepackage{soul}
\usepackage{indentfirst}
\usepackage{multicol}
\usepackage{tikz}
\usepackage{cancel}

\usetikzlibrary{calc, automata, chains, arrows.meta, math}
\setcounter{MaxMatrixCols}{20}

\usepackage{biblatex}
\addbibresource{bibliography.bib}


\title{A game theoretic model of the behavioural gaming that takes place at the EMS - ED interface}
\author{}
\date{}

\begin{document}
\maketitle

\input{Abstract/main.tex}
\newpage
\tableofcontents
\newpage

% Introduction of the project
\input{Introduction/main.tex}

% Game Theoretic Component
\input{Game_theory_component/main.tex}


\newpage
% Quick representation of the steps of methodology
\input{Methodology/Quick/main.tex}

\newpage
% Proper methodology
\input{Methodology/Proper/main.tex}

% Markov Chains
\input{MarkovChain/markov_chain_model/main.tex}
\newpage
\input{MarkovChain/closed_form_state_probs/main.tex}
\newpage
\input{MarkovChain/expressions_from_pi/main.tex}

\newpage
% Heatmap comparisons
\input{Comparisons/Example_model/main.tex}


\newpage
\input{Miscellaneous/Useful_tikz/main.tex}


% Formulas used
\newpage
\input{Miscellaneous/Formulas/main.tex}

\newpage
\printbibliography[title={References}]

\end{document}

\newpage
\documentclass{article}

\usepackage{amsmath} % For writing mathematics (align, split environments etc.)
\usepackage{mathtools}
% \usepackage{amsthm} % For the proof environment
\usepackage{amsfonts} 
\usepackage{geometry}
\usepackage{float}
\usepackage{graphicx}
\usepackage{soul}
\usepackage{indentfirst}
\usepackage{multicol}
\usepackage{tikz}
\usepackage{cancel}

\usetikzlibrary{calc, automata, chains, arrows.meta, math}
\setcounter{MaxMatrixCols}{20}

\usepackage{biblatex}
\addbibresource{bibliography.bib}


\title{A game theoretic model of the behavioural gaming that takes place at the EMS - ED interface}
\author{}
\date{}

\begin{document}
\maketitle

\input{Abstract/main.tex}
\newpage
\tableofcontents
\newpage

% Introduction of the project
\input{Introduction/main.tex}

% Game Theoretic Component
\input{Game_theory_component/main.tex}


\newpage
% Quick representation of the steps of methodology
\input{Methodology/Quick/main.tex}

\newpage
% Proper methodology
\input{Methodology/Proper/main.tex}

% Markov Chains
\input{MarkovChain/markov_chain_model/main.tex}
\newpage
\input{MarkovChain/closed_form_state_probs/main.tex}
\newpage
\input{MarkovChain/expressions_from_pi/main.tex}

\newpage
% Heatmap comparisons
\input{Comparisons/Example_model/main.tex}


\newpage
\input{Miscellaneous/Useful_tikz/main.tex}


% Formulas used
\newpage
\input{Miscellaneous/Formulas/main.tex}

\newpage
\printbibliography[title={References}]

\end{document}


\newpage
% Heatmap comparisons
\documentclass{article}

\usepackage{amsmath} % For writing mathematics (align, split environments etc.)
\usepackage{mathtools}
% \usepackage{amsthm} % For the proof environment
\usepackage{amsfonts} 
\usepackage{geometry}
\usepackage{float}
\usepackage{graphicx}
\usepackage{soul}
\usepackage{indentfirst}
\usepackage{multicol}
\usepackage{tikz}
\usepackage{cancel}

\usetikzlibrary{calc, automata, chains, arrows.meta, math}
\setcounter{MaxMatrixCols}{20}

\usepackage{biblatex}
\addbibresource{bibliography.bib}


\title{A game theoretic model of the behavioural gaming that takes place at the EMS - ED interface}
\author{}
\date{}

\begin{document}
\maketitle

\input{Abstract/main.tex}
\newpage
\tableofcontents
\newpage

% Introduction of the project
\input{Introduction/main.tex}

% Game Theoretic Component
\input{Game_theory_component/main.tex}


\newpage
% Quick representation of the steps of methodology
\input{Methodology/Quick/main.tex}

\newpage
% Proper methodology
\input{Methodology/Proper/main.tex}

% Markov Chains
\input{MarkovChain/markov_chain_model/main.tex}
\newpage
\input{MarkovChain/closed_form_state_probs/main.tex}
\newpage
\input{MarkovChain/expressions_from_pi/main.tex}

\newpage
% Heatmap comparisons
\input{Comparisons/Example_model/main.tex}


\newpage
\input{Miscellaneous/Useful_tikz/main.tex}


% Formulas used
\newpage
\input{Miscellaneous/Formulas/main.tex}

\newpage
\printbibliography[title={References}]

\end{document}



\newpage
\documentclass{article}

\usepackage{amsmath} % For writing mathematics (align, split environments etc.)
\usepackage{mathtools}
% \usepackage{amsthm} % For the proof environment
\usepackage{amsfonts} 
\usepackage{geometry}
\usepackage{float}
\usepackage{graphicx}
\usepackage{soul}
\usepackage{indentfirst}
\usepackage{multicol}
\usepackage{tikz}
\usepackage{cancel}

\usetikzlibrary{calc, automata, chains, arrows.meta, math}
\setcounter{MaxMatrixCols}{20}

\usepackage{biblatex}
\addbibresource{bibliography.bib}


\title{A game theoretic model of the behavioural gaming that takes place at the EMS - ED interface}
\author{}
\date{}

\begin{document}
\maketitle

\input{Abstract/main.tex}
\newpage
\tableofcontents
\newpage

% Introduction of the project
\input{Introduction/main.tex}

% Game Theoretic Component
\input{Game_theory_component/main.tex}


\newpage
% Quick representation of the steps of methodology
\input{Methodology/Quick/main.tex}

\newpage
% Proper methodology
\input{Methodology/Proper/main.tex}

% Markov Chains
\input{MarkovChain/markov_chain_model/main.tex}
\newpage
\input{MarkovChain/closed_form_state_probs/main.tex}
\newpage
\input{MarkovChain/expressions_from_pi/main.tex}

\newpage
% Heatmap comparisons
\input{Comparisons/Example_model/main.tex}


\newpage
\input{Miscellaneous/Useful_tikz/main.tex}


% Formulas used
\newpage
\input{Miscellaneous/Formulas/main.tex}

\newpage
\printbibliography[title={References}]

\end{document}



% Formulas used
\newpage
\documentclass{article}

\usepackage{amsmath} % For writing mathematics (align, split environments etc.)
\usepackage{mathtools}
% \usepackage{amsthm} % For the proof environment
\usepackage{amsfonts} 
\usepackage{geometry}
\usepackage{float}
\usepackage{graphicx}
\usepackage{soul}
\usepackage{indentfirst}
\usepackage{multicol}
\usepackage{tikz}
\usepackage{cancel}

\usetikzlibrary{calc, automata, chains, arrows.meta, math}
\setcounter{MaxMatrixCols}{20}

\usepackage{biblatex}
\addbibresource{bibliography.bib}


\title{A game theoretic model of the behavioural gaming that takes place at the EMS - ED interface}
\author{}
\date{}

\begin{document}
\maketitle

\input{Abstract/main.tex}
\newpage
\tableofcontents
\newpage

% Introduction of the project
\input{Introduction/main.tex}

% Game Theoretic Component
\input{Game_theory_component/main.tex}


\newpage
% Quick representation of the steps of methodology
\input{Methodology/Quick/main.tex}

\newpage
% Proper methodology
\input{Methodology/Proper/main.tex}

% Markov Chains
\input{MarkovChain/markov_chain_model/main.tex}
\newpage
\input{MarkovChain/closed_form_state_probs/main.tex}
\newpage
\input{MarkovChain/expressions_from_pi/main.tex}

\newpage
% Heatmap comparisons
\input{Comparisons/Example_model/main.tex}


\newpage
\input{Miscellaneous/Useful_tikz/main.tex}


% Formulas used
\newpage
\input{Miscellaneous/Formulas/main.tex}

\newpage
\printbibliography[title={References}]

\end{document}


\newpage
\printbibliography[title={References}]

\end{document}



\newpage
\documentclass{article}

\usepackage{amsmath} % For writing mathematics (align, split environments etc.)
\usepackage{mathtools}
% \usepackage{amsthm} % For the proof environment
\usepackage{amsfonts} 
\usepackage{geometry}
\usepackage{float}
\usepackage{graphicx}
\usepackage{soul}
\usepackage{indentfirst}
\usepackage{multicol}
\usepackage{tikz}
\usepackage{cancel}

\usetikzlibrary{calc, automata, chains, arrows.meta, math}
\setcounter{MaxMatrixCols}{20}

\usepackage{biblatex}
\addbibresource{bibliography.bib}


\title{A game theoretic model of the behavioural gaming that takes place at the EMS - ED interface}
\author{}
\date{}

\begin{document}
\maketitle

\documentclass{article}

\usepackage{amsmath} % For writing mathematics (align, split environments etc.)
\usepackage{mathtools}
% \usepackage{amsthm} % For the proof environment
\usepackage{amsfonts} 
\usepackage{geometry}
\usepackage{float}
\usepackage{graphicx}
\usepackage{soul}
\usepackage{indentfirst}
\usepackage{multicol}
\usepackage{tikz}
\usepackage{cancel}

\usetikzlibrary{calc, automata, chains, arrows.meta, math}
\setcounter{MaxMatrixCols}{20}

\usepackage{biblatex}
\addbibresource{bibliography.bib}


\title{A game theoretic model of the behavioural gaming that takes place at the EMS - ED interface}
\author{}
\date{}

\begin{document}
\maketitle

\input{Abstract/main.tex}
\newpage
\tableofcontents
\newpage

% Introduction of the project
\input{Introduction/main.tex}

% Game Theoretic Component
\input{Game_theory_component/main.tex}


\newpage
% Quick representation of the steps of methodology
\input{Methodology/Quick/main.tex}

\newpage
% Proper methodology
\input{Methodology/Proper/main.tex}

% Markov Chains
\input{MarkovChain/markov_chain_model/main.tex}
\newpage
\input{MarkovChain/closed_form_state_probs/main.tex}
\newpage
\input{MarkovChain/expressions_from_pi/main.tex}

\newpage
% Heatmap comparisons
\input{Comparisons/Example_model/main.tex}


\newpage
\input{Miscellaneous/Useful_tikz/main.tex}


% Formulas used
\newpage
\input{Miscellaneous/Formulas/main.tex}

\newpage
\printbibliography[title={References}]

\end{document}

\newpage
\tableofcontents
\newpage

% Introduction of the project
\documentclass{article}

\usepackage{amsmath} % For writing mathematics (align, split environments etc.)
\usepackage{mathtools}
% \usepackage{amsthm} % For the proof environment
\usepackage{amsfonts} 
\usepackage{geometry}
\usepackage{float}
\usepackage{graphicx}
\usepackage{soul}
\usepackage{indentfirst}
\usepackage{multicol}
\usepackage{tikz}
\usepackage{cancel}

\usetikzlibrary{calc, automata, chains, arrows.meta, math}
\setcounter{MaxMatrixCols}{20}

\usepackage{biblatex}
\addbibresource{bibliography.bib}


\title{A game theoretic model of the behavioural gaming that takes place at the EMS - ED interface}
\author{}
\date{}

\begin{document}
\maketitle

\input{Abstract/main.tex}
\newpage
\tableofcontents
\newpage

% Introduction of the project
\input{Introduction/main.tex}

% Game Theoretic Component
\input{Game_theory_component/main.tex}


\newpage
% Quick representation of the steps of methodology
\input{Methodology/Quick/main.tex}

\newpage
% Proper methodology
\input{Methodology/Proper/main.tex}

% Markov Chains
\input{MarkovChain/markov_chain_model/main.tex}
\newpage
\input{MarkovChain/closed_form_state_probs/main.tex}
\newpage
\input{MarkovChain/expressions_from_pi/main.tex}

\newpage
% Heatmap comparisons
\input{Comparisons/Example_model/main.tex}


\newpage
\input{Miscellaneous/Useful_tikz/main.tex}


% Formulas used
\newpage
\input{Miscellaneous/Formulas/main.tex}

\newpage
\printbibliography[title={References}]

\end{document}


% Game Theoretic Component
\documentclass{article}

\usepackage{amsmath} % For writing mathematics (align, split environments etc.)
\usepackage{mathtools}
% \usepackage{amsthm} % For the proof environment
\usepackage{amsfonts} 
\usepackage{geometry}
\usepackage{float}
\usepackage{graphicx}
\usepackage{soul}
\usepackage{indentfirst}
\usepackage{multicol}
\usepackage{tikz}
\usepackage{cancel}

\usetikzlibrary{calc, automata, chains, arrows.meta, math}
\setcounter{MaxMatrixCols}{20}

\usepackage{biblatex}
\addbibresource{bibliography.bib}


\title{A game theoretic model of the behavioural gaming that takes place at the EMS - ED interface}
\author{}
\date{}

\begin{document}
\maketitle

\input{Abstract/main.tex}
\newpage
\tableofcontents
\newpage

% Introduction of the project
\input{Introduction/main.tex}

% Game Theoretic Component
\input{Game_theory_component/main.tex}


\newpage
% Quick representation of the steps of methodology
\input{Methodology/Quick/main.tex}

\newpage
% Proper methodology
\input{Methodology/Proper/main.tex}

% Markov Chains
\input{MarkovChain/markov_chain_model/main.tex}
\newpage
\input{MarkovChain/closed_form_state_probs/main.tex}
\newpage
\input{MarkovChain/expressions_from_pi/main.tex}

\newpage
% Heatmap comparisons
\input{Comparisons/Example_model/main.tex}


\newpage
\input{Miscellaneous/Useful_tikz/main.tex}


% Formulas used
\newpage
\input{Miscellaneous/Formulas/main.tex}

\newpage
\printbibliography[title={References}]

\end{document}



\newpage
% Quick representation of the steps of methodology
\documentclass{article}

\usepackage{amsmath} % For writing mathematics (align, split environments etc.)
\usepackage{mathtools}
% \usepackage{amsthm} % For the proof environment
\usepackage{amsfonts} 
\usepackage{geometry}
\usepackage{float}
\usepackage{graphicx}
\usepackage{soul}
\usepackage{indentfirst}
\usepackage{multicol}
\usepackage{tikz}
\usepackage{cancel}

\usetikzlibrary{calc, automata, chains, arrows.meta, math}
\setcounter{MaxMatrixCols}{20}

\usepackage{biblatex}
\addbibresource{bibliography.bib}


\title{A game theoretic model of the behavioural gaming that takes place at the EMS - ED interface}
\author{}
\date{}

\begin{document}
\maketitle

\input{Abstract/main.tex}
\newpage
\tableofcontents
\newpage

% Introduction of the project
\input{Introduction/main.tex}

% Game Theoretic Component
\input{Game_theory_component/main.tex}


\newpage
% Quick representation of the steps of methodology
\input{Methodology/Quick/main.tex}

\newpage
% Proper methodology
\input{Methodology/Proper/main.tex}

% Markov Chains
\input{MarkovChain/markov_chain_model/main.tex}
\newpage
\input{MarkovChain/closed_form_state_probs/main.tex}
\newpage
\input{MarkovChain/expressions_from_pi/main.tex}

\newpage
% Heatmap comparisons
\input{Comparisons/Example_model/main.tex}


\newpage
\input{Miscellaneous/Useful_tikz/main.tex}


% Formulas used
\newpage
\input{Miscellaneous/Formulas/main.tex}

\newpage
\printbibliography[title={References}]

\end{document}


\newpage
% Proper methodology
\documentclass{article}

\usepackage{amsmath} % For writing mathematics (align, split environments etc.)
\usepackage{mathtools}
% \usepackage{amsthm} % For the proof environment
\usepackage{amsfonts} 
\usepackage{geometry}
\usepackage{float}
\usepackage{graphicx}
\usepackage{soul}
\usepackage{indentfirst}
\usepackage{multicol}
\usepackage{tikz}
\usepackage{cancel}

\usetikzlibrary{calc, automata, chains, arrows.meta, math}
\setcounter{MaxMatrixCols}{20}

\usepackage{biblatex}
\addbibresource{bibliography.bib}


\title{A game theoretic model of the behavioural gaming that takes place at the EMS - ED interface}
\author{}
\date{}

\begin{document}
\maketitle

\input{Abstract/main.tex}
\newpage
\tableofcontents
\newpage

% Introduction of the project
\input{Introduction/main.tex}

% Game Theoretic Component
\input{Game_theory_component/main.tex}


\newpage
% Quick representation of the steps of methodology
\input{Methodology/Quick/main.tex}

\newpage
% Proper methodology
\input{Methodology/Proper/main.tex}

% Markov Chains
\input{MarkovChain/markov_chain_model/main.tex}
\newpage
\input{MarkovChain/closed_form_state_probs/main.tex}
\newpage
\input{MarkovChain/expressions_from_pi/main.tex}

\newpage
% Heatmap comparisons
\input{Comparisons/Example_model/main.tex}


\newpage
\input{Miscellaneous/Useful_tikz/main.tex}


% Formulas used
\newpage
\input{Miscellaneous/Formulas/main.tex}

\newpage
\printbibliography[title={References}]

\end{document}


% Markov Chains
\documentclass{article}

\usepackage{amsmath} % For writing mathematics (align, split environments etc.)
\usepackage{mathtools}
% \usepackage{amsthm} % For the proof environment
\usepackage{amsfonts} 
\usepackage{geometry}
\usepackage{float}
\usepackage{graphicx}
\usepackage{soul}
\usepackage{indentfirst}
\usepackage{multicol}
\usepackage{tikz}
\usepackage{cancel}

\usetikzlibrary{calc, automata, chains, arrows.meta, math}
\setcounter{MaxMatrixCols}{20}

\usepackage{biblatex}
\addbibresource{bibliography.bib}


\title{A game theoretic model of the behavioural gaming that takes place at the EMS - ED interface}
\author{}
\date{}

\begin{document}
\maketitle

\input{Abstract/main.tex}
\newpage
\tableofcontents
\newpage

% Introduction of the project
\input{Introduction/main.tex}

% Game Theoretic Component
\input{Game_theory_component/main.tex}


\newpage
% Quick representation of the steps of methodology
\input{Methodology/Quick/main.tex}

\newpage
% Proper methodology
\input{Methodology/Proper/main.tex}

% Markov Chains
\input{MarkovChain/markov_chain_model/main.tex}
\newpage
\input{MarkovChain/closed_form_state_probs/main.tex}
\newpage
\input{MarkovChain/expressions_from_pi/main.tex}

\newpage
% Heatmap comparisons
\input{Comparisons/Example_model/main.tex}


\newpage
\input{Miscellaneous/Useful_tikz/main.tex}


% Formulas used
\newpage
\input{Miscellaneous/Formulas/main.tex}

\newpage
\printbibliography[title={References}]

\end{document}

\newpage
\documentclass{article}

\usepackage{amsmath} % For writing mathematics (align, split environments etc.)
\usepackage{mathtools}
% \usepackage{amsthm} % For the proof environment
\usepackage{amsfonts} 
\usepackage{geometry}
\usepackage{float}
\usepackage{graphicx}
\usepackage{soul}
\usepackage{indentfirst}
\usepackage{multicol}
\usepackage{tikz}
\usepackage{cancel}

\usetikzlibrary{calc, automata, chains, arrows.meta, math}
\setcounter{MaxMatrixCols}{20}

\usepackage{biblatex}
\addbibresource{bibliography.bib}


\title{A game theoretic model of the behavioural gaming that takes place at the EMS - ED interface}
\author{}
\date{}

\begin{document}
\maketitle

\input{Abstract/main.tex}
\newpage
\tableofcontents
\newpage

% Introduction of the project
\input{Introduction/main.tex}

% Game Theoretic Component
\input{Game_theory_component/main.tex}


\newpage
% Quick representation of the steps of methodology
\input{Methodology/Quick/main.tex}

\newpage
% Proper methodology
\input{Methodology/Proper/main.tex}

% Markov Chains
\input{MarkovChain/markov_chain_model/main.tex}
\newpage
\input{MarkovChain/closed_form_state_probs/main.tex}
\newpage
\input{MarkovChain/expressions_from_pi/main.tex}

\newpage
% Heatmap comparisons
\input{Comparisons/Example_model/main.tex}


\newpage
\input{Miscellaneous/Useful_tikz/main.tex}


% Formulas used
\newpage
\input{Miscellaneous/Formulas/main.tex}

\newpage
\printbibliography[title={References}]

\end{document}

\newpage
\documentclass{article}

\usepackage{amsmath} % For writing mathematics (align, split environments etc.)
\usepackage{mathtools}
% \usepackage{amsthm} % For the proof environment
\usepackage{amsfonts} 
\usepackage{geometry}
\usepackage{float}
\usepackage{graphicx}
\usepackage{soul}
\usepackage{indentfirst}
\usepackage{multicol}
\usepackage{tikz}
\usepackage{cancel}

\usetikzlibrary{calc, automata, chains, arrows.meta, math}
\setcounter{MaxMatrixCols}{20}

\usepackage{biblatex}
\addbibresource{bibliography.bib}


\title{A game theoretic model of the behavioural gaming that takes place at the EMS - ED interface}
\author{}
\date{}

\begin{document}
\maketitle

\input{Abstract/main.tex}
\newpage
\tableofcontents
\newpage

% Introduction of the project
\input{Introduction/main.tex}

% Game Theoretic Component
\input{Game_theory_component/main.tex}


\newpage
% Quick representation of the steps of methodology
\input{Methodology/Quick/main.tex}

\newpage
% Proper methodology
\input{Methodology/Proper/main.tex}

% Markov Chains
\input{MarkovChain/markov_chain_model/main.tex}
\newpage
\input{MarkovChain/closed_form_state_probs/main.tex}
\newpage
\input{MarkovChain/expressions_from_pi/main.tex}

\newpage
% Heatmap comparisons
\input{Comparisons/Example_model/main.tex}


\newpage
\input{Miscellaneous/Useful_tikz/main.tex}


% Formulas used
\newpage
\input{Miscellaneous/Formulas/main.tex}

\newpage
\printbibliography[title={References}]

\end{document}


\newpage
% Heatmap comparisons
\documentclass{article}

\usepackage{amsmath} % For writing mathematics (align, split environments etc.)
\usepackage{mathtools}
% \usepackage{amsthm} % For the proof environment
\usepackage{amsfonts} 
\usepackage{geometry}
\usepackage{float}
\usepackage{graphicx}
\usepackage{soul}
\usepackage{indentfirst}
\usepackage{multicol}
\usepackage{tikz}
\usepackage{cancel}

\usetikzlibrary{calc, automata, chains, arrows.meta, math}
\setcounter{MaxMatrixCols}{20}

\usepackage{biblatex}
\addbibresource{bibliography.bib}


\title{A game theoretic model of the behavioural gaming that takes place at the EMS - ED interface}
\author{}
\date{}

\begin{document}
\maketitle

\input{Abstract/main.tex}
\newpage
\tableofcontents
\newpage

% Introduction of the project
\input{Introduction/main.tex}

% Game Theoretic Component
\input{Game_theory_component/main.tex}


\newpage
% Quick representation of the steps of methodology
\input{Methodology/Quick/main.tex}

\newpage
% Proper methodology
\input{Methodology/Proper/main.tex}

% Markov Chains
\input{MarkovChain/markov_chain_model/main.tex}
\newpage
\input{MarkovChain/closed_form_state_probs/main.tex}
\newpage
\input{MarkovChain/expressions_from_pi/main.tex}

\newpage
% Heatmap comparisons
\input{Comparisons/Example_model/main.tex}


\newpage
\input{Miscellaneous/Useful_tikz/main.tex}


% Formulas used
\newpage
\input{Miscellaneous/Formulas/main.tex}

\newpage
\printbibliography[title={References}]

\end{document}



\newpage
\documentclass{article}

\usepackage{amsmath} % For writing mathematics (align, split environments etc.)
\usepackage{mathtools}
% \usepackage{amsthm} % For the proof environment
\usepackage{amsfonts} 
\usepackage{geometry}
\usepackage{float}
\usepackage{graphicx}
\usepackage{soul}
\usepackage{indentfirst}
\usepackage{multicol}
\usepackage{tikz}
\usepackage{cancel}

\usetikzlibrary{calc, automata, chains, arrows.meta, math}
\setcounter{MaxMatrixCols}{20}

\usepackage{biblatex}
\addbibresource{bibliography.bib}


\title{A game theoretic model of the behavioural gaming that takes place at the EMS - ED interface}
\author{}
\date{}

\begin{document}
\maketitle

\input{Abstract/main.tex}
\newpage
\tableofcontents
\newpage

% Introduction of the project
\input{Introduction/main.tex}

% Game Theoretic Component
\input{Game_theory_component/main.tex}


\newpage
% Quick representation of the steps of methodology
\input{Methodology/Quick/main.tex}

\newpage
% Proper methodology
\input{Methodology/Proper/main.tex}

% Markov Chains
\input{MarkovChain/markov_chain_model/main.tex}
\newpage
\input{MarkovChain/closed_form_state_probs/main.tex}
\newpage
\input{MarkovChain/expressions_from_pi/main.tex}

\newpage
% Heatmap comparisons
\input{Comparisons/Example_model/main.tex}


\newpage
\input{Miscellaneous/Useful_tikz/main.tex}


% Formulas used
\newpage
\input{Miscellaneous/Formulas/main.tex}

\newpage
\printbibliography[title={References}]

\end{document}



% Formulas used
\newpage
\documentclass{article}

\usepackage{amsmath} % For writing mathematics (align, split environments etc.)
\usepackage{mathtools}
% \usepackage{amsthm} % For the proof environment
\usepackage{amsfonts} 
\usepackage{geometry}
\usepackage{float}
\usepackage{graphicx}
\usepackage{soul}
\usepackage{indentfirst}
\usepackage{multicol}
\usepackage{tikz}
\usepackage{cancel}

\usetikzlibrary{calc, automata, chains, arrows.meta, math}
\setcounter{MaxMatrixCols}{20}

\usepackage{biblatex}
\addbibresource{bibliography.bib}


\title{A game theoretic model of the behavioural gaming that takes place at the EMS - ED interface}
\author{}
\date{}

\begin{document}
\maketitle

\input{Abstract/main.tex}
\newpage
\tableofcontents
\newpage

% Introduction of the project
\input{Introduction/main.tex}

% Game Theoretic Component
\input{Game_theory_component/main.tex}


\newpage
% Quick representation of the steps of methodology
\input{Methodology/Quick/main.tex}

\newpage
% Proper methodology
\input{Methodology/Proper/main.tex}

% Markov Chains
\input{MarkovChain/markov_chain_model/main.tex}
\newpage
\input{MarkovChain/closed_form_state_probs/main.tex}
\newpage
\input{MarkovChain/expressions_from_pi/main.tex}

\newpage
% Heatmap comparisons
\input{Comparisons/Example_model/main.tex}


\newpage
\input{Miscellaneous/Useful_tikz/main.tex}


% Formulas used
\newpage
\input{Miscellaneous/Formulas/main.tex}

\newpage
\printbibliography[title={References}]

\end{document}


\newpage
\printbibliography[title={References}]

\end{document}



% Formulas used
\newpage
\documentclass{article}

\usepackage{amsmath} % For writing mathematics (align, split environments etc.)
\usepackage{mathtools}
% \usepackage{amsthm} % For the proof environment
\usepackage{amsfonts} 
\usepackage{geometry}
\usepackage{float}
\usepackage{graphicx}
\usepackage{soul}
\usepackage{indentfirst}
\usepackage{multicol}
\usepackage{tikz}
\usepackage{cancel}

\usetikzlibrary{calc, automata, chains, arrows.meta, math}
\setcounter{MaxMatrixCols}{20}

\usepackage{biblatex}
\addbibresource{bibliography.bib}


\title{A game theoretic model of the behavioural gaming that takes place at the EMS - ED interface}
\author{}
\date{}

\begin{document}
\maketitle

\documentclass{article}

\usepackage{amsmath} % For writing mathematics (align, split environments etc.)
\usepackage{mathtools}
% \usepackage{amsthm} % For the proof environment
\usepackage{amsfonts} 
\usepackage{geometry}
\usepackage{float}
\usepackage{graphicx}
\usepackage{soul}
\usepackage{indentfirst}
\usepackage{multicol}
\usepackage{tikz}
\usepackage{cancel}

\usetikzlibrary{calc, automata, chains, arrows.meta, math}
\setcounter{MaxMatrixCols}{20}

\usepackage{biblatex}
\addbibresource{bibliography.bib}


\title{A game theoretic model of the behavioural gaming that takes place at the EMS - ED interface}
\author{}
\date{}

\begin{document}
\maketitle

\input{Abstract/main.tex}
\newpage
\tableofcontents
\newpage

% Introduction of the project
\input{Introduction/main.tex}

% Game Theoretic Component
\input{Game_theory_component/main.tex}


\newpage
% Quick representation of the steps of methodology
\input{Methodology/Quick/main.tex}

\newpage
% Proper methodology
\input{Methodology/Proper/main.tex}

% Markov Chains
\input{MarkovChain/markov_chain_model/main.tex}
\newpage
\input{MarkovChain/closed_form_state_probs/main.tex}
\newpage
\input{MarkovChain/expressions_from_pi/main.tex}

\newpage
% Heatmap comparisons
\input{Comparisons/Example_model/main.tex}


\newpage
\input{Miscellaneous/Useful_tikz/main.tex}


% Formulas used
\newpage
\input{Miscellaneous/Formulas/main.tex}

\newpage
\printbibliography[title={References}]

\end{document}

\newpage
\tableofcontents
\newpage

% Introduction of the project
\documentclass{article}

\usepackage{amsmath} % For writing mathematics (align, split environments etc.)
\usepackage{mathtools}
% \usepackage{amsthm} % For the proof environment
\usepackage{amsfonts} 
\usepackage{geometry}
\usepackage{float}
\usepackage{graphicx}
\usepackage{soul}
\usepackage{indentfirst}
\usepackage{multicol}
\usepackage{tikz}
\usepackage{cancel}

\usetikzlibrary{calc, automata, chains, arrows.meta, math}
\setcounter{MaxMatrixCols}{20}

\usepackage{biblatex}
\addbibresource{bibliography.bib}


\title{A game theoretic model of the behavioural gaming that takes place at the EMS - ED interface}
\author{}
\date{}

\begin{document}
\maketitle

\input{Abstract/main.tex}
\newpage
\tableofcontents
\newpage

% Introduction of the project
\input{Introduction/main.tex}

% Game Theoretic Component
\input{Game_theory_component/main.tex}


\newpage
% Quick representation of the steps of methodology
\input{Methodology/Quick/main.tex}

\newpage
% Proper methodology
\input{Methodology/Proper/main.tex}

% Markov Chains
\input{MarkovChain/markov_chain_model/main.tex}
\newpage
\input{MarkovChain/closed_form_state_probs/main.tex}
\newpage
\input{MarkovChain/expressions_from_pi/main.tex}

\newpage
% Heatmap comparisons
\input{Comparisons/Example_model/main.tex}


\newpage
\input{Miscellaneous/Useful_tikz/main.tex}


% Formulas used
\newpage
\input{Miscellaneous/Formulas/main.tex}

\newpage
\printbibliography[title={References}]

\end{document}


% Game Theoretic Component
\documentclass{article}

\usepackage{amsmath} % For writing mathematics (align, split environments etc.)
\usepackage{mathtools}
% \usepackage{amsthm} % For the proof environment
\usepackage{amsfonts} 
\usepackage{geometry}
\usepackage{float}
\usepackage{graphicx}
\usepackage{soul}
\usepackage{indentfirst}
\usepackage{multicol}
\usepackage{tikz}
\usepackage{cancel}

\usetikzlibrary{calc, automata, chains, arrows.meta, math}
\setcounter{MaxMatrixCols}{20}

\usepackage{biblatex}
\addbibresource{bibliography.bib}


\title{A game theoretic model of the behavioural gaming that takes place at the EMS - ED interface}
\author{}
\date{}

\begin{document}
\maketitle

\input{Abstract/main.tex}
\newpage
\tableofcontents
\newpage

% Introduction of the project
\input{Introduction/main.tex}

% Game Theoretic Component
\input{Game_theory_component/main.tex}


\newpage
% Quick representation of the steps of methodology
\input{Methodology/Quick/main.tex}

\newpage
% Proper methodology
\input{Methodology/Proper/main.tex}

% Markov Chains
\input{MarkovChain/markov_chain_model/main.tex}
\newpage
\input{MarkovChain/closed_form_state_probs/main.tex}
\newpage
\input{MarkovChain/expressions_from_pi/main.tex}

\newpage
% Heatmap comparisons
\input{Comparisons/Example_model/main.tex}


\newpage
\input{Miscellaneous/Useful_tikz/main.tex}


% Formulas used
\newpage
\input{Miscellaneous/Formulas/main.tex}

\newpage
\printbibliography[title={References}]

\end{document}



\newpage
% Quick representation of the steps of methodology
\documentclass{article}

\usepackage{amsmath} % For writing mathematics (align, split environments etc.)
\usepackage{mathtools}
% \usepackage{amsthm} % For the proof environment
\usepackage{amsfonts} 
\usepackage{geometry}
\usepackage{float}
\usepackage{graphicx}
\usepackage{soul}
\usepackage{indentfirst}
\usepackage{multicol}
\usepackage{tikz}
\usepackage{cancel}

\usetikzlibrary{calc, automata, chains, arrows.meta, math}
\setcounter{MaxMatrixCols}{20}

\usepackage{biblatex}
\addbibresource{bibliography.bib}


\title{A game theoretic model of the behavioural gaming that takes place at the EMS - ED interface}
\author{}
\date{}

\begin{document}
\maketitle

\input{Abstract/main.tex}
\newpage
\tableofcontents
\newpage

% Introduction of the project
\input{Introduction/main.tex}

% Game Theoretic Component
\input{Game_theory_component/main.tex}


\newpage
% Quick representation of the steps of methodology
\input{Methodology/Quick/main.tex}

\newpage
% Proper methodology
\input{Methodology/Proper/main.tex}

% Markov Chains
\input{MarkovChain/markov_chain_model/main.tex}
\newpage
\input{MarkovChain/closed_form_state_probs/main.tex}
\newpage
\input{MarkovChain/expressions_from_pi/main.tex}

\newpage
% Heatmap comparisons
\input{Comparisons/Example_model/main.tex}


\newpage
\input{Miscellaneous/Useful_tikz/main.tex}


% Formulas used
\newpage
\input{Miscellaneous/Formulas/main.tex}

\newpage
\printbibliography[title={References}]

\end{document}


\newpage
% Proper methodology
\documentclass{article}

\usepackage{amsmath} % For writing mathematics (align, split environments etc.)
\usepackage{mathtools}
% \usepackage{amsthm} % For the proof environment
\usepackage{amsfonts} 
\usepackage{geometry}
\usepackage{float}
\usepackage{graphicx}
\usepackage{soul}
\usepackage{indentfirst}
\usepackage{multicol}
\usepackage{tikz}
\usepackage{cancel}

\usetikzlibrary{calc, automata, chains, arrows.meta, math}
\setcounter{MaxMatrixCols}{20}

\usepackage{biblatex}
\addbibresource{bibliography.bib}


\title{A game theoretic model of the behavioural gaming that takes place at the EMS - ED interface}
\author{}
\date{}

\begin{document}
\maketitle

\input{Abstract/main.tex}
\newpage
\tableofcontents
\newpage

% Introduction of the project
\input{Introduction/main.tex}

% Game Theoretic Component
\input{Game_theory_component/main.tex}


\newpage
% Quick representation of the steps of methodology
\input{Methodology/Quick/main.tex}

\newpage
% Proper methodology
\input{Methodology/Proper/main.tex}

% Markov Chains
\input{MarkovChain/markov_chain_model/main.tex}
\newpage
\input{MarkovChain/closed_form_state_probs/main.tex}
\newpage
\input{MarkovChain/expressions_from_pi/main.tex}

\newpage
% Heatmap comparisons
\input{Comparisons/Example_model/main.tex}


\newpage
\input{Miscellaneous/Useful_tikz/main.tex}


% Formulas used
\newpage
\input{Miscellaneous/Formulas/main.tex}

\newpage
\printbibliography[title={References}]

\end{document}


% Markov Chains
\documentclass{article}

\usepackage{amsmath} % For writing mathematics (align, split environments etc.)
\usepackage{mathtools}
% \usepackage{amsthm} % For the proof environment
\usepackage{amsfonts} 
\usepackage{geometry}
\usepackage{float}
\usepackage{graphicx}
\usepackage{soul}
\usepackage{indentfirst}
\usepackage{multicol}
\usepackage{tikz}
\usepackage{cancel}

\usetikzlibrary{calc, automata, chains, arrows.meta, math}
\setcounter{MaxMatrixCols}{20}

\usepackage{biblatex}
\addbibresource{bibliography.bib}


\title{A game theoretic model of the behavioural gaming that takes place at the EMS - ED interface}
\author{}
\date{}

\begin{document}
\maketitle

\input{Abstract/main.tex}
\newpage
\tableofcontents
\newpage

% Introduction of the project
\input{Introduction/main.tex}

% Game Theoretic Component
\input{Game_theory_component/main.tex}


\newpage
% Quick representation of the steps of methodology
\input{Methodology/Quick/main.tex}

\newpage
% Proper methodology
\input{Methodology/Proper/main.tex}

% Markov Chains
\input{MarkovChain/markov_chain_model/main.tex}
\newpage
\input{MarkovChain/closed_form_state_probs/main.tex}
\newpage
\input{MarkovChain/expressions_from_pi/main.tex}

\newpage
% Heatmap comparisons
\input{Comparisons/Example_model/main.tex}


\newpage
\input{Miscellaneous/Useful_tikz/main.tex}


% Formulas used
\newpage
\input{Miscellaneous/Formulas/main.tex}

\newpage
\printbibliography[title={References}]

\end{document}

\newpage
\documentclass{article}

\usepackage{amsmath} % For writing mathematics (align, split environments etc.)
\usepackage{mathtools}
% \usepackage{amsthm} % For the proof environment
\usepackage{amsfonts} 
\usepackage{geometry}
\usepackage{float}
\usepackage{graphicx}
\usepackage{soul}
\usepackage{indentfirst}
\usepackage{multicol}
\usepackage{tikz}
\usepackage{cancel}

\usetikzlibrary{calc, automata, chains, arrows.meta, math}
\setcounter{MaxMatrixCols}{20}

\usepackage{biblatex}
\addbibresource{bibliography.bib}


\title{A game theoretic model of the behavioural gaming that takes place at the EMS - ED interface}
\author{}
\date{}

\begin{document}
\maketitle

\input{Abstract/main.tex}
\newpage
\tableofcontents
\newpage

% Introduction of the project
\input{Introduction/main.tex}

% Game Theoretic Component
\input{Game_theory_component/main.tex}


\newpage
% Quick representation of the steps of methodology
\input{Methodology/Quick/main.tex}

\newpage
% Proper methodology
\input{Methodology/Proper/main.tex}

% Markov Chains
\input{MarkovChain/markov_chain_model/main.tex}
\newpage
\input{MarkovChain/closed_form_state_probs/main.tex}
\newpage
\input{MarkovChain/expressions_from_pi/main.tex}

\newpage
% Heatmap comparisons
\input{Comparisons/Example_model/main.tex}


\newpage
\input{Miscellaneous/Useful_tikz/main.tex}


% Formulas used
\newpage
\input{Miscellaneous/Formulas/main.tex}

\newpage
\printbibliography[title={References}]

\end{document}

\newpage
\documentclass{article}

\usepackage{amsmath} % For writing mathematics (align, split environments etc.)
\usepackage{mathtools}
% \usepackage{amsthm} % For the proof environment
\usepackage{amsfonts} 
\usepackage{geometry}
\usepackage{float}
\usepackage{graphicx}
\usepackage{soul}
\usepackage{indentfirst}
\usepackage{multicol}
\usepackage{tikz}
\usepackage{cancel}

\usetikzlibrary{calc, automata, chains, arrows.meta, math}
\setcounter{MaxMatrixCols}{20}

\usepackage{biblatex}
\addbibresource{bibliography.bib}


\title{A game theoretic model of the behavioural gaming that takes place at the EMS - ED interface}
\author{}
\date{}

\begin{document}
\maketitle

\input{Abstract/main.tex}
\newpage
\tableofcontents
\newpage

% Introduction of the project
\input{Introduction/main.tex}

% Game Theoretic Component
\input{Game_theory_component/main.tex}


\newpage
% Quick representation of the steps of methodology
\input{Methodology/Quick/main.tex}

\newpage
% Proper methodology
\input{Methodology/Proper/main.tex}

% Markov Chains
\input{MarkovChain/markov_chain_model/main.tex}
\newpage
\input{MarkovChain/closed_form_state_probs/main.tex}
\newpage
\input{MarkovChain/expressions_from_pi/main.tex}

\newpage
% Heatmap comparisons
\input{Comparisons/Example_model/main.tex}


\newpage
\input{Miscellaneous/Useful_tikz/main.tex}


% Formulas used
\newpage
\input{Miscellaneous/Formulas/main.tex}

\newpage
\printbibliography[title={References}]

\end{document}


\newpage
% Heatmap comparisons
\documentclass{article}

\usepackage{amsmath} % For writing mathematics (align, split environments etc.)
\usepackage{mathtools}
% \usepackage{amsthm} % For the proof environment
\usepackage{amsfonts} 
\usepackage{geometry}
\usepackage{float}
\usepackage{graphicx}
\usepackage{soul}
\usepackage{indentfirst}
\usepackage{multicol}
\usepackage{tikz}
\usepackage{cancel}

\usetikzlibrary{calc, automata, chains, arrows.meta, math}
\setcounter{MaxMatrixCols}{20}

\usepackage{biblatex}
\addbibresource{bibliography.bib}


\title{A game theoretic model of the behavioural gaming that takes place at the EMS - ED interface}
\author{}
\date{}

\begin{document}
\maketitle

\input{Abstract/main.tex}
\newpage
\tableofcontents
\newpage

% Introduction of the project
\input{Introduction/main.tex}

% Game Theoretic Component
\input{Game_theory_component/main.tex}


\newpage
% Quick representation of the steps of methodology
\input{Methodology/Quick/main.tex}

\newpage
% Proper methodology
\input{Methodology/Proper/main.tex}

% Markov Chains
\input{MarkovChain/markov_chain_model/main.tex}
\newpage
\input{MarkovChain/closed_form_state_probs/main.tex}
\newpage
\input{MarkovChain/expressions_from_pi/main.tex}

\newpage
% Heatmap comparisons
\input{Comparisons/Example_model/main.tex}


\newpage
\input{Miscellaneous/Useful_tikz/main.tex}


% Formulas used
\newpage
\input{Miscellaneous/Formulas/main.tex}

\newpage
\printbibliography[title={References}]

\end{document}



\newpage
\documentclass{article}

\usepackage{amsmath} % For writing mathematics (align, split environments etc.)
\usepackage{mathtools}
% \usepackage{amsthm} % For the proof environment
\usepackage{amsfonts} 
\usepackage{geometry}
\usepackage{float}
\usepackage{graphicx}
\usepackage{soul}
\usepackage{indentfirst}
\usepackage{multicol}
\usepackage{tikz}
\usepackage{cancel}

\usetikzlibrary{calc, automata, chains, arrows.meta, math}
\setcounter{MaxMatrixCols}{20}

\usepackage{biblatex}
\addbibresource{bibliography.bib}


\title{A game theoretic model of the behavioural gaming that takes place at the EMS - ED interface}
\author{}
\date{}

\begin{document}
\maketitle

\input{Abstract/main.tex}
\newpage
\tableofcontents
\newpage

% Introduction of the project
\input{Introduction/main.tex}

% Game Theoretic Component
\input{Game_theory_component/main.tex}


\newpage
% Quick representation of the steps of methodology
\input{Methodology/Quick/main.tex}

\newpage
% Proper methodology
\input{Methodology/Proper/main.tex}

% Markov Chains
\input{MarkovChain/markov_chain_model/main.tex}
\newpage
\input{MarkovChain/closed_form_state_probs/main.tex}
\newpage
\input{MarkovChain/expressions_from_pi/main.tex}

\newpage
% Heatmap comparisons
\input{Comparisons/Example_model/main.tex}


\newpage
\input{Miscellaneous/Useful_tikz/main.tex}


% Formulas used
\newpage
\input{Miscellaneous/Formulas/main.tex}

\newpage
\printbibliography[title={References}]

\end{document}



% Formulas used
\newpage
\documentclass{article}

\usepackage{amsmath} % For writing mathematics (align, split environments etc.)
\usepackage{mathtools}
% \usepackage{amsthm} % For the proof environment
\usepackage{amsfonts} 
\usepackage{geometry}
\usepackage{float}
\usepackage{graphicx}
\usepackage{soul}
\usepackage{indentfirst}
\usepackage{multicol}
\usepackage{tikz}
\usepackage{cancel}

\usetikzlibrary{calc, automata, chains, arrows.meta, math}
\setcounter{MaxMatrixCols}{20}

\usepackage{biblatex}
\addbibresource{bibliography.bib}


\title{A game theoretic model of the behavioural gaming that takes place at the EMS - ED interface}
\author{}
\date{}

\begin{document}
\maketitle

\input{Abstract/main.tex}
\newpage
\tableofcontents
\newpage

% Introduction of the project
\input{Introduction/main.tex}

% Game Theoretic Component
\input{Game_theory_component/main.tex}


\newpage
% Quick representation of the steps of methodology
\input{Methodology/Quick/main.tex}

\newpage
% Proper methodology
\input{Methodology/Proper/main.tex}

% Markov Chains
\input{MarkovChain/markov_chain_model/main.tex}
\newpage
\input{MarkovChain/closed_form_state_probs/main.tex}
\newpage
\input{MarkovChain/expressions_from_pi/main.tex}

\newpage
% Heatmap comparisons
\input{Comparisons/Example_model/main.tex}


\newpage
\input{Miscellaneous/Useful_tikz/main.tex}


% Formulas used
\newpage
\input{Miscellaneous/Formulas/main.tex}

\newpage
\printbibliography[title={References}]

\end{document}


\newpage
\printbibliography[title={References}]

\end{document}


\newpage
\printbibliography[title={References}]

\end{document}

    \caption{Markov chains: number of servers=4} 
    \label{Model_mini}
\end{figure}

In addition to the Markov chain model a simulation model has also been built based 
on the same parameters. 
Comparing the results of the Markov model and the equivalent simulation model the 
resultant plots arose.

The heatmaps in figure \ref{Heatmap_mini} represent the state probabilities for 
the Markov chain model, the simulation model and the difference between the two. 
Each pixel of the heatmap corresponds to the equivalent state of figure \ref{Model_mini} 
and represents the probability of being at that state in any particular moment of time.

It can be observed that both Markov chain and simulation models' state probabilities 
vary from 5\% to 25\% and that states \( (0, 1) \) and \( (0, 2) \) are the most 
visited ones. 
Looking at the differences' heatmap, one may identify that the differences between 
the two are minimal.

\newpage

\begin{figure}[h]
    \includegraphics[width=\linewidth]{Comparisons/Example_model/Heatmap/main.pdf}
    \caption{Heatmaps of Simulation, Markov chains and differences of the two}
    \label{Heatmap_mini}
\end{figure}



